%%%%%%%%%%%%%%%%%%%%%%%%%%%%%%%%%%%%%%%%%%%%%%%%%%%%%%%%%%%%%%%%%%%%%%%%%%
%% Copyright: William Stein, 2007.
%%%%%%%%%%%%%%%%%%%%%%%%%%%%%%%%%%%%%%%%%%%%%%%%%%%%%%%%%%%%%%%%%%%%%%%%%%

\documentclass[11pt]{book}

\usepackage{hyperref}
\usepackage{graphicx}
\usepackage[all]{xy}
\usepackage{sage}
\usepackage{tikz}

\bibliographystyle{amsalpha}
\input{macros}
\hoffset=-0.05\textwidth
\textwidth=1.1\textwidth
\voffset=-0.05\textheight
\textheight=1.1\textheight
\makeindex
\renewcommand{\edit}[1]{}
\newcommand{\zmod}[1]{{\Z/#1\Z{}}}

\title{\Huge\bf\sc Adeles}
\author{William Stein}

\begin{document}
\maketitle
\newpage
\tableofcontents
\newpage

%%%%%%%%%%%%%%%%%%%%%%%%%%%%%%%%%%%%%%%%%%%%%%%%%%%%%%%%%%%%%%%%%%%%%%%%%%%%%%%
%%%%%%%%%%%%%%%%%%%%%%%%%%%% START PREFACE %%%%%%%%%%%%%%%%%%%%%%%%%%%%%%%%%%%%
%%%%%%%%%%%%%%%%%%%%%%%%%%%%%%%%%%%%%%%%%%%%%%%%%%%%%%%%%%%%%%%%%%%%%%%%%%%%%%%
\begin{ch}
\chapter*{Preface}
%TODO write this
???????????????????

????????

?????????????

???????????????????????

???????????????????

??????????????????????????????
\end{ch}
%%%%%%%%%%%%%%%%%%%%%%%%%%%%%%%%%%%%%%%%%%%%%%%%%%%%%%%%%%%%%%%%%%%%%%%%%%%%%%%
%%%%%%%%%%%%%%%%%%%%%%%%%%%% END PREFACE %%%%%%%%%%%%%%%%%%%%%%%%%%%%%%%%%%%%%%
%%%%%%%%%%%%%%%%%%%%%%%%%%%%%%%%%%%%%%%%%%%%%%%%%%%%%%%%%%%%%%%%%%%%%%%%%%%%%%%





















%%%%%%%%%%%%%%%%%%%%%%%%%%%%%%%%%%%%%%%%%%%%%%%%%%%%%%%%%%%%%%%%%%%%%%%%%%%%%%%
%%%%%%%%%%%%%%%%%%%%%%%%%%%% START INTRO %%%%%%%%%%%%%%%%%%%%%%%%%%%%%%%%%%%%%%
%%%%%%%%%%%%%%%%%%%%%%%%%%%%%%%%%%%%%%%%%%%%%%%%%%%%%%%%%%%%%%%%%%%%%%%%%%%%%%%
\begin{ch}
\chapter{Introduction}
%TODO write this
???????????????????

????????

?????????????

???????????????????????

???????????????????

??????????????????????????????
\end{ch}
%%%%%%%%%%%%%%%%%%%%%%%%%%%%%%%%%%%%%%%%%%%%%%%%%%%%%%%%%%%%%%%%%%%%%%%%%%%%%%%
%%%%%%%%%%%%%%%%%%%%%%%%%%%% END INTRO %%%%%%%%%%%%%%%%%%%%%%%%%%%%%%%%%%%%%%%%
%%%%%%%%%%%%%%%%%%%%%%%%%%%%%%%%%%%%%%%%%%%%%%%%%%%%%%%%%%%%%%%%%%%%%%%%%%%%%%%

























%%%%%%%%%%%%%%%%%%%%%%%%%%%%%%%%%%%%%%%%%%%%%%%%%%%%%%%%%%%%%%%%%%%%%%%%%%%%%%%
%%%%%%%%%%%%%%%%%%%%%%%%%%%% START VALUATIONS %%%%%%%%%%%%%%%%%%%%%%%%%%%%%%%%%
%%%%%%%%%%%%%%%%%%%%%%%%%%%%%%%%%%%%%%%%%%%%%%%%%%%%%%%%%%%%%%%%%%%%%%%%%%%%%%%
\begin{ch}
\chapter{Valuations}
The rest of this book is a partial rewrite of \cite{cassels:global}
meant to make it more accessible.  I have attempted to add examples
and details of the implicit exercises and remarks that are left to the
reader.

\section{Valuations}

\begin{definition}[Valuation]
A \defn{valuation} $\absspc{}$ on a field $K$ is a function
defined on $K$ with values in $\R_{\geq 0}$ satisfying
the following axioms:
\begin{enumerate}
\item[(1)] $\abs{a} = 0$ if and only if $a = 0$,
\item[(2)] $\abs{ab}=\abs{a}\abs{b}$, and
\item[(3)] there is a constant $C\geq 1$ such that
$\abs{1+a}\leq C$ whenever $\abs{a}\leq 1$.
\end{enumerate}
\end{definition}

The \defn{trivial valuation} is the valuation for which
$\abs{a}=1$ for all $a\neq 0$.  We will often tacitly
exclude the trivial valuation from consideration.

From (2) we have
$$
  \abs{1} = \abs{1}\cdot \abs{1},
$$
so $\abs{1} = 1$ by (1).
If $w\in K$ and $w^n=1$, then $|w|=1$ by (2).
In particular, the only valuation of a finite field
is the trivial one.    The same argument shows that $|-1|=|1|$,
so
$$
  |-a| = |a|\qquad \text{all }a \in K.
$$

\begin{definition}[Equivalent]
Two valuations $\absspc{}_1$ and $\absspc{}_2$ on the
same field are \defn{equivalent}\i{valuation!equivalence of}
if there exists $c>0$ such
that $$\abs{a}_2 = \abs{a}_1^c$$
for all $a\in K$.
\end{definition}
Note that if $\absspc{}_1$ is a valuation, then
$\absspc{}_2=\absspc{}_1^c$ is also a valuation.
Also, equivalence of valuations is an equivalence relation.

If $\absspc{}$ is a valuation and $C>1$ is the constant from Axiom
(3), then there is a $c>0$ such that $C^c=2$ (i.e.,
$c=\log(2)/\log(C)$).  Then we can take $2$ as constant for the
equivalent valuation $\absspc{}^c$.  Thus every valuation is
equivalent to a valuation with $C=2$. Note that if $C=1$, e.g.,
if $\absspc{}$ is the trivial valuation, then we could
simply take $C=2$ in Axiom (3).
\begin{proposition}\iprop{triangle inequality}
Suppose $\absspc{}$ is a valuation with $C\leq 2$.
Then for all $a, b\in K$ we have
\begin{equation}\label{val3p}
  |a + b| \leq |a| + |b|\qquad \text{(triangle inequality)}.
\end{equation}
\end{proposition}
\begin{proof}
Suppose $a_1, a_2\in K$ with $|a_1|\geq|a_2|$.  Then $a=a_2/a_1$
satisfies $|a|\leq 1$.  By Axiom (3) we have $|1+a|\leq 2$, so
multiplying by $a_1$ we see that
$$|a_1+ a_2|\leq 2|a_1| = 2\cdot\max\{|a_1|,|a_2|\}.$$
Also we have
$$|a_1+ a_2 + a_3 + a_4|\leq 2\cdot\max\{|a_1+a_2|,|a_3+a_4|\}
   \leq 4\cdot \max\{|a_1|,|a_2|,|a_3|,|a_4|\},
$$
and inductively we have for any $r>0$ that
$$|a_1 + a_2 + \cdots  + a_{2^r}| \leq 2^r\cdot\max{|a_j|}.$$
If $n$ is any positive integer, let $r$ be such
that $2^{r-1}\leq n\leq 2^r$. Thenn
$$|a_1 + a_2 + \cdots + a_{n}| \leq 2^r\cdot \max\{|a_j|\}
 \leq 2n\cdot \max\{|a_j|\},$$
since $2^r\leq 2n$.  In particular,
\begin{equation}\label{eqn:absn}
  |n| \leq 2n\cdot |1| = 2n \qquad\text{(for $n>0$)}.
\end{equation}
Applying (\ref{eqn:absn}) to $\ds\abs{\binom{n}{j}}$ and using
the binomial expansion, we have for any $a,b\in K$ that
\begin{align*}
|a+b|^n &= \abs{\sum_{j=0}^n \binom{n}{j} a^j b^{n-j}}\\
   &\leq  2(n+1)\max_j\left\{ \abs{\binom{n}{j}} \abs{a}^j\abs{b}^{n-j}\right\}\\
   &\leq  2(n+1)\max_j\left\{ 2 \binom{n}{j} \abs{a}^j\abs{b}^{n-j}\right\}\\
   &\leq  4(n+1)\max_j\left\{ \binom{n}{j} \abs{a}^j\abs{b}^{n-j}\right\}\\
   &\leq  4(n+1)(\abs{a}+\abs{b})^n.
\end{align*}
Now take $n$th roots of both sides to obtain
$$
|a+b| \leq \sqrt[n]{4(n+1)}\cdot (|a| + |b|).
$$
We have by elementary calculus that
$$
  \lim_{n\to \infty} \sqrt[n]{4(n+1)} = 1,
$$ so $|a+b| \leq |a|+|b|$.
(The ``elementary calculus'': We instead prove that $\sqrt[n]{n}\to 1$, since
the argument is the same and the notation is simpler.  First, for any
$n\geq 1$ we have $\sqrt[n]{n}\geq 1$, since upon taking $n$th powers
this is equivalent to $n\geq 1^n$, which is true by hypothesis.
Second, suppose there is an $\eps>0$ such that $\sqrt[n]{n}\geq
1+\eps$ for all $n\geq 1$.  Then taking logs of boths sides we see
that $\frac{1}{n}\log(n)\geq \log(1+\eps) > 0$.  But
 $\log(n)/n\to 0$, so there is no such $\eps$.  Thus
$\sqrt[n]{n}\to 1$ as $n\to \infty$.)
\end{proof}
Note that Axioms (1), (2) and Equation (\ref{val3p}) imply Axiom (3)
with $C=2$.  We take Axiom (3) instead of Equation (\ref{val3p}) for
the technical reason that we will want to call the square of the
absolute value of the complex numbers a valuation.

\begin{lemma}\ilem{$\Bigl||a| - |b|\Bigr| \leq \abs{a-b}$}
Suppose $a, b \in K$, and $\absspc{}$ is a valuation on $K$
with $C\leq 2$.
Then
$$
  \Bigl||a| - |b|\Bigr| \leq \abs{a-b}.
$$
(Here the big absolute value on the outside of the left-hand
side of the inequality is the usual absolute value on
real numbers, but the other absolute values are a valuation
on an arbitrary field~$K$.)
\end{lemma}
\begin{proof}
We have
$$|a| = |b + (a-b)| \leq |b| + |a-b|,$$
so $|a|-|b|\leq \abs{a-b}$.  The same argument
with $a$ and $b$ swapped implies that
$|b|-|a|\leq \abs{a-b}$, which proves the lemma.
\end{proof}


\section{Types of Valuations}
We define two important properties of valuations, both of which
apply to  equivalence classes of valuations (i.e., the property
holds for $\absspc{}$ if and only if it holds for a valuation
equivalent to $\absspc{}$).
\begin{definition}[Discrete]
A valuation $\absspc{}$ is \defn{discrete}\i{valuation!discrete}
if there is a $\delta>0$
such that for any $a\in K$
$$
   1-\delta < \abs{a} < 1+\delta \implies |a|=1.
$$
Thus the absolute values are bounded away from $1$.
%then $|a|=1$.
\end{definition}
To say that $\absspc{}$ is discrete is the same as saying
that the set
$$G=\bigl\{
  \log\abs{a} : a \in K, a\neq 0
\bigr\} \subset \R
$$
forms a discrete subgroup of the reals under addition (because
the elements of the group $G$ are bounded away from $0$).
\begin{proposition}\label{prop:discrete}\iprop{discrete subgroup of $\R$}
A nonzero discrete subgroup $G$ of $\R$ is free on one generator.
\end{proposition}
\begin{proof}
Since $G$ is discrete there is a positive  $m\in G$
such that for any positive $x\in G$ we have $m\leq x$.
Suppose $x\in G$ is an arbitrary positive element.
By subtracting off integer multiples of~$m$, we
find that there is a unique $n$ such that
$$
   0\leq x-nm <m.
$$
Since $x-nm\in G$ and $0 \leq x-nm<m$, it follows
that $x-nm=0$, so $x$ is a multiple of $m$.
\end{proof}
By Proposition~\ref{prop:discrete}, the set
of $\log\abs{a}$ for nonzero $a\in K$
is free on one generator, so there
is a $c<1$ such that $\abs{a}$, for $a\neq 0$,
runs precisely through the set $$c^\Z = \{c^m : m\in \Z\}$$
(Note: we can replace $c$ by $c^{-1}$ to see that we
can assume that $c<1$).

\begin{definition}[Order]\label{def:ord}
If $\abs{a} = c^m$, we call $m=\ord(a)$ the \defn{order}
of $a$.
\end{definition}
Axiom (2) of valuations
translates into
$$
  \ord(ab) = \ord(a) + \ord(b).
$$

\begin{definition}[Non-archimedean]
A valuation $\absspc{}$ is \defn{non-archimedean}
if we can take $C=1$ in Axiom (3), i.e., if
\begin{equation}\label{eqn:na}
   |a + b| \leq \max\bigl\{|a|,|b|\bigr\}.
\end{equation}
If $\absspc{}$ is not non-archimedean then
it is \defn{archimedean}.
\end{definition}
Note that if we can take $C=1$ for $\absspc{}$
then we can take $C=1$ for any valuation equivalent to
$\absspc{}$.
To see that (\ref{eqn:na}) is equivalent to Axiom (3) with
$C=1$, suppose $|b|\leq |a|$.  Then $|b/a|\leq 1$, so
Axiom (3) asserts that $|1+b/a|\leq 1$, which implies
that $|a+b| \leq |a| = \max\{|a|,|b|\}$, and conversely.

We note at once the following consequence:
\begin{lemma}\ilem{$|a+b|=|a|$}
Suppose $\absspc{}$ is a non-archimedean valuation.
If $a,b\in K$ with $|b|<|a|$, then
$
 |a+b|=|a|.
$
\end{lemma}
\begin{proof}
Note that $|a+b|\leq \max\{|a|,|b|\} = |a|$, which
is true even if $|b|=|a|$.  Also,
$$
  |a| = |(a+b) - b| \leq \max\{|a+b|, |b|\} = |a+b|,
$$
where for the last equality we have used that $|b|<|a|$
(if $\max\{|a+b|,|b|\} = |b|$, then $|a|\leq |b|$,
a contradiction).

\end{proof}

\begin{definition}[Ring of Integers]
Suppose $\absspc$ is a non-archimedean absolute
value on a  field $K$.  Then
$$
  \O = \{a\in K : |a|\leq 1\}
$$ is a ring called the \defn{ring of integers} of $K$
with respect to $\absspc{}$.
\end{definition}

\begin{lemma}\ilem{equivalent non-archimedean valuations and $\O$'s}
Two non-archimedean valuations $\absspc{}_1$ and
$\absspc{}_2$ are equivalent if and only if they
give the same $\O$.
\end{lemma}
We will prove this modulo the claim (to
be proved later in Section~\ref{sec:topology}) that
valuations are equivalent if (and only if) they induce the
same topology.
\begin{proof}
Suppose suppose $\absspc{}_1$ is equivalent to
$\absspc{}_2$, so $\absspc{}_1 = \absspc{}_2^c$,
for some $c>0$.  Then $\abs{c}_1 \leq 1$ if and only if
$\abs{c}_2^c \leq 1$, i.e., if $\abs{c}_2 \leq 1^{1/c}=1$.
Thus $\O_1 = \O_2$.

Conversely, suppose $\O_1 = \O_2$.
Then $|a|_1<|b|_1$ if and only if $a/b\in \O_1$
and $b/a\not\in \O_1$, so
\begin{equation}\label{eqn:ineqiff}
  |a|_1<|b|_1 \iff |a|_2 < |b|_2.
\end{equation}
The topology induced by $|\mbox{ }|_1$ has as basis
of open neighborhoods the set of open balls
$$
 B_1(z,r) = \{x \in K : |x-z|_1<r \},
$$
for $r>0$, and likewise for $|\mbox{ }|_2$.  Since
the absolute values $|b|_1$ get arbitrarily close
to $0$, the set $\mathcal{U}$ of open balls $B_1(z,|b|_1)$ also
forms a  basis of the topology induced
by $|\mbox{ }|_1$ (and similarly for $|\mbox{ }|_2$).
By (\ref{eqn:ineqiff}) we have
$$
 B_1(z,|b|_1) = B_2(z,|b|_2),
$$
so the two topologies both have $\mathcal{U}$ as
a basis, hence are equal.  That equal topologies
imply equivalence of the corresponding valuations
will be proved in Section~\ref{sec:topology}.
\end{proof}

The set of $a\in \O$ with $|a|<1$ forms an ideal $\p$ in $\O$.  The
ideal $\p$ is maximal, since if $a\in\O$ and $a\not\in\p$ then
$|a|=1$, so $|1/a| = 1/|a| = 1$, hence $1/a\in \O$, so $a$ is a unit.

\begin{lemma}\label{lem:discrete_principal}\ilem{characterization of discrete}
A non-archimedean valuation $\absspc{}$ is
discrete if and only if $\p$ is a principal ideal.
\end{lemma}
\begin{proof}
First suppose that $\absspc{}$ is discrete.
Choose $\pi \in \p$ with $|\pi|$ maximal, which
we can do since
$$
  S=\{\log|a| : a \in \p\} \subset (-\infty,1],
$$
so the discrete set~$S$ is bounded above.
Suppose $a\in \p$.   Then
$$
  \abs{\frac{a}{\pi}} = \frac{\abs{a}}{\abs{\pi }} \leq 1,
$$
so $a/\pi\in \O$.
Thus $$a = \pi \cdot \frac{a}{\pi} \in \pi \O.$$

Conversely, suppose $\p=(\pi)$ is principal.  For any $a\in \p$
we have $a=\pi b$ with $b\in\O$.  Thus
$$
  |a| = |\pi|\cdot |b| \leq |\pi| < 1.
$$
Thus $\{|a| : |a|<1\}$ is bounded away from $1$,
which is exactly the definition of discrete.
\end{proof}

\begin{example}\label{ex:padic_valuation}
For any prime $p$, define the $p$-adic valuation
$\absspc{}_p:\Q\to\R$ as follows.  Write a nonzero $\alpha\in K$
as $p^n\cdot \frac{a}{b}$, where $\gcd(a,p)=\gcd(b,p)=1$.  Then
$$\abs{p^n\cdot \frac{a}{b}}_p := p^{-n} = \left(\frac{1}{p}\right)^{n}.$$
This valuation is both discrete and non-archimedean.
The ring $\O$ is the local ring
$$
  \Z_{(p)} = \left\{\frac{a}{b}\in\Q : p\nmid b\right\},
$$
which has maximal ideal generated by $p$.  Note that
$\ord(p^n\cdot \frac{a}{b}) = n.$
\end{example}

\begin{exercise} \label{ex:valuations1}
Give an example of a non-archimedean valuation on a field that
is not discrete.
\end{exercise}

We will use the following lemma later (e.g., in
the proof of Corollary~\ref{cor:valna} and Theorem~\ref{thm:ostrowski}).
\begin{lemma}\label{lem:nonarch}\ilem{non-archimedean valuation characterization}
  A valuation $\absspc{}$ is non-archimedean if and only if $|n|\leq
  1$ for all $n$ in the ring generated by $1$ in $K$.
\end{lemma}
Note that we cannot identify the ring generated by $1$ with~$\Z$
in general, because~$K$ might have characteristic $p>0$.
\begin{proof}
If $\absspc{}$ is non-archimedean, then $|1|\leq 1$,
so by Axiom (3) with $a=1$, we have  $|1+1|\leq 1$.  By
induction it follows that $|n|\leq 1$.

Conversely, suppose $|n|\leq 1$ for all integer multiples~$n$ of~$1$.
This condition is also true if we replace $\absspc{}$ by
any equivalent valuation, so replace $\absspc{}$ by
one with $C\leq 2$, so that the triangle inequality holds.
Suppose $a\in K$ with $|a|\leq 1$.  Then
by the triangle inequality,
\begin{align*}
  \abs{1+a}^n &= \abs{(1+a)^n} \\
     \leq& \sum_{j=0}^n \abs{\binom{n}{j}} \abs{a}^j\\
     \leq& 1 + 1 + \cdots + 1 = n+1.
\end{align*}
Now take $n$th roots of both sides to get
$$\abs{1+a} \leq \sqrt[n]{n+1},$$
and take the limit as $n\to \infty$ to see
that $\abs{1+a} \leq 1$.  This proves that one
can take $C=1$ in Axiom (3), hence that $\absspc{}$
is non-archimedean.
\end{proof}

\section{Examples of Valuations}
The archetypal example of an archimedean valuation is the absolute
value on the complex numbers.  It is essentially the only one:
\begin{theorem}[Gelfand-Tornheim]\ithm{Gelfand-Tornheim} Any field~$K$ with
an archimedean valuation is isomorphic to a subfield of~$\C$,
the valuation being equivalent to that induced by the usual
absolute value on~$\C$.
\end{theorem}
We do not prove this here as we do not need it.  For a proof,
see \cite[pg. 45, 67]{artin:ant}.

There are many non-archimedean valuations.  On the rationals $\Q$
there is one for every prime $p>0$, the $p$-adic valuation, as
in Example~\ref{ex:padic_valuation}.

\begin{theorem}[Ostrowski]\label{thm:ostrowski}\ithm{Ostrowski}
\ithm{valuations on $\Q$}
The nontrivial valuations
on $\Q$ are those equivalent to $\absspc_p$, for some
prime $p$, and the usual absolute value $\absspc_\infty$.
\end{theorem}
\begin{remark}
Before giving the proof, we pause with a brief remark about
Ostrowski.  According to
\begin{verbatim}
   http://www-gap.dcs.st-and.ac.uk/~history/Mathematicians/Ostrowski.html
\end{verbatim}
\noindent{}Ostrowski was a Ukrainian mathematician who lived
1893--1986.  Gautschi writes about Ostrowski as follows: ``... you are
able, on the one hand, to emphasise the abstract and axiomatic side of
mathematics, as for example in your theory of general norms, or, on
the other hand, to concentrate on the concrete and constructive
aspects of mathematics, as in your study of numerical methods, and to
do both with equal ease. {\em You delight in finding short and
succinct proofs, of which you have given many examples} ...'' [italics mine]
\end{remark}
We will now give an example of one of these short and succinct proofs.
\begin{proof}
Suppose $\absspc$ is a nontrivial valuation on $\Q$.
\par{\em Nonarchimedean case:}
Suppose $\abs{c}\leq 1$ for all $c\in\Z$, so by
Lemma~\ref{lem:nonarch}, $\absspc$ is nonarchimedean.
Since $\absspc$ is nontrivial, the set
$$
  \p=\{a\in\Z : \abs{a}<1\}
$$
is nonzero.  Also $\p$ is an ideal and if $\abs{ab}<1$,
then $\abs{a}\abs{b}=\abs{ab}<1$, so $\abs{a}<1$ or $\abs{b}<1$,
so $\p$ is a prime ideal of~$\Z$.  Thus $\p=p\Z$, for some prime
number~$p$.  Since every element of $\Z$ has valuation at most
$1$, if  $u\in\Z$ with $\gcd(u,p)=1$, then $u\not\in\p$,
so $\abs{u}=1$.  Let $\alpha=\log_{\abs{p}}\frac{1}{p}$, so
$\abs{p}^\alpha = \frac{1}{p}$.    Then for any $r$ and any $u\in\Z$
with $\gcd(u,p)=1$, we have
$$
\abs{up^r}^{\alpha} = \abs{u}^{\alpha}\abs{p}^{\alpha r}
   = \abs{p}^{\alpha r} = p^{-r} = \abs{up^r}_p.
$$
Thus $\absspc^{\alpha} =  \absspc_p$ on $\Z$, hence on $\Q$
by multiplicativity, so $\absspc$ is equivalent to $\absspc_p$,
as claimed.

{\em Archimedean case:} By replacing $\absspc$ by a power of
$\absspc$, we may assume without loss that $\absspc$ satisfies the
triangle inequality.  We first make some general remarks about any
valuation that satisfies the triangle inequality.
Suppose $a\in\Z$ is greater than $1$.  Consider, for any positive $b\in\Z$
the base-$a$ expansion of $b$:
$$
  b = b_m a^m + b_{m-1} a^{m-1} + \cdots + b_0,
$$
where
$$
  0 \leq b_j < a \qquad (0\leq j \leq m),
$$
and $b_m\neq 0$.
Since $a^m\leq b$, taking logs we see that
$m\log(a)\leq \log(b)$, so
$$m \leq \frac{\log(b)}{\log(a)}.$$
Let $\ds M=\max_{1\leq d<a}\abs{d}$.  Then by the triangle
inequality for $\absspc$, we have
\begin{align*}
\abs{b}&\leq \abs{b_m}\abs{a}^m + \cdots + \abs{b_1}\abs{a} + \abs{b_0}\\
    & \leq M\cdot (\abs{a}^m + \cdots + \abs{a} + 1)\\
   & \leq M\cdot (m+1)\cdot \max(1,\abs{a}^m)\\
   & \leq M\cdot\left(\frac{\log(b)}{\log(a)} + 1\right)
        \cdot \max\left(1,\abs{a}^{\log(b)/\log(a)}\right),
\end{align*}
where in the last step we use that $m\leq \frac{\log(b)}{\log(a)}$.
Setting $b=c^n$, for some positive $c\in\Z$, in the above inequality and
taking $n$th roots, we have
\begin{align*}
\abs{c} &\leq \left( M\cdot
\left(\frac{\log(c^n)}{\log(a)}+1\right)\cdot
  \max(1,\abs{a}^{log(c^n)/\log(a)})\right)^{1/n}\\
  & = M^{1/n}\cdot\left(
       \frac{\log(c^n)}{\log(a)}+1\right)^{1/n}\cdot
      \max\left(1,\abs{a}^{\log(c^n)/\log(a)}\right)^{1/n}.
\end{align*}
The first factor $M^{1/n}$ converges to~$1$ as $n\to\infty$,
since $M\geq 1$ (because $\abs{1}=1$).  The second factor
is
$$
  \left(\frac{\log(c^n)}{\log(a)}+1\right)^{1/n}
   =
\left(n \cdot \frac{\log(c)}{\log(a)}+1\right)^{1/n}
$$
which also converges to $1$, for the
same reason that $n^{1/n}\to 1$
(because $\log(n^{1/n})=\frac{1}{n}\log(n)\to 0$ as
$n\to\infty$).
The third factor is
$$
\max\left(1,\abs{a}^{\log(c^n)/\log(a)}\right)^{1/n}
  = \begin{cases}
   1 & \text{if }\abs{a}<1,\\
\abs{a}^{\log(c)/\log(a)} & \text{if }\abs{a}\geq 1.
\end{cases}
$$
Putting this all together, we see that
$$
\abs{c} \leq \max\left(1,\abs{a}^{\frac{\log(c)}{\log(a)}}\right).
$$

By Lemma~\ref{lem:nonarch}, our assumption that $\absspc$ is archimedean implies
that there is $c\in\Z$ with $c>1$ and $\abs{c}>1$.
Then for all $a\in\Z$ with $a>1$ we have
\begin{equation}\label{eqn:pow}
  1 < \abs{c} \leq
  \max\left(1,\abs{a}^{\frac{\log(c)}{\log(a)}}\right),
\end{equation}
so $1<\abs{a}^{\log(c)/\log(a)}$, so
$1<\abs{a}$ as well (i.e., any $a\in\Z$ with
$a>1$ automatically satisfies $\abs{a}>1$).  Also, taking the
$1/\log(c)$ power on both sides of (\ref{eqn:pow})
we see that
\begin{equation}\label{eqn:ineqac}
  \abs{c}^{\frac{1}{\log(c)}}
    \leq   \abs{a}^{\frac{1}{\log(a)}}.
\end{equation}
Because, as mentioned above, $\abs{a}>1$, we can interchange the role
of $a$ and $c$ to obtain the reverse inequality of (\ref{eqn:ineqac}).
We thus have
$$
  \abs{c}
    =   \abs{a}^{\frac{\log(c)}{\log(a)}}.
$$ Letting $\alpha=\log(2)\cdot \log_{\abs{2}}(e)$ and setting $a=2$,
we have
$$
  \abs{c}^{\alpha} = \abs{2}^{\frac{\alpha}{\log(2)}\cdot \log(c)}
      = \left(\abs{2}^{\log_{\abs{2}}(e)}\right)^{\log(c)} =
   e^{\log(c)} = c = \abs{c}_\infty.
$$
Thus for all integers $c\in\Z$ with $c>1$ we have
$\abs{c}^{\alpha} = \abs{c}_{\infty}$, which implies
that $\absspc$ is equivalent to $\absspc_\infty$.
\end{proof}

Let $k$ be any field and let $K=k(t)$, where $t$
is transcendental.  Fix a real number $c>1$.
If $p=p(t)$ is an irreducible
polynomial in the ring $k[t]$, we define a valuation
by
\begin{equation}\label{eqn:ffabsp}
  \abs{p^a \cdot \frac{u}{v}}_p = c^{-\deg(p)\cdot a},
\end{equation}
where $a\in\Z$ and $u,v\in k[t]$ with
$p\nmid u$ and $p\nmid v$. We will use $\ord_p$ for the
order associated to $\abs{\cdot}_p$, see definition~\ref{def:ord}.
\begin{remark}
This definition differs from the one page 46 of [Cassels-Frohlich,
Ch. 2] in two ways.   First, we assume that $c>1$ instead
of $c<1$, since otherwise $\absspc_p$ does not satisfy
Axiom 3 of a valuation.  Also, we write $c^{-\deg(p)\cdot a}$
instead of $c^{-a}$, so that the product formula will
hold.  (For more about the product formula, see
Section~\ref{sec:global_fields}.)
\end{remark}
In addition there is a non-archimedean valuation
$\absspc_\infty$ defined by
\begin{equation}\label{eqn:ffabsoo}
  \abs{\frac{u}{v}}_\infty = c^{\deg(u)-\deg(v)}.
\end{equation}


This definition differs from the one in \cite[pg.~46]{cassels:global}
in two ways.  First, we assume that $c>1$ instead of $c<1$, since
otherwise $\absspc_p$ does not satisfy Axiom 3 of a valuation.  Here's
why: Recall that Axiom 3 for a non-archimedean valuation on $K$
asserts that whenever $a\in K$ and $\abs{a}_p\leq 1$, then
$\abs{a+1}_p\leq 1$.  Set $a=p-1$, where $p=p(t)\in K[t]$ is an
irreducible polynomial.  Then $\abs{a}_p=c^0 = 1$, since $\ord_p(p-1) =
0$.  However, $\abs{a+1}_p = \abs{p-1+1}_p = \abs{p}_p=c^{-\deg(p)}>1$, since
$\ord_p(p) = 1$.  If we take $c>1$ instead of $c<1$, as I propose,
then $\abs{p}_p=c^{-\deg(p)}<1$, as required.


Note the (albeit imperfect) analogy between $K=k(t)$ and $\Q$.
If $s=t^{-1}$, so $k(t)=k(s)$, the valuation $\absspc_{\infty}$
is of the type (\ref{eqn:ffabsp}) belonging to the irreducible
polynomial $p(s)=s$.

The reader is urged to prove the following lemma as a homework
problem.
\begin{lemma}\ilem{valuations on $k(t)$}
The only nontrivial valuations on $k(t)$ which are trivial
on $k$ are equivalent to the valuation (\ref{eqn:ffabsp})
or (\ref{eqn:ffabsoo}).
\end{lemma}
For example, if $k$ is a finite field, there are no
nontrivial valuations on $k$, so the only
nontrivial valuations on $k(t)$ are equivalent to
(\ref{eqn:ffabsp}) or (\ref{eqn:ffabsoo}).

\begin{exercise} \label{ex:valuations2}
Let $k$ be any field. Prove that the only nontrivial valuations
on $k(t)$ which are trivial on $k$ are equivalent to the valuation
(\ref{eqn:ffabsp}) or (\ref{eqn:ffabsoo}) of page~\pageref{eqn:ffabsp}.
\end{exercise}
\end{ch}
%%%%%%%%%%%%%%%%%%%%%%%%%%%%%%%%%%%%%%%%%%%%%%%%%%%%%%%%%%%%%%%%%%%%%%%%%%%%%%%
%%%%%%%%%%%%%%%%%%%%%%%%%%%% END VALUATIONS %%%%%%%%%%%%%%%%%%%%%%%%%%%%%%%%%%%
%%%%%%%%%%%%%%%%%%%%%%%%%%%%%%%%%%%%%%%%%%%%%%%%%%%%%%%%%%%%%%%%%%%%%%%%%%%%%%%



























%%%%%%%%%%%%%%%%%%%%%%%%%%%%%%%%%%%%%%%%%%%%%%%%%%%%%%%%%%%%%%%%%%%%%%%%%%%%%%%
%%%%%%%%%%%%%%%%%%%%%%%%%%%% START TOPOLOGY %%%%%%%%%%%%%%%%%%%%%%%%%%%%%%%%%%%
%%%%%%%%%%%%%%%%%%%%%%%%%%%%%%%%%%%%%%%%%%%%%%%%%%%%%%%%%%%%%%%%%%%%%%%%%%%%%%%
\begin{ch}
\chapter{Topology and  Completeness}

\section{Topology}\label{sec:topology}
A valuation $\absspc$ on a field $K$ induces a topology in which a
basis for the neighborhoods of $a$ are the \defn{open balls}
$$
  B(a,d) = \{x\in K : \abs{x-a} < d\}
$$
for $d>0$.
\begin{lemma}\ilem{equivalent valuations, same topology}
Equivalent valuations induce the same topology.
\end{lemma}
\begin{proof}
If $\absspc_1=\absspc_2^r$, then
$\abs{x-a}_1 < d$ if and only if
$\abs{x-a}_2^r<d$ if and only if
$\abs{x-a}_2<d^{1/r}$
so $B_1(a,d) = B_2(a,d^{1/r})$.
Thus the basis of open neighborhoods of $a$
for $\absspc_1$ and $\absspc_2$ are identical.
\end{proof}

A valuation satisfying the triangle inequality gives a metric for the
topology on defining the distance from $a$ to $b$ to be $\abs{a-b}$.
Assume for the rest of this section that we only consider valuations
that satisfy the triangle inequality.
\begin{lemma}\ilem{topological field}
A field with the topology induced by a valuation is
a \defn{topological field}, i.e., the operations sum, product,
and reciprocal are continuous.
\end{lemma}
\begin{proof}
For example (product) the triangle inequality implies that
$$
  \abs{(a+\eps)(b+\delta) - ab}
   \leq \abs{\eps}\abs{\delta} + \abs{a}\abs{\delta}
     + \abs{b}\abs{\eps}
$$
     is small when $\abs{\eps}$ and $\abs{\delta}$ are
small (for fixed $a, b$).
\end{proof}
\begin{exercise}\label{ex:topology1}
Prove the previous lemma, i.e., prove that the operations sum, product,
and reciprocal are continuous.
\end{exercise}


\begin{lemma}\label{lem:absvalconv}
Suppose two valuations $\absspc_1$ and $\absspc_2$ on the same
field $K$ induce the same topology. Then
for any sequence $\{x_n\}$ in~$K$ we
have
$$
  \abs{x_n}_1 \to 0 \iff \abs{x_n}_2 \to 0.
$$
\end{lemma}
\begin{proof}
It suffices to prove that if $\abs{x_n}_1\to 0$
then $\abs{x_n}_2\to 0$, since the proof of the
other implication is the same.
Let $\eps>0$.  The topologies induced by the two absolute
values are the same, so $B_2(0,\eps)$ can be covered by
open balls $B_1(a_i,r_i)$.  One of these open balls
$B_1(a,r)$ contains~$0$. There is $\eps'>0$ such that
$$
  B_1(0,\eps') \subset B_1(a,r)\subset B_2(0,\eps).
$$
Since $\abs{x_n}_1\to 0$, there exists $N$ such
that for $n\geq N$ we have $\abs{x_n}_1 <\eps'$.
For such~$n$, we have $x_n\in B_1(0,\eps')$, so $x_n\in B_2(0,\eps)$,
so $\abs{x_n}_2<\eps$.  Thus $\abs{x_n}_2\to 0$.
\end{proof}

\begin{proposition}\label{prop:same_topo}\iprop{same topology implies equivalent valuations}
If two valuations $\absspc_1$ and $\absspc_2$ on the same
field induce the same topology, then they are equivalent in
the sense that there is a positive real $\alpha$ such that
$\absspc_1 = \absspc_2^{\alpha}$.
\end{proposition}
\begin{proof}
If $x\in K$ and $i=1,2$, then $\abs{x^n}_i \to 0$
if and only if $\abs{x}_i^n\to 0$, which is the
case if and only if $\abs{x}_i<1$.   Thus
Lemma~\ref{lem:absvalconv} implies that
$\abs{x}_1<1$ if and only if $\abs{x}_2<1$.
On taking reciprocals we see that $\abs{x}_1>1$
if and only if $\abs{x}_2>1$, so finally
$\abs{x}_1 = 1$ if and only if $\abs{x}_2=1$.

Let now $w,z\in K$ be nonzero elements with $\abs{w}_i \neq 1$ and
$\abs{z}_i\neq 1$.  On
applying the foregoing to
$$
  x = w^m z^n \qquad (m,n\in\Z)
$$
we see that
$$
  m\log\abs{w}_1 + n\log\abs{z}_1 \geq 0
$$
if and only if
$$
  m\log\abs{w}_2 + n\log\abs{z}_2 \geq 0.
$$
Dividing through by $\log\abs{z}_i$, and rearranging,
we see that for every rational number $\alpha=-n/m$,
$$
  \frac{\log\abs{w}_1}{\log \abs{z}_1} \geq \alpha
\iff
  \frac{\log\abs{w}_2}{\log \abs{z}_2} \geq \alpha.
$$
Thus
$$
  \frac{\log\abs{w}_1}{\log \abs{z}_1} =
                \frac{\log\abs{w}_2}{\log \abs{z}_2},
$$
so
$$
  \frac{\log\abs{w}_1}{\log \abs{w}_2} =
                \frac{\log\abs{z}_1}{\log \abs{z}_2}.
$$
Since this equality does not depend on the choice of~$z$,
we see that there is a constant $c$ ($=\log\abs{z}_1/\log \abs{z}_2$)
such that $\log\abs{w}_1/\log \abs{w}_2 = c$ for all $w$.
Thus $\log\abs{w}_1 = c\cdot \log\abs{w}_2$, so
$\abs{w}_1 = \abs{w}_2^c$, which implies that $\absspc_1$
is equivalent to $\absspc_2$.
\end{proof}

\section{Completeness}\label{sec:completeness}
We recall the definition of metric on a set $X$.
\begin{definition}[Metric]\label{defn:metric}
A \defn{metric} on a set~$X$ is a map
$$
  d : X \cross X \ra \R
$$
such that for all $x,y,z\in X$,
\begin{enumerate}
\item $d(x,y)\geq 0$ and $d(x,y)=0$ if and only if $x=y$,
\item $d(x,y)=d(y,x)$, and
\item $d(x,z)\leq d(x,y)+d(y,z)$.
\end{enumerate}
\end{definition}
A \defn{Cauchy sequence} is a sequence
$(x_n)$ in $X$ such that for all $\eps>0$ there exists~$M$ such that
for all $n,m>M$ we have $d(x_n,x_m)<\eps$.  The \defn{completion}
of~$X$ is the set of Cauchy sequences $(x_n)$ in~$X$ modulo the
equivalence relation in which two Cauchy sequences $(x_n)$ and $(y_n)$
are equivalent if $\lim_{n\ra\infty} d(x_n,y_n)=0$.  A metric space is
\defn{complete}\index{complete|nn} if every Cauchy sequence converges,
and one can show that the completion of~$X$ with respect to a metric
is complete.

For example, $d(x,y)=|x-y|$ (usual archimedean absolute value) defines
a metric on~$\Q$.  The completion of $\Q$ with respect to this metric
is the field $\R$ of real numbers.  More generally, whenever $\absspc$
is a valuation on a field $K$ that satisfies the triangle inequality,
then $d(x,y)=\abs{x-y}$ defines a metric on $K$.
Consider for the rest of this section only valuations that
satisfy the triangle inequality.

\begin{definition}[Complete]
A field $K$ is \defn{complete} with respect to a valuation $\absspc$
if given any Cauchy sequence $a_n$, ($n=1,2,\ldots$), i.e.,
one for which
$$
\abs{a_m - a_n} \to 0 \qquad(m,n\to \infty,\infty),
$$
there is an $a^*\in K$ such that
$$
  a_n \to a^* \qquad \text{ w.r.t. }\absspc
$$
(i.e., $\abs{a_n-a^*}\to 0$).
\end{definition}

\begin{theorem}\ithm{complete embedding}
Every field $K$ with valuation $v=\absspc$ can be
embedded in a complete field $K_v$ with a valuation $\absspc{}$
extending the original one in such a way that $K_v$ is the closure of
$K$ with respect to $\absspc{}$.  Further $K_v$ is unique up to
a unique isomorphism fixing $K$.
\end{theorem}
\begin{proof}
Define $K_v$ to be the completion of $K$ with respect to the metric
defined by $\absspc$.  Thus $K_v$ is the set of equivalence classes of
Cauchy sequences, and there is a natural injective map from $K$ to
$K_v$ sending an element $a\in K$ to the constant Cauchy sequence
$(a)$.  Because the field operations on $K$ are continuous, they
induce well-defined field operations on equivalence classes of Cauchy
sequences componentwise.   Also, define a valuation on $K_v$ by
$$\abs{(a_n)_{n=1}^{\infty}} = \lim_{n\to\infty} \abs{a_n},$$ and note
that this is well defined and extends the valuation on $K$.

To see that $K_v$ is unique up to a unique isomorphism fixing~$K$, we
observe that there are no nontrivial continuous automorphisms $K_v\to
K_v$ that fix~$K$.  This is because, by denseness, a continuous
automorphism $\sigma: K_v\to K_v$ is determined by what it does
to~$K$, and by assumption~$\sigma$ is the identity map on~$K$.  More
precisely, suppose $a\in K_v$ and~$n$ is a positive integer.  Then by
continuity there is $\delta>0$ (with $\delta<1/n$) such that if
$a_n\in K_v$ and $\abs{a-a_n}<\delta$ then
$\abs{\sigma(a)-\sigma(a_n)}<1/n$.  Since $K$ is dense in $K_v$, we
can choose the $a_n$ above to be an element of~$K$.  Then by
hypothesis $\sigma(a_n)=a_n$, so $\abs{\sigma(a) - a_n} < 1/n$.  Thus
$\sigma(a) = \lim_{n\to\infty} a_n = a$.
\end{proof}

\begin{corollary}\label{cor:valna}\icor{valuation stays non-archimedean}
\icor{value set stays same}
The valuation $\absspc$ is non-archimedean
on $K_v$ if and only if it is so on $K$.  If $\absspc$ is
non-archimedean, then the set of values taken by $\absspc{}$ on $K$
and $K_v$ are the same.
\end{corollary}
\begin{proof}
  The first part follows from Lemma~\ref{lem:nonarch} which asserts
  that a valuation is non-archimedean if and only if $\abs{n}<1$ for
  all integers $n$.  Since the valuation on $K_v$ extends the
  valuation on~$K$, and all $n$ are in $K$, the first statement
  follows.

For the second, suppose that $\absspc{}$ is non-archimedean (but
not necessarily discrete).
Suppose $b\in K_v$ with $b\neq 0$.
First I claim that there is $c\in K$ such that $\abs{b-c} < \abs{b}$.
To see this, let $c'=b-\frac{b}{a}$, where~$a$ is some
element of $K_v$ with $\abs{a}>1$, note that
$\abs{b-c'}=\abs{\frac{b}{a}}<\abs{b}$, and choose $c\in K$ such
that $\abs{c-c'} < \abs{b-c'}$, so
$$\abs{b-c} = \abs{b-c' - (c-c')}
 \leq \max\left(\abs{b-c'},\abs{c-c'}\right) = \abs{b-c'}<\abs{b}.
$$
Since $\absspc{}$ is non-archimedean,
we have
$$
  \abs{b} = \abs{(b-c)+c} \leq \max\left(\abs{b-c},\abs{c}\right) = \abs{c},
$$
where in the last equality we use that $\abs{b-c}<\abs{b}$.
Also,
$$
  \abs{c} = \abs{b + (c-b)} \leq \max\left(\abs{b},\abs{c-b}\right) = \abs{b},
$$
so $\abs{b} = \abs{c}$, which is in the set of values of $\absspc{}$
on $K$.
\end{proof}

\subsection{$p$-adic Numbers}
This section is about the $p$-adic numbers $\Q_p$, which are the
completion of $\Q$ with respect to the $p$-adic valuation.
Alternatively, to give a $p$-adic {\em integer} in $\Z_p$ is the same
as giving for every prime power $p^r$ an element $a_r\in \Z/p^r\Z$
such that if $s\leq r$ then $a_s$ is the reduction of $a_r$ modulo
$p^s$.  The field $\Q_p$ is then the field of fractions of $\Z_p$.

\begin{exercise} \label{ex:topology2}
Prove that the field $\Q_p$ of $p$-adic numbers is uncountable.
\end{exercise}


We begin with the definition of the $N$-adic numbers for any positive
integer~$N$.  Section~\ref{sec:tenadic} is about the $N$-adics in the
special case $N=10$; these are fun because they can be represented as
decimal expansions that go off infinitely far to the left.
Section~\ref{sec:qnweird} is about how the topology of $\Q_N$ is
nothing like the topology of $\R$.  Finally, in
Section~\ref{sec:hasse} we state the Hasse-Minkowski theorem, which
shows how to use $p$-adic numbers to decide whether or not a quadratic
equation in~$n$ variables has a rational zero.

\subsubsection{The $N$-adic Numbers}\index{N@$N$-adic!numbers}
\label{sec:nadic}

\begin{lemma}\label{lem:ord_lemma}
Let $N$ be a positive integer.  Then for any
nonzero rational number~$\alpha$ there exists a
unique $e\in\Z$ and  integers~$a$,~$b$, with $b$ positive, such that
$\alpha = N^e \cdot \frac{a}{b}$ with
$N\nmid a$, $\gcd(a,b)=1$, and $\gcd(N,b)=1$.
\end{lemma}
\begin{proof}
Write $\alpha = c/d$ with $c,d\in\Z$ and $d>0$.
First suppose~$d$ is exactly divisible by a power of~$N$,
so for some~$r$ we have $N^r\mid d$ but $\gcd(N,d/N^r)=1$.
Then
$$
  \frac{c}{d} = N^{-r} \frac{c}{d/N^r}.
$$
If $N^s$ is the largest power of $N$ that divides~$c$, then $e=s-r$,
$a=c/N^s$, $b=d/N^r$ satisfy the conclusion of the lemma.

By unique factorization of integers, there is a smallest multiple~$f$
of~$d$ such that $fd$ is exactly divisible by~$N$.  Now apply the
above argument with~$c$ and~$d$ replaced by $cf$ and $df$.
\end{proof}


\begin{definition}[$N$-adic valuation]
\label{def:nadicvaluation}
Let~$N$ be a positive integer.  For any positive $\alpha\in\Q$, the
\defn{$N$-adic valuation} of~$\alpha$ is~$e$, where~$e$ is as in
Lemma~\ref{lem:ord_lemma}.  The $N$-adic  valuation of~$0$ is~$\infty$.
\end{definition}
We denote the $N$-adic valuation of $\alpha$ by $\ord_N(\alpha)$.
(Note: Here we are using ``valuation'' in a different way than in the
rest of the text.  This valuation is not an absolute value, but the
logarithm of one.)

\begin{definition}[$N$-adic metric]
\label{def:nadicmetric}
For $x,y\in\Q$ the \defn{$N$-adic distance}
between~$x$ and~$y$ is
$$
  d_N(x,y) = N^{-\ord_N(x-y)}.
$$
We let $d_N(x,x) = 0$, since $\ord_N(x-x)=\ord_N(0)=\infty$.
\end{definition}
For example, $x,y\in\Z$ are close  in the $N$-adic metric if their
difference is divisible by a large power of~$N$.   E.g., if $N=10$ then
$93427$ and $13427$ are close because their difference is $80000$, which
is divisible by a large power of~$10$.

\begin{proposition}\label{prop:ismetric}\iprop{$N$-distance is metric}
The distance $d_N$ on~$\Q$ defined above is a metric.  Moreover,
for all $x,y,z\in\Q$ we have
$$
 d(x,z) \leq \max(d(x,y),d(y,z)).
$$
(This is the ``nonarchimedean'' triangle inequality.)
\end{proposition}
\begin{proof}
The first two properties of Definition~\ref{defn:metric} are
immediate.  For the third, we first prove that if $\alpha,\beta\in\Q$
then
$$
 \ord_N(\alpha+\beta)\geq \min(\ord_N(\alpha),\ord_N(\beta)).
$$
Assume, without loss, that $\ord_N(\alpha) \leq \ord_N(\beta)$ and
that both $\alpha$ and $\beta$ are nonzero.
Using Lemma~\ref{lem:ord_lemma} write $\alpha=N^e(a/b)$ and
$\beta=N^f(c/d)$ with~$a$ or~$c$ possibly negative.  Then
$$
 \alpha + \beta = N^e \left(\frac{a}{b} + N^{f-e}\frac{c}{d}\right)
                = N^e \left(\frac{ad+bcN^{f-e}}{bd}\right).
$$
Since $\gcd(N,bd)=1$ it follows that $\ord_N(\alpha+\beta)\geq e$.
Now suppose $x,y, z\in \Q$.  Then
$$
 x-z = (x-y) + (y-z),
$$
so
$$
 \ord_N(x-z) \geq \min (\ord_N(x-y), \ord_N(y-z)),
$$
hence $d_N(x,z) \leq \max(d_N(x,y), d_N(y,z))$.
\end{proof}

We can finally define the $N$-adic numbers.
\begin{definition}[The $N$-adic Numbers]
  The set of \defn{$N$-adic numbers}, denoted $\Q_N$, is the
  completion of~$\Q$ with respect to the metric $d_N$.
\end{definition}
The set $\Q_N$ is a ring\index{ring!of $N$-adic numbers|nn},
but it need not be a field as you will show in Exercises~\ref{ex:topology3} and
\ref{ex:padic3}. It is a field if and only if $N$ is prime.
Also, $\Q_N$ has a ``bizarre'' topology,
as we will see in Section~\ref{sec:qnweird}.

\begin{exercise}\label{ex:topology3}
Let $N>1$ be an integer.
\begin{enumerate}
\item Prove that $\Q_N$ is equipped with a natural ring structure.
\item If $N$ is prime, prove that $\Q_N$ is a field.
\end{enumerate}
\end{exercise}

\begin{exercise}\label{ex:topology4}
\begin{enumerate}
\item Let $p$ and $q$ be distinct primes.  Prove that
$\Q_{pq} \isom \Q_p \cross \Q_q$.
\item Is $\Q_{p^2}$ isomorphic to either of $\Q_p\cross \Q_p$ or $\Q_p$?
\end{enumerate}

\end{exercise}



\subsubsection{The $10$-adic Numbers}
\label{sec:tenadic}
It's a familiar fact that every real number can be written in the
form
$$
d_n\ldots d_1 d_0.d_{-1}d_{-2}\ldots
 = d_n 10^{n} + \cdots + d_1 10 + d_0
   + d_{-1} 10^{-1} + d_{-2} 10^{-2} + \cdots
$$
where each digit $d_i$ is between~$0$ and~$9$, and the sequence can
continue indefinitely to the right.

The $10$-adic numbers also have decimal expansions, but everything is backward!
To get a feeling for why this might be the case, we consider Euler's\index{Euler}
nonsensical series
$$
  \sum_{n=1}^{\infty} (-1)^{n+1}n! = 1! - 2! + 3! - 4! + 5! - 6! + \cdots.
$$
One can prove (see Exercise~\ref{ex:padic1}) that this series
converges in $\Q_{10}$ to some element $\alpha\in\Q_{10}$.

\begin{exercise}\label{ex:topology5}
Let $N>1$ be an integer.  Prove that the series
$$
  \sum_{n=1}^{\infty} (-1)^{n+1}n! = 1! - 2! + 3! - 4! + 5! - 6! + \cdots.
$$
converges in $\Q_N$.
\end{exercise}


What is $\alpha$?  How can we write it down?  First note that for all
$M\geq 5$, the terms of the sum are divisible by~$10$, so the difference
between~$\alpha$ and $1! - 2! + 3! - 4!$ is divisible by~$10$.  Thus
we can compute $\alpha$ modulo~$10$ by computing $1! - 2! + 3! - 4!$
modulo~$10$.  Likewise, we can compute~$\alpha$ modulo~$100$
by compute $1! - 2! + \cdots + 9! - 10!$, etc.
We obtain the following table:
\begin{center}
\begin{tabular}{|rl|}\hline
$\alpha$ & $\mod 10^r$\\\hline
$1$ & $\mod 10$\\
$81$ & $\mod 10^2$\\
$981$ & $\mod 10^3$\\
$2981$ & $\mod 10^4$\\
$22981$ & $\mod 10^5$\\
$422981$ & $\mod 10^6$\\\hline
\end{tabular}
\end{center}
Continuing we see that
$$
 1! - 2! + 3! - 4! + \cdots =
  \ldots 637838364422981\qquad\text{in $\Q_{10}$ !}
$$

Here's another example.  Reducing $1/7$ modulo larger and larger powers of~$10$ we
see that
$$
 \frac{1}{7} = \ldots857142857143\qquad\text{in $\Q_{10}$}.
$$

Here's another example, but with a decimal point.
$$
\frac{1}{70} = \frac{1}{10}\cdot \frac{1}{7} = \ldots85714285714.3
$$
We have
$$\frac{1}{3} + \frac{1}{7} =
   \ldots66667 + \ldots57143 = \frac{10}{21} = \ldots 23810,
$$
which illustrates that addition with carrying works as usual.

\subsubsection{Fermat's Last Theorem in $\Z_{10}$}
An amusing observation, which people often argued about
on USENET news back
in the 1990s, is that Fermat's last theorem\index{Fermat's last theorem}
is false in $\Z_{10}$. For example, $x^3 + y^3 = z^3$ has a nontrivial solution, namely
$x = 1$, $y=2$, and $z=\ldots60569$.   Here~$z$ is a cube
root of $9$ in $\Z_{10}$.  Note that it takes some work to prove that there
is a cube root of $9$ in $\Z_{10}$ (see Exercise~\ref{ex:padic2}).

\begin{exercise} \label{ex:topology6}
  Prove that $9$ has a cube root in $\Q_{10}$ using the following
  strategy (this is a special case of Hensel's Lemma, which you can
  read about in an appendix to Cassel's article).

\begin{enumerate}
\item Show that there is an element $\alpha\in\Z$ such that $\alpha^3\con 9\pmod{10^3}$.
\item Suppose $n\geq 3$.
Use induction to show that if $\alpha_1\in\Z$ and
$\alpha^3\con 9\pmod{10^n}$,  then there exists $\alpha_2\in\Z$ such
that $\alpha_2^3\con 9\pmod{10^{n+1}}$.
(Hint: Show that there is an integer~$b$ such that
$(\alpha_1 + b\cdot 10^{n})^3 \con 9\pmod{10^{n+1}}$.)
\item Conclude that $9$ has a cube root in $\Q_{10}$.
\end{enumerate}
\end{exercise}



\subsection{The Field of $p$-adic Numbers}\index{$p$-adic field}
The ring $\Q_{10}$ of $10$-adic numbers is isomorphic to
$\Q_{\hspace{.2ex}2}\cross\Q_{\hspace{.2ex}5}$ (see Exercise~\ref{ex:padic3}), so it is not a
field.  For example, the element $\ldots8212890625$ corresponding to
$(1,0)$ under this isomorphism has no inverse.  (To compute~$n$ digits
of $(1,0)$ use the Chinese remainder theorem to find a number that
is~$1$ modulo~$2^{n}$ and~$0$ modulo $5^n$.)

If~$p$ is prime then $\Q_p$ is a field (see Exercise~\ref{ex:padic4}).
Since $p\neq 10$ it is a little more
complicated to write $p$-adic numbers down.  People typically write
$p$-adic numbers in the form
$$
  \frac{a_{-\!d}}{p^{d}} + \cdots + \frac{a_{-1}}{p} + a_0 + a_1 p + a_2 p^2 + a_3 p^3 + \cdots
$$
where $0\leq a_i < p$ for each~$i$.

\begin{exercise} \label{ex:topology7}
Compute the first~$5$ digits of the $10$-adic expansions of the following
rational numbers:
$$
 \frac{13}{2}, \quad \frac{1}{389}, \quad \frac{17}{19},
 \quad \text{ the 4 square roots of $41$}.
$$
\end{exercise}




\subsection{The Topology of $\Q_N$ (is Weird)}\label{sec:qnweird}
\begin{definition}[Connected]\label{def:connected}\index{connected|nn}
Let $X$ be a topological space.  A subset~$S$ of~$X$ is
\defn{disconnected}
if there exist open subsets $U_1, U_2\subset X$ with $U_1\intersect U_2\intersect S=\emptyset$
and $S=(S\intersect U_1)\union (S\intersect U_2)$ with
$S\intersect U_1$ and $S\intersect U_2$ nonempty.
If~$S$ is not disconnected it is \defn{connected}.
\end{definition}

The topology on $\Q_N$ is induced by $d_N$, so every open set is a union
of open balls
$$
  B(x,r) = \{y \in \Q_N : d_N(x,y) < r\}.
$$
Recall Proposition~\ref{prop:ismetric}, which asserts that for
all $x,y,z$,
$$
 d(x,z) \leq \max(d(x,y), d(y,z)).
$$
This translates into the following shocking and bizarre lemma:
\begin{lemma}\label{lem:balls}
Suppose $x\in \Q_N$ and $r>0$.  If $y\in\Q_N$ and $d_N(x,y)\geq r$, then
$B(x,r)\intersect B(y,r) = \emptyset$.
\end{lemma}
\begin{proof}
Suppose $z\in B(x,r)$ and $z\in B(y,r)$.  Then
$$
 r\leq d_N(x,y) \leq \max(d_N(x,z), d_N(z,y)) < r,
$$
a contradiction.
\end{proof}
You should draw a picture to illustrates Lemma~\ref{lem:balls}.
\begin{lemma}\label{lem:opencomplement}\ilem{open ball is closed}
The open ball $B(x,r)$ is also closed.
\end{lemma}
\begin{proof}
Suppose $y\not\in B(x,r)$.  Then $r\leq d(x,y)$ so
$$
 B(y,d(x,y)) \intersect B(x,r)
\subset
 B(y,d(x,y)) \intersect B(x,d(x,y))
 = \emptyset.
$$
Thus the complement of $B(x,r)$ is a union of open balls.
\end{proof}
\begin{exercise}\label{ex:topology8}
Prove that the polynomial $f(x)=x^3 - 3x^2 + 2x + 5$
has all its roots in $\Q_5$, and find the $5$-adic valuations
of each of these roots.  (You might need to use
Hensel's lemma, which we don't discuss in detail
in this book. See \cite[App.~C]{cassels:global}.)
\end{exercise}

The lemmas imply that $\Q_N$ is \defn{totally disconnected},\index{totally disconnected}
\index{N@$N$-adic!totally disconnected}
in the following sense.
\begin{proposition}\label{prop:disconnected}\iprop{$\Q_N$ totally disconnected}
The only connected subsets of $\Q_N$ are the singleton sets
$\{x\}$ for $x\in\Q_N$ and the empty set.
\end{proposition}
\begin{proof}
Suppose $S\subset \Q_N$ is a nonempty connected set and $x, y$ are distinct
elements of~$S$.  Let $r=d_N(x,y)>0$.  Let $U_1=B(x,r)$ and $U_2$ be
the complement of $U_1$, which is open by Lemma~\ref{lem:opencomplement}.
Then $U_1$ and $U_2$ satisfies the conditions of Definition~\ref{def:connected},
so~$S$ is not connected, a contradiction.
\end{proof}


\subsection{The Local-to-Global Principle of Hasse and Minkowski}\label{sec:hasse}
\index{Hasse}\index{Minkowski}\index{local-to-global principal}
Section~\ref{sec:qnweird} might have convinced you that $\Q_N$ is a
bizarre pathology.  In fact, $\Q_N$ is omnipresent in number theory,
as the following two fundamental examples illustrate.

In the statement of the following theorem, a \defn{nontrivial solution}
to a homogeneous polynomial equation is a solution where not all
indeterminates are~$0$.
\begin{theorem}[Hasse-Minkowski]
\label{thm:hasse_minkowski}\ithm{Hasse-Minkowski}
The quadratic equation
\begin{equation}\label{eqn:quad}
a_1x_1^2 + a_2 x_2^2 + \cdots + a_n x_n^2 = 0,
\end{equation}
with $a_i\in\Q^{\star}$,
has a nontrivial solution with $x_1,\ldots, x_n$ in $\Q$ if and only
if  (\ref{eqn:quad}) has a solution in $\R$ and in $\Q_p$ for
all primes~$p$.
\end{theorem}
This theorem is very useful in practice because the
$p$-adic condition turns out to be easy to check.  For more details,
including a complete proof, see
\cite[IV.3.2]{serre:arithmetic}.

The analogue of Theorem~\ref{thm:hasse_minkowski}
for cubic equations is false.
For example, Selmer\index{Selmer}\index{Selmer curve} proved that the cubic
$$
 3x^3 + 4y^3 + 5z^3 = 0
$$
has a solution other than $(0,0,0)$ in $\R$ and in $\Q_p$ for all primes~$p$
but has no solution other than $(0,0,0)$ in~$\Q$ (for a proof
see \cite[\S 18]{cassels:lectures}).

\vspace{1ex}
\noindent{\bf Open Problem. }\index{open problem!solvability of plane cubics}
Give an algorithm that decides whether or not a cubic $$ax^3 + by^3 + cz^3=0$$
has a nontrivial solution in~$\Q$.
\vspace{1ex}

This open problem is closely related to the Birch and Swinnerton-Dyer
Conjecture\index{Birch and Swinnerton-Dyer conjecture} for elliptic
curves\index{elliptic curve}.  The truth of the conjecture would
follow if we knew that ``Shafarevich-Tate
Groups''\index{Shafarevich-Tate group} of certain elliptic curves are
finite.


\section{Weak Approximation}

The following theorem asserts that inequivalent valuations are in fact
almost totally independent.  For our purposes it will be superseded by
the strong approximation theorem (Theorem~\ref{thm:strong}).

\begin{theorem}[Weak Approximation]\label{thm:weakapprox}\ithm{weak approximation}
  Let $\absspc_n$, for $1\leq n \leq N$, be inequivalent nontrivial
  valuations of a field $K$.  For each $n$, let $K_n$ be the
  topological space consisting of the set of elements of $K$ with the
  topology induced by $\absspc_n$.  Let $\Delta$ be the image of $K$
  in the topological product $$A=\prod_{1\leq n\leq N} K_n$$ equipped
  with the product topology.  Then $\Delta$ is dense in $A$.
\end{theorem}
The conclusion of the theorem may be expressed in a less topological
manner as follows: given any $a_n\in K$, for $1\leq n \leq N$, and
real $\eps>0$, there is an $b\in K$ such that simultaneously
$$
  \abs{a_n - b}_n  < \eps \qquad (1\leq n\leq N).
$$

If $K=\Q$ and the $\absspc{}$ are $p$-adic valuations,
Theorem~\ref{thm:weakapprox} is related to the Chinese Remainder
Theorem (Theorem~\ref{thm:crt}), but the strong approximation theorem
(Theorem~\ref{thm:strong}) is the
real generalization.

\begin{proof}
  We note first that it will be enough to find, for each $n$, an
  element $c_n\in K$ such that
$$
  \abs{c_n}_n > 1 \quad \text{ and } \quad \abs{c_n}_m < 1
  \quad\text{ for }n\neq m,
  $$
  where $1\leq n,m\leq N$.  For then as $r\to+\infty$, we have
$$
\frac{c_n^r}{1+c_n^r} = \frac{1}{1+\left(\frac{1}{c_n}\right)^{r}}
 \to \begin{cases} 1 & \text{with respect to } \absspc_n \text{ and }\\
                   0 & \text{with respect to } \absspc_m, \text{ for }
                   m\neq n.
     \end{cases}
     $$
  It is then enough to take
$$
   b = \sum_{n=1}^N \frac{c_n^r}{1+c_n^r} \cdot a_n
$$

By symmetry it is enough to show the existence of $c=c_1$ with
$$
  \abs{c}_1 > 1 \qquad \text{and} \qquad \abs{c}_n<1 \quad
\text{for} \quad 2\leq n\leq N.
$$
We will do this by induction on $N$.

First suppose $N=2$.  Since $\absspc_1$ and $\absspc_2$ are
inequivalent (and all absolute values are assumed nontrivial)
there is an $a\in K$ such that
\begin{equation}\label{eqn:nonobvious}
  \abs{a}_1 < 1\qquad \text{and}\qquad \abs{a}_2 \geq 1
\end{equation}
and similarly a $b$ such that
$$
  \abs{b}_1 \geq 1\qquad \text{and}\qquad \abs{b}_2 < 1.
$$
Then $\ds c=\frac{b}{a}$ will do.

\begin{remark}
  It is not completely clear that
  one can choose an~$a$ such that (\ref{eqn:nonobvious}) is satisfied.
  Suppose it were impossible.  Then because the valuations are
  nontrivial, we would have that for any $a\in K$ if $\abs{a}_1<1$
  then $\abs{a}_2<1$.  This implies the converse statement: if $a\in
  K$ and $\abs{a}_2<1$ then $\abs{a}_1<1$.  To see this, suppose there
  is an $a\in K$ such that $\abs{a}_2<1$ and $\abs{a}_1\geq 1$.
  Choose $y\in K$ such that $\abs{y}_1<1$.  Then for any integer $n>0$
  we have $\abs{y/a^n}_1<1$, so by hypothesis $\abs{y/a^n}_2<1$.  Thus
  $\abs{y}_2 < \abs{a}_2^n < 1$ for all $n$.  Since $\abs{a}_2<1$ we
  have $\abs{a}_2^n\to 0$ as $n\to\infty$, so $\abs{y}_2=0$, a
  contradiction since $y\neq 0$.  Thus $\abs{a}_1<1$ if and only if
  $\abs{a}_2<1$, and we have proved before that this implies that
  $\absspc_1$ is equivalent to $\absspc_2$.
\end{remark}


Next suppose $N\geq 3$.  By the case $N-1$, there is an $a\in K$ such
that
$$
  \abs{a}_1 > 1 \qquad \text{and} \qquad \abs{a}_n<1 \quad
\text{for} \quad 2\leq n\leq N-1.
$$
By the case for $N=2$ there is a $b\in K$ such that
$$
   \abs{b}_1>1 \qquad\text{and}\qquad\abs{b}_N<1.
$$
Then put
$$
c = \begin{cases}
  a & \text{if } \abs{a}_N<1\\
  a^r\cdot b & \text{if } \abs{a}_N=1\\
  {\ds\frac{a^r}{1+a^r}}\cdot b & \text{if }\abs{a}_N>1
\end{cases}
$$
where $r\in\Z$ is sufficiently large so that
$\abs{c}_1>1$ and $\abs{c}_n<1$ for $2\leq n\leq N$.
\end{proof}

\begin{example}\label{ex:weakapprox}
  Suppose $K=\Q$, let $\absspc_1$ be the archimedean absolute value
  and let $\absspc_2$ be the $2$-adic absolute value.  Let $a_1=-1$,
  $a_2=8$, and $\eps=1/10$, as in the remark right after
  Theorem~\ref{thm:weakapprox}.  Then the theorem implies that there
  is an element $b\in \Q$ such that
$$
 \abs{-1-b}_1 < \frac{1}{10} \qquad\text{and}\qquad \abs{8-b}_2 < \frac{1}{10}.
 $$
 As in the proof of the theorem, we can find such a $b$ by finding
 a $c_1, c_2\in\Q$ such that $\abs{c_1}_1>1$ and $\abs{c_1}_2<1$, and
a $\abs{c_2}_1<1$ and $\abs{c_2}_2>1$.  For example,
 $c_1=2$ and $c_2=1/2$ works, since $\abs{2}_1 = 2$ and $\abs{2}_2 =
 1/2$ and
$\abs{1/2}_1=1/2$ and $\abs{1/2}_2=2$.  Again
 following the proof, we see that for sufficiently large $r$
we can take
\begin{align*}
   b_r &= \frac{c_1^r}{1+c_1^r} \cdot a_1 +
     \frac{c_2^r}{1+c_2^r} \cdot a_2\\
     &=\frac{2^r}{1+2^r} \cdot (-1) +
     \frac{(1/2)^r}{1+(1/2)^r} \cdot 8.
   \end{align*}
We have $b_1 = 2$, $b_2 = 4/5$, $b_3 = 0$, $b_4 = -8/17$,
$b_5 = -8/11$, $b_6 = -56/55$.  None of the $b_i$ work for $i<6$,
but $b_6$ works.
\end{example}

\begin{exercise}\label{ex:topology9}
  In this problem you will compute an example of weak
  approximation, like I did in the Example~\ref{ex:weakapprox}.  Let
  $K=\Q$, let $\absspc_7$ be the $7$-adic absolute value, let
  $\absspc_{11}$ be the $11$-adic absolute value, and let
  $\absspc_{\infty}$ be the usual archimedean absolute value.  Find an
  element $b\in \Q$ such that $\abs{b-a_i}_i<\frac{1}{10}$, where $a_7
  = 1$, $a_{11} = 2$, and $a_{\infty} = -2004$.
\end{exercise}

\begin{exercise}\label{ex:topology10}
Find the $3$-adic expansion to precision 4 of each root of the following polynomial over $\Q_3$:
$$
  f = x^3 - 3x^2 + 2x + 3 \in \Q_3[x].
$$
Your solution should conclude with three expressions of the form
$$a_0 + a_1\cdot 3 + a_2\cdot 3^2 + a_3 \cdot 3^3 + O(3^4).$$
\end{exercise}

\begin{exercise}\label{ex:topology11}
  Prove that every finite extension of
  $\Q_p$ ``comes from'' an extension of~$\Q$, in the following sense.
  Given an irreducible polynomial $f\in\Q_p[x]$ there exists an
  irreducible polynomial $g\in \Q[x]$ such that the fields
  $\Q_p[x]/(f)$ and $\Q_p[x]/(g)$ are isomorphic.  [Hint: Choose each
  coefficient of $g$ to be sufficiently close to the corresponding
  coefficient of $f$, then use Hensel's lemma to show that $g$ has a
  root in $\Q_p[x]/(f)$.]
\end{exercise}

\begin{exercise}\label{ex:topology12}
Suppose that $K$ is a finite extension of $\Q_p$ and $L$
is a finite extension of $\Q_q$, with $p\neq q$ and assume
that $K$ and $L$ have the same degree.  Prove that
there is a polynomial $g\in \Q[x]$ such that $\Q_p[x]/(g)\isom K$
and $\Q_q[x]/(g)\isom L$.  [Hint: Combine your solution to exercise \ref{ex:topology11} with the weak approximation theorem.]
\end{exercise}
\end{ch}
%%%%%%%%%%%%%%%%%%%%%%%%%%%%%%%%%%%%%%%%%%%%%%%%%%%%%%%%%%%%%%%%%%%%%%%%%%%%%%%
%%%%%%%%%%%%%%%%%%%%%%%%%%%% END TOPOLOGY %%%%%%%%%%%%%%%%%%%%%%%%%%%%%%%%%%%%%
%%%%%%%%%%%%%%%%%%%%%%%%%%%%%%%%%%%%%%%%%%%%%%%%%%%%%%%%%%%%%%%%%%%%%%%%%%%%%%%































%%%%%%%%%%%%%%%%%%%%%%%%%%%%%%%%%%%%%%%%%%%%%%%%%%%%%%%%%%%%%%%%%%%%%%%%%%%%%%%
%%%%%%%%%%%%%%%%%%%%%%%%%%%% START ADICS %%%%%%%%%%%%%%%%%%%%%%%%%%%%%%%%%%%%%%
%%%%%%%%%%%%%%%%%%%%%%%%%%%%%%%%%%%%%%%%%%%%%%%%%%%%%%%%%%%%%%%%%%%%%%%%%%%%%%%
\begin{ch}
\chapter{Adic Numbers: The Finite Residue Field Case}

\section{Finite Residue Field Case}
Let $K$ be a field with a non-archimedean valuation $v=\absspc$.
Recall that the set of $a\in K$ with $\abs{a}\leq 1$ forms a ring
$\O$, the ring of integers for $v$.  The set of $u\in K$ with
$\abs{u}=1$ are a group $U$ under multiplication, the group of units
for $v$.  Finally, the set of $a\in K$ with $\abs{a}<1$ is a maximal
ideal $\p$, so the quotient ring $\O/\p$ is a field.  In this section
we consider the case when $\O/\p$ is a finite field of order a prime
power~$q$.  For example, $K$ could be $\Q$ and $\absspc{}$ could be a
$p$-adic valuation, or $K$ could be a number field and $\absspc{}$
could be the valuation corresponding to a maximal ideal of the ring of
integers.  Among other things, we will discuss in more depth the
topological and measure-theoretic nature of the completion of $K$ at
$v$.

Suppose further for the rest of this section that $\absspc{}$ is
discrete.  Then by Lemma~\ref{lem:discrete_principal}, the ideal $\p$
is a principal ideal $(\pi)$, say, and every $a\in K$ is of the form
$a=\pi^n\eps$, where $n\in\Z$ and $\eps\in U$ is a unit. We call
$$
n = \ord(a) = \ord_\pi(a) = \ord_\p(a) = \ord_v(a)
$$
the ord of~$a$ at~$v$.  (Some authors, including me (!) also call
this integer the \defn{valuation} of~$a$ with respect to~$v$.)  If
$\p=(\pi')$, then $\pi/\pi'$ is a unit, and conversely, so $\ord(a)$
is independent of the choice of~$\pi$.

Let $\O_v$ and $\p_v$ be defined with respect to the completion $K_v$
of $K$ at $v$.
\begin{lemma}\ilem{reduction homomorphism}
There is a natural isomorphism
$$
\vphi:\O_v/\p_v \to \O/\p,
$$
and $\p_v = (\pi)$ as an $\O_v$-ideal.
\end{lemma}
\begin{proof}
  Since we are assuming that $\abs{\cdot}$ is discrete, we may view $\O_v$ as the set of equivalence classes of Cauchy
  sequences $(a_n)$ in $K$ such that $a_n\in \O$ for $n$ sufficiently
  large, and similarly $\p_v$ as those such that $a_n \in \p$ for $n$ sufficiently large.  For any $\eps$, given such a sequence $(a_n)$, there is $N$
  such that for $n,m\geq N$, we have $\abs{a_n-a_m}<\eps$.  In
  particular, we can choose $N$ such that $n,m\geq N$ implies that
  $a_n\con a_m\pmod{\p}$.  Let $\vphi((a_n)) = a_N\pmod{\p}$, which is
  well-defined.  The map $\vphi$ is surjective because the constant
  sequences are in $\O_v$.  Its kernel is the set of Cauchy sequences
  whose elements are eventually all in $\p$, which is exactly $\p_v$.
  This proves the first part of the lemma.  The second part is true
  because any element of $\p_v$ is a sequence all of whose terms are
  eventually in $\p$, hence all a multiple of $\pi$ (we can set to $0$
  a finite number of terms of the sequence without changing the
  equivalence class of the sequence).
\end{proof}

{\em Assume for the rest of this section that $K$ is complete with
  respect to $\absspc$.}
\begin{lemma}\ilem{adic-expansion}
Then ring $\O$ is precisely the set of infinite sums
\begin{equation}\label{eq:infsum}
  a = \sum_{j=0}^{\infty} a_j \cdot \pi^j
\end{equation}
where the $a_j$ run independently through some set $\cR$ of
representatives of $\O$ in $\O/\p$.
\end{lemma}
By (\ref{eq:infsum}) is meant the limit of the Cauchy sequence
$\sum_{j=0}^n a_j\cdot \pi^j$ as $j\to\infty$.
\begin{proof}
There is a uniquely defined $a_0\in \cR$ such that $\abs{a-a_0}<1$.
Then $a' = \pi^{-1}\cdot (a-a_0) \in \O$.  Now define
$a_1\in \cR$ by $\abs{a'-a_1}<1$.  And so on.
\end{proof}
\begin{example}
  Suppose $K=\Q$ and $\absspc=\absspc_p$ is the $p$-adic valuation,
  for some prime~$p$.  We can take $\cR=\{0,1,\ldots, p-1\}$.
  The lemma asserts that
  $$\O=\Z_p = \left\{ \sum_{j=0}^{\infty} a_n p^n : 0\leq a_n\leq
    p-1\right\}.$$
  Notice that $\O$ is uncountable since there are $p$
  choices for each $p$-adic ``digit''.  We can do arithmetic with
  elements of $\Z_p$, which can be thought of ``backwards'' as numbers
  in base $p$.  For example, with $p=3$ we have
  \begin{align*}
&   (1+2\cdot 3 + 3^2 + \cdots ) + (2 + 2\cdot 3 + 3^2 + \cdots ) \\
& = 3+4\cdot 3 + 2\cdot 3^2 + \cdots   \qquad \text{not in canonical form}\\
& = 0 + 2\cdot 3 + 3\cdot 3 + 2\cdot 3^2 + \cdots \qquad\text{still not canonical}\\
& = 0 + 2\cdot 3 + 0\cdot 3^2 + \cdots
\end{align*}

% Basic arithmetic with the $p$-adics in \magma{} is really weird (even
% weirder than it was a year ago...  There are presumably efficiency
% advantages to using the \magma{} formalization, and it's supposed to be
% better for working with extension fields.  But I can't get it to do
% even the calculation below in a way that is clear.)
Here is an example of doing basic arithmetic with $p$-adic
numbers in Sage:
\begin{lstlisting}
sage: a = 1 + 2*3 + 3^2 + O(3^3)
sage: b = 2 + 2*3 + 3^2 + O(3^3)
sage: a + b
2*3 + O(3^3)
sage: sqrt(a)
1 + 3 + O(3^3)
sage: sqrt(a)^2
1 + 2*3 + 3^2 + O(3^3)
sage: a * b
2 + O(3^3)
\end{lstlisting}
Type {\tt Zp?} and {\tt Qp?} in Sage for much more information
about the various computer models of $p$-adic arithmetic that
are available.

% In PARI (gp) the
% $p$-adics work as expected:
% \begin{verbatim}
%    ? a = 1 + 2*3 + 3^2 + O(3^3);
%    ? b = 2 + 2*3 + 3^2 + O(3^3);
%    ? a+b
%    %3 = 2*3 + O(3^3)
%    ? sqrt(1+2*3+O(3^20))
%    %5 = 1 + 3 + 3^2 + 2*3^4 + 2*3^7 + 3^8 + 3^9 + 2*3^10 + 2*3^12
%           + 2*3^13 + 2*3^14 + 3^15 + 2*3^17 + 3^18 + 2*3^19 + O(3^20)
%    ? 1/sqrt(1+2*3+O(3^20))
%    %6 = 1 + 2*3 + 2*3^2 + 2*3^7 + 2*3^10 + 2*3^11 + 2*3^12 + 2*3^13
%           + 2*3^14 + 3^15 + 2*3^16 + 2*3^17 + 3^18 + 3^19 + O(3^20)
% \end{verbatim}
\end{example}

\begin{theorem}\label{thm:compact}\ithm{compactness of ring of integers}
Under the conditions of the preceding lemma, $\O$ is compact with
respect to the $\absspc{}$-topology.
\end{theorem}
\begin{proof}
  Let $V_\lambda$, for $\lambda$ running through some index set
  $\Lambda$, be some family of open sets that cover $\O$.  We must
  show that there is a finite subcover.  We suppose not.

  Let $\cR$ be a set of representatives for $\O/\p$.  Then $\O$ is the
  union of the finite number of cosets $a+\pi\O$, for $a\in \cR$.
    Hence for at lest one $a_0\in \cR$ the set $a_0+\pi \O$
    is not covered by finitely many of the $V_\lambda$.  Then similarly
  there is an $a_1\in \cR$ such that $a_0 + a_1\pi + \pi^2\O$ is not
  finitely covered. And so on.  Let
  $$
  a = a_0 + a_1\pi + a_2 \pi^2 + \cdots \in \O.
  $$
  Then $a\in V_{\lambda_0}$ for some $\lambda_0\in\Lambda$.  Since
  $V_{\lambda_0}$ is an open set, $a+\pi^J\cdot \O\subset
  V_{\lambda_0}$ for some~$J$ (since those are exactly the open balls
  that form a basis for the topology). This is a contradiction because
  we constructed $a$ so that none of the sets $a+\pi^n\cdot \O$, for
  each $n$, are not covered by any finite subset of the $V_{\lambda}$.
\end{proof}

\begin{definition}[Locally compact]
  A topological space $X$ is \defn{locally compact} at a point $x$ if
  there is some compact subset $C$ of $X$ that contains a neighborhood
  of~$x$.  The space $X$ is locally compact if it is locally compact
  at each point in $X$.
\end{definition}
\begin{corollary}\label{cor:locally_compact}\icor{complete local field locally compact}
The complete local field $K$ is locally compact.
\end{corollary}
\begin{proof}
  If $x\in K$, then $x \in C=x+\O$, and $C$ is a compact subset of $K$
  by Theorem~\ref{thm:compact}.  Also $C$ contains the neighborhood
  $x+\pi\O = B(x,1)$ of $x$.  Thus $K$ is locally compact at $x$.
\end{proof}

\begin{remark}\label{rem:locally_compact}
The converse is also true.  If $K$ is locally compact with respect to
a non-archimedean valuation $\absspc{}$, then
\begin{enumerate}
\item $K$ is complete,
\item the residue field is finite, and
\item the valuation is discrete.
\end{enumerate}
For there is a compact neighbourhood $C$ of $0$.
Let $\pi$ be any nonzero with $\abs{\pi}<1$.
Then $\pi^n\cdot
\O\subset C$ for sufficiently large $n$, so $\pi^n\cdot \O$ is
compact, being closed.  Hence $\O$ is compact.  Since $\absspc$ is a
metric, $\O$ is sequentially compact, i.e., every fundamental sequence
in $\O$ has a limit, which implies (1).  Let $a_\lambda$ (for
$\lambda\in\Lambda$) be a set of representatives in $\O$ of $\O/\p$.
Then $\O_{\lambda} = \{z : \abs{z-a_{\lambda}}<1\}$ is an open
covering of $\O$.  Thus (2) holds since $\O$ is compact.  Finally,
$\p$ is compact, being a closed subset of $\O$.  Let $S_n$ be the set
of $a\in K$ with $\abs{a}<1-1/n.$  Then $S_n$ (for $1\leq n < \infty$)
is an open covering of $\p$, so $\p=S_n$ for some $n$, i.e., (3) is
true.

If we allow $\absspc{}$ to be archimedean the only further
possibilities are $k=\R$ and $k=\C$ with $\absspc{}$ equivalent to the
usual absolute value.
\end{remark}

We denote by $K^+$ the commutative topological group whose points are
the elements of $K$, whose group law is addition and whose topology is
that induced by $\absspc$.  General theory tells us that there is an
invariant Haar measure defined on $K^+$ and that this
measure is unique up to a multiplicative constant.

\begin{definition}[Haar Measure]\label{defn:haar}
A \defn{Haar measure} on a locally compact topological group
$G$ is a translation invariant measure such that every open
set can be covered by open sets with finite measure.
\end{definition}

\begin{lemma}\ilem{Haar measure on compact}
  Haar measure of any compact subset $C$ of $G$ is finite.
\end{lemma}
\begin{proof}
The whole group $G$ is open, so there is a covering $U_\alpha$
of $G$ by open sets each of which has finite measure.
Since $C$ is compact, there is a finite subset of the $U_\alpha$
that covers $C$.  The measure of $C$ is at most the sum of
the measures of these finitely many $U_\alpha$, hence finite.
\end{proof}

\begin{remark}
  Usually one defined Haar measure to be a translation invariant
  measure such that the measure of compact sets is finite.  Because of
  local compactness, this definition is equivalent to
  Definition~\ref{defn:haar}.  We take this alternative viewpoint
  because Haar measure is constructed naturally on the topological
  groups we will consider by defining the measure on each member of a
  basis of open sets for the topology.
\end{remark}

We now deduce what any such measure $\mu$ on $G=K^+$ must be.  Since
$\O$ is compact (Theorem~\ref{thm:compact}), the measure of $\O$ is
finite.  Since $\mu$ is translation invariant,
$$
  \mu_n = \mu(a + \pi^n \O)
$$
is independent of $a$.  Further,
$$
a + \pi^n\O = \bigcup_{1\leq j\leq q} a + \pi^n a_j + \pi^{n+1}\O,
\qquad\text{(disjoint union)}
$$
where $a_j$ (for $1\leq j \leq q$) is a set of representatives of
$\O/\p$. Hence
$$
 \mu_n = q\cdot \mu_{n+1}.
$$
If we normalize $\mu$ by putting
$$
 \mu(\O) = 1
 $$
 we have $\mu_0 = 1$, hence $\mu_1 = q^{-1}$, and in general $$\mu_n =
 q^{-n}.$$

 Conversely, without the theory of Haar measure, we could {\em define}
 $\mu$ to be the necessarily unique measure on $K^+$ such that
 $\mu(\O)=1$ that is translation invariant.  This would have to be the
 $\mu$ we just found above.

 Everything so far in this section has depended not on the valuation
 $\absspc$ but only on its equivalence class.  The above
 considerations now single out one valuation in the equivalence class
 as particularly important.
\begin{definition}[Normalized valuation]\label{defn:normalized}
Let $K$ be a field equipped with a discrete valuation $\absspc$
and residue class field with $q<\infty$ elements.  We say that
$\absspc$ is \defn{normalized} if
$$
\abs{\pi} = \frac{1}{q},
$$
where $\p=(\pi)$ is the maximal ideal of $\O$.
\end{definition}
\begin{example}
The normalized valuation on the $p$-adic numbers $\Q_p$ is
$\abs{u\cdot p^n} = p^{-n}$, where $u$ is a rational number
whose numerator and denominator are coprime to $p$.

Next suppose $K=\Q_p(\sqrt{p})$.  Then the $p$-adic valuation on
$\Q_p$ extends uniquely to one on $K$ such that
$\abs{\sqrt{p}}^2 = \abs{p} = 1/p$.  Since $\pi=\sqrt{p}$
for $K$, this valuation is not normalized.  (Note that
the ord of $\pi=\sqrt{p}$ is $1/2$.)
The normalized valuation is $v=\absspc' = \absspc^2$.  Note that
$\abs{p}' = 1/p^2$, or $\ord_v(p)=2$ instead of $1$.

Finally suppose that $K=\Q_p(\sqrt{q})$ where $x^2-q$
has no root mod $p$.  Then the residue class field
degree is $2$, and the normalized valuation must
satisfy $\abs{\sqrt{q}} = 1/p^2$.
\end{example}


The following proposition makes clear why this is the best choice of
normalization.
\begin{theorem}\ithm{properties of Haar measure}
Suppose further that $K$ is complete with respect to the normalized
valuation $\absspc{}$.  Then
$$
\mu(a + b\O) = \abs{b},
$$
where $\mu$ is the Haar measure on $K^+$ normalized so
that $\mu(\O)=1$.
\end{theorem}
\begin{proof}
Since $\mu$ is translation invariant, $\mu(a+b\O) = \mu(b\O)$.
Write $b=u\cdot \pi^n$, where $u$ is a unit. Then since $u\cdot
\O=\O$, we have
$$\mu(b\O)
  = \mu(u\cdot \pi^n\cdot \O) = \mu(\pi^n \cdot u\cdot\O)
    = \mu(\pi^n\cdot \O) = q^{-n} = \abs{\pi^n} = \abs{b}.
$$
Here we have $\mu(\pi^n\cdot \O) = q^{-n}$ by the discussion
before Definition~\ref{defn:normalized}.
\end{proof}

\begin{exercise}\label{ex:adics1}
\begin{enumerate}
\item Find the normalized Haar measure of the following subset of
$\Q_7^+$:
$$
U = B\left(28,\frac{1}{50}\right) =
\left\lbrace x\in \Q_7 : \abs{x-28} < \frac{1}{50}\right\rbrace.
$$
\item
Find the normalized Haar measure of the subset $\Z_7^*$ of
$\Q_7^*$.
\end{enumerate}
\end{exercise}

We can express the result of the theorem in a more suggestive way.
Let $b\in K$ with $b\neq 0$, and let $\mu$ be a Haar measure on $K^+$
(not necessarily normalized as in the theorem).  Then we can define a
new Haar measure $\mu_b$ on $K^+$ by putting $\mu_b(E) = \mu(bE)$ for
$E\subset K^+$.  But Haar measure is unique up to a multiplicative
constant and so $\mu_b(E) = \mu(bE) = c\cdot \mu(E)$ for all
measurable sets $E$, where the factor~$c$ depends only on~$b$.
Putting $E=\O$, shows that the theorem implies that~$c$ is just
$\abs{b}$, when $\absspc$ is the normalized valuation.

\begin{remark}
  The theory of locally compact topological groups leads to the
  consideration of the dual (character) group of $K^+$.  It turns out
  that it is isomorphic to $K^+$.  We do not need this fact for class
  field theory, so do not prove it here.  For a proof and applications
  see Tate's thesis or Lang's {\em Algebraic Numbers}, and for
  generalizations see Weil's {\em Adeles and Algebraic Groups} and
  Godement's Bourbaki seminars 171 and 176.  The determination of the
  character group of $K^*$ is local class field theory.
\end{remark}

The set of nonzero elements of~$K$ is a group $K^*$ under
multiplication.  Multiplication and inverses are continuous with
respect to the topology induced on $K^*$ as a subset of $K$, so $K^*$
is a topological group with this topology.  We have
$$
  U_1 \subset U \subset K^*
  $$
  where $U$ is the group of units of $\O\subset K$ and $U_1$ is
  the group of $1$-units, i.e., those units $\eps\in U$ with
  $\abs{\eps-1}<1$, so
  $$U_1 = 1 + \pi\O.$$
  The set $U$ is open because of the discreteness of the metric, and $U$ is closed because $U = \O \setminus \p$ and we already proved that $\O$ is closed and $\p$ is open in this case.  Likewise, $U_1$ is both open and closed.

  The quotient $K^*/U = \{ \pi^n \cdot U : n \in \Z\}$ is isomorphic to the additive group $\Z^+$
of integers with the discrete topology, where the map is
$$
 \pi^n\cdot U \mapsto n  \qquad \text{ for } n\in\Z.
 $$

The quotient
 $U/U_1$ is isomorphic to the multiplicative group $\F^*$ of the
 nonzero elements of the residue class field $\F = \O/\p$, where the finite group
 $\F^*$ has the discrete topology.
Note that $\F^*$ is cyclic
 of order $q-1$, and Hensel's lemma implies that $K^*$ contains a
 primitive $(q-1)$th root of unity $\zeta$.  Thus $K^*$ has
the following structure:
$$
K^* = \{ \pi^n\zeta^m\eps : n\in \Z, m\in\Z/(q-1)\Z, \eps\in U_1\}
\isom \Z\,\, \cross\,\, \Z/(q-1)\Z \,\,\cross\,\, U_1.
$$
(How to apply Hensel's lemma: Let $f(x) = x^{q-1}-1$ and let
$a\in\O$ be such that $a\mod\p$ generates $K^*$.  Then $\abs{f(a)}<1$
and $\abs{f'(a)}=1$.  By Hensel's lemma  there is a
$\zeta\in K$ such that $f(\zeta)=0$ and $\zeta\con a\pmod{\p}$.)

Since $U$ is compact and the cosets of $U$ cover $K$, we see that
$K^*$ is locally compact.
\begin{lemma}\ilem{Haar measure on $K^*$}
The additive Haar measure $\mu$ on $K^+$,
when restricted to $U_1$ gives a measure on $U_1$ that is also
invariant under multiplication, so gives a Haar measure on $U_1$.
\end{lemma}
\begin{proof}
It suffices to show that
$$\mu(1+\pi^n\O) = \mu(u\cdot(1+\pi^n\O)),$$
for any $u\in U_1$ and $n>0$.
Write $u=1+a_1\pi + a_2\pi^2 +\cdots$.
We have
\begin{align*}
 u\cdot (1+\pi^n\O) &= (1+a_1\pi + a_2\pi^2 +\cdots)\cdot (1+\pi^n\O)\\
  &= 1+a_1\pi + a_2\pi^2 + \cdots + \pi^n\O\\
  &= a_1\pi + a_2\pi^2 + \cdots + ( 1+\pi^n\O),
\end{align*}
which is an additive translate of $1+\pi^n\O$, hence has the
same measure.
\end{proof}
Thus $\mu$ gives a Haar measure on $K^*$ by translating $U_1$ around
to cover $K^*$.

\begin{lemma}\ilem{$K^+$ and $K^*$ are totally disconnected}
The topological spaces $K^+$ and $K^*$ are totally disconnected (the
only connected sets are points).
\end{lemma}
\begin{proof}
  The proof is the same as that of
  Proposition~\ref{prop:disconnected}.  The point is that the
  non-archimedean triangle inequality forces the complement of an open
  disc to be open, hence any set with at least two distinct elements
  ``falls apart'' into a disjoint union of two disjoint open subsets.
\end{proof}

\begin{remark}
Note that $K^*$ and $K^+$ are locally isomorphic if $K$ has
characteristic~$0$.   We have the exponential map
$$
a \mapsto \exp(a) = \sum_{n=0}^{\infty} \frac{a^n}{n!}
$$
defined for all sufficiently small $a$ with its inverse
$$
\log(a) = \sum_{n=1}^{\infty} \frac{(-1)^{n-1}(a-1)^n}{n},
$$
which is defined for all $a$ sufficiently close to $1$.
\end{remark}
\end{ch}
%%%%%%%%%%%%%%%%%%%%%%%%%%%%%%%%%%%%%%%%%%%%%%%%%%%%%%%%%%%%%%%%%%%%%%%%%%%%%%%
%%%%%%%%%%%%%%%%%%%%%%%%%%%% END ADICS %%%%%%%%%%%%%%%%%%%%%%%%%%%%%%%%%%%%%%%%
%%%%%%%%%%%%%%%%%%%%%%%%%%%%%%%%%%%%%%%%%%%%%%%%%%%%%%%%%%%%%%%%%%%%%%%%%%%%%%%








































%%%%%%%%%%%%%%%%%%%%%%%%%%%%%%%%%%%%%%%%%%%%%%%%%%%%%%%%%%%%%%%%%%%%%%%%%%%%%%%
%%%%%%%%%%%%%%%%%%%%%%%%%%%% START NORMED %%%%%%%%%%%%%%%%%%%%%%%%%%%%%%%%%%%%%
%%%%%%%%%%%%%%%%%%%%%%%%%%%%%%%%%%%%%%%%%%%%%%%%%%%%%%%%%%%%%%%%%%%%%%%%%%%%%%%
\begin{ch}
\chapter{Normed Spaces and Tensor Products}
Much of this chapter is preparation for what we will do later
when we will prove that if~$K$ is complete with respect to a valuation
(and locally compact) and~$L$ is a finite extension of~$K$, then there
is a {\em unique} valuation on~$L$ that extends the valuation on~$K$.
Also, if~$K$ is a number field, $v=\absspc{}$ is a valuation on~$K$,
$K_v$ is the completion of~$K$ with respect to~$v$, and~$L$ is a
finite extension of~$K$, we'll prove that
$$
 K_v \tensor_K L   = \bigoplus_{j=1}^J L_j,
$$
 where the $L_j$ are the completions of~$L$ with respect to the
 equivalence classes of extensions of~$v$ to~$L$.  In particular,
 if~$L$ is a number field defined by a root of $f(x)\in \Q[x]$, then
$$
  \Q_p  \tensor_\Q  L = \bigoplus_{j=1}^J L_j,
$$
 where the $L_j$ correspond to the irreducible factors of
 the polynomial $f(x) \in \Q_p[x]$ (hence the extensions of
$\absspc_p$ correspond to irreducible factors of $f(x)$
over $\Q_p[x]$).

In preparation for this clean view of the local nature of number
fields, we will prove that the norms on a finite-dimensional
vector space over a complete field are all equivalent.  We will also
explicitly construct tensor products of fields and deduce some of
their properties.

\section{Normed Spaces}
\begin{definition}[Norm]\label{defn:norm}
Let $K$ be a field with valuation $\absspc$ and let $V$ be a vector space
over $K$.  A real-valued function $\normspc$ on $V$ is called a \defn{norm} if
\begin{enumerate}
\item $\norm{v}>0$ for all nonzero $v\in V$ (positivity).
\item $\norm{v+w} \leq \norm{v} + \norm{w}$ for all $v,w\in V$ (triangle inequality).
\item $\norm{av} = \abs{a}\norm{v}$ for all $a\in K$ and $v\in V$ (homogeneity).
\end{enumerate}
\end{definition}
Note that setting $\norm{v}=1$ for all $v\neq 0$ does {\em not} define
a norm unless the absolute value on $K$ is trivial, as $1=\norm{av} =
\abs{a}\norm{v}=\abs{a}$.  We assume for the rest of this section
that $\absspc{}$ is not trivial.

\begin{definition}[Equivalent]
  Two norms $\normspc_1$ and $\normspc_2$ on the same vector space~$V$
  are \defn{equivalent} if there exists positive real numbers $c_1$ and $c_2$
  such that for all $v\in V$
$$
  \norm{v}_1 \leq c_1 \norm{v}_2
  \qquad\text{and}\qquad
  \norm{v}_2 \leq c_2 \norm{v}_1.
$$
\end{definition}

\begin{exercise}{ex:normed1}
Suppose $\normspc_1$ and $\normspc_2$ are
equivalent norms on a finite-dimensional vector space
$V$ over a field $K$ (with valuation $\absspc$).
Carefully prove that the topology induced by $\normspc_1$
is the same as that induced by $\normspc_2$.
\end{exercise}


\begin{lemma}\label{lem:ext_unique}\ilem{any two norms equivalent}
Suppose that $K$ is a field that is complete with respect to a valuation
$\absspc{}$ and that $V$ is a finite dimensional~$K$ vector space.
Then any two norms on $V$ are equivalent.
\end{lemma}
\begin{remark}
  As we shall see soon (see Theorem~\ref{thm:extensions}), the lemma
  is usually false if we do not assume that~$K$ is complete.  For
  example, when $K=\Q$ and $\absspc_p$ is the $p$-adic valuation, and
  $V$ is a number field, then there may be several extensions of
  $\absspc_p$ to inequivalent norms on $V$.
\end{remark}
If two norms are equivalent then the corresponding topologies on~$V$
are equal, since very open ball for $\normspc_1$ is contained in an
open ball for $\normspc_2$, and conversely. (The converse is also
true, since, as we will show, all norms on~$V$ are equivalent.)
\begin{proof}
Let $v_1,\ldots, v_N$ be a basis for~$V$.  Define the max
norm $\normspc_0$ by
$$
\norm{\sum_{n=1}^N a_n v_n}_0 = \max \left\{\abs{a_n} : n=1,\ldots, N\right\}.
$$
It is enough to show that any norm $\normspc$ is equivalent to
$\normspc_0$.  We have
\begin{align*}
\norm{\sum_{n=1}^N a_n v_n} & \leq
    \sum_{n=1}^N \abs{a_n} \norm{v_n} \\
    &\leq \sum_{n=1}^N \max{\abs{a_n}} \norm{v_n}\\
    & = c_1 \cdot \norm{\sum_{n=1}^N a_n v_n}_0,
\end{align*}
where $c_1 = \sum_{n=1}^N \norm{v_n}$.

To finish the proof, we show that there is a
$c_2\in \R$ such that for all $v\in V$,
$$
 \norm{v}_0 \leq c_2 \cdot \norm{v}.
$$
We will only prove this in the case when $K$ is not just merely complete
with respect to $\absspc{}$ but also locally compact.  This will
be the case of primary interest to us.  For a proof in the general case,
see the original article by Cassels (page 53).

By what we have already shown, the function $\norm{v}$ is continuous
in the $\normspc_0$-topology, so by local compactness it attains its
lower bound $\delta$ on the unit circle $\left\{v\in V :
  \norm{v}_0=1\right\}$.  (Why is the unit circle compact?  With
respect to $\normspc_0$, the topology on $V$ is the same as that of a
product of copies of $K$.  If the valuation is archimedean then
$K\isom \R$ or $\C$ with the standard topology and the unit circle is
compact.  If the valuation is non-archimedean, then we saw (see
Remark~\ref{rem:locally_compact}) that if~$K$ is locally compact, then
the valuation is discrete, in which case we showed that the unit disc
is compact, hence the unit circle is also compact since it is closed.)
Note that $\delta>0$ by part 1 of Definition~\ref{defn:norm}.  Also,
by definition of $\normspc_0$, for any $v\in V$ there exists $a\in K$
such that $\norm{v}_0 = \abs{a}$ (just take the max coefficient in our
basis).  Thus we can write any $v\in V$ as $a\cdot w$ where $a\in K$
and $w\in V$ with $\norm{w}_0=1$.  We then have
$$
\frac{\norm{v}_0}{\norm{v}} =
\frac{\norm{aw}_0}{\norm{aw}}
= \frac{\abs{a}\norm{w}_0}{\abs{a}\norm{w}}
= \frac{1}{\norm{w}} \leq \frac{1}{\delta}.
$$
Thus for
all~$v$ we have
$$\norm{v}_0\leq c_2\cdot \norm{v},$$
where $c_2 = 1/\delta$, which proves the theorem.
\end{proof}


\section{Tensor Products} \label{sec:tensor}
We need only a special case of the tensor product construction.
Let~$A$ and~$B$ be commutative rings containing a field~$K$ and suppose that~$B$ is of finite dimension~$N$ over~$K$, say, with basis
$$
  1=w_1, w_2, \ldots, w_N.
$$
Then~$B$ is determined up to isomorphism as a ring over~$K$
by the multiplication table $(c_{i,j,n})$ defined by
$$
   w_i \cdot w_j = \sum_{n=1}^N c_{i,j,n} \cdot w_n.
$$
We define a new ring~$C$ containing~$K$ whose elements are
the set of all expressions
$$
\sum_{n=1}^N a_n \ww_n
$$
where the $\ww_n$ have the same multiplication rule
$$
   \ww_i \cdot \ww_j = \sum_{n=1}^N c_{i,j,n} \cdot \ww_n
$$
as the $w_n$.

There are injective ring homomorphisms
$$
i:A\hra C, \qquad i(a) = a \ww_1  \qquad \text{(note that $\ww_1=1$)}
$$
and
$$
j:B\hra C, \qquad j\left(\sum_{n=1}^N c_n w_n\right) = \sum_{n=1}^N c_n \ww_n.
\qquad\quad\,\,\,\,\mbox{}
$$
Moreover~$C$ is defined, up to isomorphism, by~$A$ and~$B$ and is
independent of the particular choice of basis $w_n$ of~$B$ (i.e., a
change of basis of $B$ induces a canonical isomorphism of the $C$
defined by the first basis to the $C$ defined by the second basis).
We write
$$
  C = A\tensor_K B
$$
since~$C$ is, in fact, a special case of the ring tensor product.

\begin{exercise}\label{ex:normed2}
Prove that the ring $C$ defined in Section~\ref{sec:tensor} really is the tensor
product of $A$ and $B$, i.e., that it satisfies the defining universal
mapping property for tensor products.  Part of this problem is for you
to look up a functorial definition of tensor product.
\end{exercise}



Let us now suppose, further, that~$A$ is a topological ring, i.e., has
a topology with respect to which addition and multiplication are
continuous.  Then the map
$$
C\to A \oplus \cdots \oplus A,\qquad
  \sum_{m=1}^N a_m \ww_m \mapsto (a_1,\ldots, a_N)
$$
defines a bijection between~$C$ and the product of~$N$ copies of~$A$
(considered as sets). We give~$C$ the product topology.  It is readily
verified that this topology is independent of the choice of basis
$w_1, \ldots, w_N$ and that multiplication and addition on~$C$ are
continuous, so~$C$ is a topological ring.  We call this topology
on~$C$ the \defn{tensor product topology}.

Now drop our assumption that~$A$ and~$B$ have a topology, but suppose
that~$A$ and~$B$ are not merely rings but fields.  Recall that a
finite extension $L/K$ of fields is \defn{separable} if the number of
embeddings $L\hra \Kbar$ that fix~$K$ equals the degree of~$L$
over~$K$, where $\Kbar$ is an algebraic closure of~$K$.  The primitive
element theorem from Galois theory asserts that any such extension is
generated by a single element, i.e., $L=K(a)$ for some $a\in L$.
\begin{lemma}\label{lem:tensor_prod}\ilem{structure of tensor product of fields}
  Let~$A$ and~$B$ be fields containing the field~$K$ and suppose
  that~$B$ is a separable extension of finite degree $N=[B:K]$.  Then
  $C=A\tensor_K B$ is the direct sum of a finite number of fields
  $K_j$, each containing an isomorphic image of~$A$ and an isomorphic
  image of~$B$.
\end{lemma}
\begin{proof}
  By the primitive element theorem, we have $B=K(b)$, where~$b$ is a
  root of some separable irreducible polynomial $f(x)\in K[x]$ of
  degree~$N$.  Then $1,b,\ldots, b^{N-1}$ is a basis
for~$B$ over~$K$, so
$$
  A\tensor_K B = A[\bb] \isom A[x]/(f(x))
$$
where $1,\bb,\bb^2,\ldots,\bb^{N-1}$ are
linearly independent over~$A$ and $\bb$ satisfies
$f(\bb)=0$.

Although the polynomial $f(x)$ is irreducible as an element
of $K[x]$, it need not be irreducible in $A[x]$.  Since~$A$
is a field,  we have a factorization
$$
   f(x) = \prod_{j=1}^J g_j(x)
$$
where $g_j(x)\in A[x]$ is irreducible.  The $g_j(x)$ are
distinct because $f(x)$ is separable (i.e., has distinct
roots in any algebraic closure).

For each~$j$, let $\bb_j\in \overline{A}$ be a root of $g_j(x)$, where
$\overline{A}$ is a fixed
algebraic closure of the field~$A$.  Let $K_j = A(\bb_j)$.
Then the map
\begin{equation}\label{eqn:tensor}
  \vphi_j : A\tensor_K B \to K_j
\end{equation}
given by sending any polynomial $h(\bb)$ in $\bb$ (where $h\in A[x]$)
to $h(\bb_j)$ is a ring homomorphism, because the image
of~$\bb$ satisfies the polynomial $f(x)$, and $A\tensor_K B\isom A[x]/(f(x))$.

By the Chinese Remainder Theorem, the maps from (\ref{eqn:tensor})
combine to define a ring isomorphism
$$
 A\tensor_K B \isom A[x]/(f(x)) \isom \bigoplus_{j=1}^J A[x]/(g_j(x))
   \isom \bigoplus_{j=1}^J K_j.
$$

Each $K_j$ is of the form $A[x]/(g_j(x))$, so contains an isomorphic
image of $A$.  It thus remains to show that the ring
homomorphisms
$$
  \lambda_j : B \xra{b\,\mapsto 1\tensor b} A\tensor_K B \xra{\vphi_j} K_j
$$
are injections.  Since $B$ and $K_j$ are both fields, $\lambda_j$
is either the $0$ map or injective.  However, $\lambda_j$ is
not the $0$ map since $\lambda_j(1)=1\in K_j$.
\end{proof}
\begin{example}
  If $A$ and $B$ are finite extensions of $\Q$, then $A\tensor_\Q B$
  is an algebra of degree $[A:\Q]\cdot [B:\Q]$. For example, suppose
  $A$ is generated by a root of $x^2+1$ and $B$ is generated by a root
  of $x^3-2$.  We can view $A\tensor_\Q B$ as either $A[x]/(x^3-2)$ or
  $B[x]/(x^2+1)$.  The polynomial $x^2+1$ is irreducible over $\Q$,
  and if it factored over the cubic field $B$, then there would be a
  root of $x^2+1$ in $B$, i.e., the quadratic field $A=\Q(i)$ would be
  a subfield of the cubic field $B=\Q(\sqrt[3]{2})$, which is
  impossible.  Thus $x^2+1$ is irreducible over $B$, so $A\tensor_\Q B
  = A.B = \Q(i,\sqrt[3]{2})$ is a degree $6$ extension of $\Q$.
  Notice that $A.B$ contains a copy~$A$ and a copy of~$B$. By the
  primitive element theorem the composite field $A.B$ can be generated
  by the root of a single polynomial. For example, the minimal
  polynomial of $i+\sqrt[3]{2}$ is $x^6 + 3x^4 - 4x^3 + 3x^2 + 12x +
  5$, hence $\Q(i+\sqrt[3]{2})=A.B$.
\end{example}

\begin{example}
  The case $A\isom B$ is even more exciting.  For example, suppose
  $A=B=\Q(i)$. Using the Chinese Remainder Theorem we have that
$$
  \Q(i)\tensor_\Q \Q(i) \isom \Q(i)[x]/(x^2+1)
\isom \Q(i)[x]/((x-i)(x+i))
\isom \Q(i) \oplus \Q(i),
$$
since $(x-i)$ and $(x+i)$ are coprime.  The last isomorphism
sends $a + b x$, with $a,b\in\Q(i)$, to $(a+bi, a-bi)$.
Since $\Q(i)\oplus \Q(i)$ has zero divisors, the tensor
product $\Q(i)\tensor_\Q \Q(i)$ must also have zero divisors.
For example, $(1,0)$ and $(0,1)$ is a zero divisor pair
on the right hand side, and we can trace back to the elements
of the tensor product that they define.  First, by solving
the system
$$ a+bi=1\qquad \text{ and }\qquad a-bi=0$$
we see that
$(1,0)$ corresponds to $a=1/2$ and $b=-i/2$, i.e., to the element
$$\frac{1}{2}- \frac{i}{2} x\in \Q(i)[x]/(x^2+1).$$
This element in turn
corresponds to
$$
\frac{1}{2}\tensor 1 - \frac{i}{2}\tensor i \in \Q(i)\tensor_\Q\Q(i).
$$
Similarly the other element $(0,1)$ corresponds to
$$
 \frac{1}{2}\tensor 1 + \frac{i}{2}\tensor i \in \Q(i)\tensor_\Q\Q(i).
$$
As a double check, observe that
\begin{align*}
\left(\frac{1}{2}\tensor 1 - \frac{i}{2}\tensor i\right)\cdot
 \left(\frac{1}{2}\tensor 1 + \frac{i}{2}\tensor i\right)
&= \frac{1}{4}\tensor 1 + \frac{i}{4}\tensor i - \frac{i}{4}\tensor i
    -\frac{i^2}{4}\tensor i^2\\
 &= \frac{1}{4}\tensor 1 - \frac{1}{4}\tensor 1 = 0 \in \Q(i)\tensor_\Q\Q(i).
\end{align*}
Clearing the denominator of $2$ and writing $1\tensor 1 = 1$, we have
$(1-i\tensor i)(1+i\tensor i) = 0$, so $i\tensor i$ is a root of the
polynomimal $x^2-1$, and $i\tensor i$ is not $\pm 1$, so $x^2-1$ has
more than $2$ roots.

In general, to understand $A\tensor_K B$ explicitly
is the same as factoring either the defining polynomial of~$B$
over the field~$A$, or factoring the defining polynomial of~$A$
over~$B$.
\end{example}

\begin{exercise}\label{ex:normed3}
Find a zero divisor pair in $\Q(\sqrt{5})\tensor_\Q\Q(\sqrt{5})$.
\end{exercise}

\begin{exercise}\label{ex:normed4}
\begin{enumerate}
\item Is $\Q(\sqrt{5})\tensor_\Q\Q(\sqrt{-5})$ a field?
\item Is $\Q(\sqrt[4]{5})\tensor_\Q\Q(\sqrt[4]{-5})\tensor_\Q\Q(\sqrt{-1})$ a field?
\end{enumerate}
\end{exercise}

\begin{exercise}\label{ex:normed5}
  Suppose $\zeta_5$ denotes a primitive $5$th root of unity.  For
  any prime $p$, consider the tensor product $\Q_p \tensor_\Q
  \Q(\zeta_5) = K_1\oplus \cdots \oplus K_{n(p)}$.  Find a simple
  formula for the number $n(p)$ of fields appearing in the
  decomposition of the tensor product $\Q_p \tensor_\Q \Q(\zeta_5)$.
  To get full credit on this problem your formula must be correct, but
  you do {\em not} have to prove that it is correct.
\end{exercise}



\begin{corollary}\label{cor:fcp}\icor{tensor products and characteristic polynomials}
  Let $a\in B$ be any element and let $f(x)\in K[x]$ be the
  characteristic polynomials of $a$ over $K$ and let $g_j(x)\in A[x]$
  (for $1\leq j \leq J$) be the characteristic polynomials of the
  images of~$a$ under $B\to A\tensor_K B \to K_j$ over $A$,
  respectively.  Then
\begin{equation}\label{eqn:fcp}
  f(x) = \prod_{j=1}^J g_j(X).
\end{equation}
\end{corollary}
\begin{proof}
  We show that both sides of (\ref{eqn:fcp}) are the characteristic
  polynomial $T(x)$ of the image of $a$ in $A\tensor_K B$ over $A$.
  That $f(x)=T(x)$ follows at once by computing the characteristic
  polynomial in terms of a basis $\ww_1,\ldots, \ww_N$ of $A\tensor_K
  B$, where $w_1,\ldots, w_N$ is a basis for $B$ over $K$ (this is
  because the matrix of left multiplication by $b$ on $A \tensor_K B$
  is exactly the same as the matrix of left multiplication on~$B$, so
  the characteristic polynomial doesn't change).  To see that $T(X) =
  \prod g_j(X)$, compute the action of the image of~$a$ in $A\tensor_K
  B$ with respect to a basis of
\begin{equation}\label{eqn:decomp}
  A\tensor_K B \isom \bigoplus_{j=1}^J K_j
\end{equation}
composed of basis of the individual extensions $K_j$ of $A$.  The
  resulting matrix will be a block direct sum of submatrices, each of
  whose characteristic polynomials is one of the $g_j(X)$.  Taking
  the product gives the claimed identity (\ref{eqn:fcp}).
\end{proof}

\begin{exercise}\label{ex:normed6}
Suppose $K$ and $L$ are number fields (i.e., finite
extensions of $\Q$).  Is it possible for the tensor
product $K\tensor_\Q L$ to contain a nilpotent element?
(A nonzero element $a$ in a ring $R$ is \defn{nilpotent} if
there exists $n>1$ such that $a^n=0$.)
\end{exercise}

\begin{corollary}\icor{completion, norms, and traces}
\icor{norms, traces, and completions}
For $a\in B$ we have
$$
 \Norm_{B/K}(a) = \prod_{j=1}^J \Norm_{K_j/A}(a),
$$
and
$$
 \Tr_{B/K}(a) = \sum_{j=1}^J \Tr_{K_j/A}(a),
$$
\end{corollary}
\begin{proof}
  This follows from Corollary~\ref{cor:fcp}.  First, the norm is $\pm$
  the constant term of the characteristic polynomial, and the constant
  term of the product of polynomials is the product of the constant
  terms (and one sees that the sign matches up correctly).  Second,
  the trace is minus the second coefficient of the characteristic
  polynomial, and second coefficients add when one multiplies
  polynomials:
  $$
  (x^n + a_{n-1}x^{n-1} + \cdots ) \cdot (x^m + a_{m-1}x^{m-1} +
  \cdots ) = x^{n+m} + x^{n+m-1} (a_{m-1} + a_{n-1}) + \cdots.
  $$
  One could also see both the statements by considering a matrix of
  left multiplication by $a$ first with respect to the basis of
  $\ww_n$ and second with respect to the basis coming from the left
  side of (\ref{eqn:decomp}).

\end{proof}
\end{ch}
%%%%%%%%%%%%%%%%%%%%%%%%%%%%%%%%%%%%%%%%%%%%%%%%%%%%%%%%%%%%%%%%%%%%%%%%%%%%%%%
%%%%%%%%%%%%%%%%%%%%%%%%%%%% END NORMED %%%%%%%%%%%%%%%%%%%%%%%%%%%%%%%%%%%%%%%
%%%%%%%%%%%%%%%%%%%%%%%%%%%%%%%%%%%%%%%%%%%%%%%%%%%%%%%%%%%%%%%%%%%%%%%%%%%%%%%


































%%%%%%%%%%%%%%%%%%%%%%%%%%%%%%%%%%%%%%%%%%%%%%%%%%%%%%%%%%%%%%%%%%%%%%%%%%%%%%%
%%%%%%%%%%%%%%%%%%%%%%%%%%%% START EXTVAL %%%%%%%%%%%%%%%%%%%%%%%%%%%%%%%%%%%%%
%%%%%%%%%%%%%%%%%%%%%%%%%%%%%%%%%%%%%%%%%%%%%%%%%%%%%%%%%%%%%%%%%%%%%%%%%%%%%%%
\begin{ch}
\chapter{Extensions and Normalizations of Valuations}
\section{Extensions of Valuations}
In this section we continue to tacitly assume that all valuations are
nontrivial.  We do not assume all our valuations satisfy the triangle inequality.


Suppose $K\subset L$ is a finite extension of fields, and that $\absspc{}$ and $\normspc{}$
are valuations on~$K$ and~$L$, respectively.
\begin{definition}[Extends]
We say that $\normspc{}$ \defn{extends}
$\absspc$ if $\abs{a} = \norm{a}$ for all $a\in K$.
\end{definition}
\begin{theorem}\label{thm:extunique}\ithm{uniqueness of valuation extension}
Suppose that $K$ is a field that is complete with respect to $\absspc$
and that~$L$ is a finite extension of~$K$ of degree $N=[L:K]$.
Then there is precisely
one extension of $\absspc{}$ to $K$, namely
\begin{equation}
  \norm{a} = \abs{\Norm_{L/K}(a)}^{1/N},
\label{eqn:normdef}\end{equation}
where the $N$th root is the non-negative real $N$th root of the
nonnegative real number $\abs{\Norm_{L/K}(a)}$.
\end{theorem}
\begin{proof}
We may assume that $\absspc$ is normalized so as
to satisfy the triangle inequality.  Otherwise, normalize
$\absspc$ so that it does, prove the theorem for the normalized
valuation $\absspc^c$, then raise both sides of (\ref{eqn:normdef})
to the power $1/c$.  In the uniqueness proof, by the same
argument we may assume that $\normspc$ also satisfies the triangle
inequality.

\vspace{1ex}\noindent{\em Uniqueness.}  View $L$ as a
finite-dimensional vector space over~$K$. Then $\normspc$ is a norm in
the sense defined earlier (Definition~\ref{defn:norm}).  Hence any two
extensions $\normspc_1$ and $\normspc_2$ of $\absspc$ are equivalent
as norms, so induce the same topology on $K$.  But as we have
seen (Proposition~\ref{prop:same_topo}), two valuations which induce the same topology are
equivalent valuations, i.e., $\normspc_1 = \normspc_2^c$, for some
positive real $c$.  Finally $c=1$ since $\norm{a}_1 = \abs{a} =
\norm{a}_2$ for all $a\in K$.

\vspace{1ex}\noindent{\em Existence.}  We do not give a proof of
existence in the general case.  Instead we give a proof, which was
suggested by Dr. Geyer at the conference out of which
\cite{cassels:global} arose. It is valid when~$K$ is locally
compact, which is the only case we will use later.

We see at once that the function defined in (\ref{eqn:normdef})
satisfies the condition (i) that $\norm{a}\geq 0$ with equality only
for $a=0$, and (ii) $\norm{ab}=\norm{a}\cdot \norm{b}$ for all $a,b\in
L$.  The difficult part of the proof is to show that there is a
constant $C>0$ such that $$\norm{a}\leq 1 \implies \norm{1+a}\leq C.$$
Note that we do not know (and will not show) that $\normspc$ as
defined by (\ref{eqn:normdef}) is a norm as in
Definition~\ref{defn:norm}, since showing that $\normspc$ is a norm
would entail showing that it satisfies the triangle inequality, which
is not obvious.

Choose a basis $b_1,\ldots, b_N$ for~$L$ over~$K$.  Let $\normspc_0$
be the max norm on $L$, so for $a=\sum_{i=1}^N c_i b_i$ with $c_i\in K$ we have
$$
\norm{a}_0 = \norm{\sum_{i=1}^N c_i b_i}_0 = \max \{\abs{c_i} : i=1,\ldots, N\}.
$$
(Note: in Cassels's original article he let $\normspc_0$ be {\em
  any} norm, but we don't because the rest of the proof does not work,
since we can't use homogeneity as he claims to do.  This is because it need not
be possible to find, for any nonzero $a\in L$ some element $c\in K$ such that
$\norm{ac}_0=1$.  This would fail, e.g., if $\norm{a}_0\neq \abs{c}$
for any $c\in K$.)
The rest of the argument is very similar to our proof from
Lemma~\ref{lem:ext_unique} of uniqueness of norms on vector spaces
over complete fields.

With respect to the $\normspc_0$-topology, $L$ has the product topology
as a product of copies of $K$.  The
function $a\mapsto \norm{a}$ is a composition of continuous functions on $L$
with respect to this topology (e.g., $\Norm_{L/K}$ is the determinant, hence
polynomial),
hence $\normspc$ defines nonzero continuous function on the compact set
$$
 S = \{a \in L : \norm{a}_0 = 1\}.
$$
By compactness, there are  real numbers $\delta,\Delta\in\R_{>0}$ such that
$$
0 < \delta \leq \norm{a} \leq \Delta \qquad\text{for all $a\in S$}.
$$
For any nonzero $a\in L$ there exists $c\in K$ such that
$\norm{a}_0 = \abs{c}$; to see this take $c$ to be a $c_i$
in the expression $a=\sum_{i=1}^N c_i b_i$ with $\abs{c_i}\geq \abs{c_j}$
for any~$j$.  Hence $\norm{a/c}_0 = 1$, so $a/c\in S$ and
$$
0 \leq \delta < \frac{\norm{a/c}}{\norm{a/c}_0} \leq \Delta.
$$
Then by homogeneity
$$
0 \leq \delta < \frac{\norm{a}}{\norm{a}_0} \leq \Delta.
$$
Suppose now that $\norm{a}\leq 1$.  Then $\norm{a}_0\leq \delta^{-1}$, so
\begin{align*}
 \norm{1+a} &\leq \Delta\cdot \norm{1+a}_0 \\
  &\leq \Delta\cdot \left( \norm{1}_0 + \norm{a}_0\right)\\
  &\leq \Delta\cdot \left( \norm{1}_0 + \delta^{-1}\right)\\
  &=C \quad\text{(say)},
\end{align*}
as required.
\end{proof}

\begin{example}
Consider the extension $\C$ of $\R$ equipped with the archimedean valuation.
The unique extension is the ordinary absolute value on $\C$:
$$\norm{x+iy} = \left(x^2 + y^2\right)^{1/2}.$$
\end{example}

\begin{example}
Consider the extension $\Q_2(\sqrt{2})$ of $\Q_2$
equipped with the $2$-adic absolute value.
Since $x^2-2$ is irreducible over $\Q_2$ we can do
some computations by working in the subfield $\Q(\sqrt{2})$
of $\Q_2(\sqrt{2})$.
\begin{lstlisting}
sage: K.<a> = NumberField(x^2 - 2); K
Number Field in a with defining polynomial x^2 - 2
sage: norm = lambda z: math.sqrt(2^(-z.norm().valuation(2)))
sage: norm(1 + a)
1.0
sage: norm(1 + a + 1)
0.70710678118654757
sage: z = 3 + 2*a
sage: norm(z)
1.0
sage: norm(z + 1)
0.35355339059327379
\end{lstlisting}

%\begin{verbatim}
%> K<a> := NumberField(x^2-2);
%> K;
%Number Field with defining polynomial x^2 - 2 over the Rational Field
%> function norm(x) return Sqrt(2^(-Valuation(Norm(x),2))); end function;
%> norm(1+a);
%1.0000000000000000000000000000
%> norm(1+a+1);
%0.70710678118654752440084436209
%> z := 3+2*a;
%> norm(z);
%1.0000000000000000000000000000
%> norm(z+1);
%0.353553390593273762200422181049
%\end{verbatim}
\end{example}

\begin{remark}
  Geyer's existence proof gives (\ref{eqn:normdef}).  But it is
  perhaps worth noting that in any case (\ref{eqn:normdef}) is a
  consequence of unique existence, as follows.  Suppose $L/K$ is as
  above.  Suppose $M$ is a finite Galois extension of $K$ that
  contains~$L$.  Then by assumption there is a unique extension of
  $\absspc{}$ to $M$, which we shall also denote by $\normspc$.  If
  $\sigma\in\Gal(M/K)$, then
$$\norm{a}_\sigma := \norm{\sigma(a)}
$$
is also an extension of $\absspc{}$ to $M$, so $\normspc_\sigma = \normspc$,
i.e.,
$$
  \norm{\sigma(a)} = \norm{a}\qquad\text{for all $a\in M$}.
$$
But now
$$
\Norm_{L/K}(a) = \sigma_1(a) \cdot \sigma_2(a) \cdots \sigma_N(a)
$$
for $a\in L$, where $\sigma_1,\ldots, \sigma_N\in \Gal(M/K)$ extend the embeddings
of $L$ into $M$.
Hence
\begin{align*}
 \abs{\Norm_{L/K}(a)} &= \norm{\Norm_{L/K}(a)} \\
     &= \prod_{1\leq n \leq N} \norm{\sigma_n(a)}\\
     &= \norm{a}^N,
\end{align*}
as required.
\end{remark}

\begin{corollary}
Let $w_1,\ldots, w_N$ be a basis for $L$ over $K$.  Then
there are positive constants $c_1$ and $c_2$ such that
$$
   c_1 \leq \frac{\norm{\ds\sum_{n=1}^{N} b_n w_n}}{\max \{ \abs{b_n} : n = 1,\ldots, N\}} \leq c_2
$$
for any $b_1,\ldots, b_N\in K$ not all $0$.
\end{corollary}
\begin{proof}
  For $\norm{\sum_{n=1}^N b_n w_n}$ and $\max{\abs{b_n}}$ are two norms
  on $L$ considered as a vector space over $K$.

I don't believe this
  proof, which I copied from Cassels's article.  My problem with it is
  that the proof of Theorem~\ref{thm:extunique} does not give that
  $C\leq 2$, i.e., that the triangle inequality holds for $\normspc$.  By
changing the basis for $L/K$ one can make any nonzero vector $a\in L$
have $\norm{a}_0=1$, so if we choose $a$ such that $\abs{a}$ is very large,
then the $\Delta$ in the proof will also be very large.  One way to fix
the corollary is to only claim that there are positive
constants $c_1, c_2,c_3, c_4$
such that
$$
   c_1 \leq \frac{\norm{\ds\sum_{n=1}^{N} b_n w_n}^{c_3}}{\max \{ \abs{b_n}^{c_4} : n = 1,\ldots, N\}} \leq c_2.
$$
Then choose $c_3, c_4$ such that $\normspc^{c_3}$ and $\absspc^{c_4}$ satisfies the triangle inequality, and prove the modified corollary
using the proof suggested by Cassels.
\end{proof}

\begin{corollary}\label{cor:iscomp}\icor{extension of complete field is complete}
A finite extension of a completely valued field $K$ is complete
with respect to the extended valuation.
\end{corollary}
\begin{proof}
By the proceeding corollary it has the topology of a finite-dimensional
vector space over $K$. (The problem with the proof of the previous
corollary is not an issue, because we can replace the extended valuation
by an equivalent one that satisfies the triangle inequality and
induces the same topology.)
\end{proof}

When $K$ is no longer complete under $\absspc$ the position is more complicated:
\begin{theorem}\label{thm:extensions}\ithm{valuation extensions}
Let $L$ be a separable extension of $K$ of finite degree
  $N=[L:K]$.  Then there are at most $N$ extensions of a valuation
  $\absspc$ on $K$ to $L$, say $\normspc_j$, for $1\leq j \leq J$.
  Let $K_v$ be the completion of $K$ with respect to $\absspc$, and for
  each~$j$ let $L_j$ be the completion of $L$ with respect to
  $\normspc_j$.  Then
\begin{equation}\label{eqn:tenslocal}
  K_v \tensor_K L \isom \bigoplus_{1\leq j\leq J} L_j
\end{equation}
algebraically and topologically, where the right hand side is given
the product topology.
\end{theorem}
\begin{proof}
  We already know (Lemma~\ref{lem:tensor_prod}) that $K_v\tensor_K L$
  is of the shape (\ref{eqn:tenslocal}), where the $L_j$ are finite
  extensions of $K_v$.  Hence there is a unique extension
  $\absspc_j^*$ of $\absspc$ to the $L_j$, and by
  Corollary~\ref{cor:iscomp} the $L_j$ are complete with respect to
  the extended valuation.  Further,  the
  ring homomorphisms
  $$
  \lambda_j : L \to K_v\tensor_K L \to L_j
  $$
  are injections.   Hence we get an extension $\normspc_j$ of $\absspc$ to $L_j$ by putting
$$
\norm{b}_j = \abs{\lambda_j(b)}_j^*.
$$
Further, $L\isom \lambda_j(L)$ is dense in $L_j$ with respect to $\normspc_j$ because
$L = K\tensor_K L$ is dense in $K_v\tensor_K L$ (since $K$ is dense
in $K_v$).  Hence $L_j$ is exactly the completion of $L$.

It remains to show that the $\normspc_j$ are distinct and that they
are the only extensions of $\absspc{}$ to~$L$.

Suppose $\normspc$ is any valuation of $L$ that extends $\absspc$.  Then
$\normspc$ extends by continuity to a real-valued function on $K_v\tensor_K L$,
which we also denote by $\normspc$. (We are again using that $L$ is dense
in $K_v\tensor_K L$.)  By continuity we have for all $a,b\in K_v\tensor_K L$,
$$
  \norm{ab} = \norm{a}\cdot \norm{b}
$$
and if $C$ is the constant in axiom (iii) for $L$ and $\normspc$, then
$$
 \norm{a}\leq 1 \implies \norm{1+a}\leq C.
$$
(In Cassels, he inexplicable assume that $C=1$ at this point in the proof.)

We consider the restriction of $\normspc$ to one of the $L_j$.  If $\norm{a}\neq 0$
for some $a\in L_j$, then $\norm{a} = \norm{b}\cdot \norm{ab^{-1}}$ for every
$b\neq 0$ in $L_j$ so $\norm{b} \neq 0$.  Hence either $\normspc$ is identically
$0$ on $L_j$ or it induces a valuation on $L_j$.

Further, $\normspc$ cannot induce a valuation on two of the $L_j$.  For
$$
  (a_1,0,\ldots, 0)\cdot (0,a_2,0,\ldots,0) = (0,0,0,\ldots,0),
$$
so for any $a_1\in L_1$, $a_2\in L_2$,
$$
  \norm{a_1}\cdot \norm{a_2} = 0.
$$
Hence $\normspc{}$ induces a valuation in precisely one of the $L_j$,
and it extends the given valuation $\absspc$ of $K_v$.  Hence $\normspc = \normspc_j$
for precisely one $j$.

It remains only to show that (\ref{eqn:tenslocal}) is a topological homomorphism.
For $$(b_1,\ldots, b_J)\in L_1\oplus \cdots \oplus L_J$$ put
$$
\norm{(b_1,\ldots, b_J)}_0 = \max_{1\leq j \leq J} \norm{b_j}_j.
$$
Then $\normspc_0$ is a norm on the right hand side of (\ref{eqn:tenslocal}),
considered as a vector space over $K_v$ and it induces the product topology.
On the other hand, any two norms are equivalent, since $K_v$ is complete,
so $\normspc_0$ induces the tensor product topology on the left hand side of
(\ref{eqn:tenslocal}).
\end{proof}

\begin{corollary}
Suppose $L=K(a)$, and let $f(x)\in K[x]$ be the minimal polynomial of~$a$.
Suppose that
$$
f(x) = \prod_{1\leq j\leq J} g_j(x)
$$
in $K_v[x]$, where the $g_j$ are irreducible.  Then $L_j = K_v(b_j)$, where
$b_j$ is a root of $g_j$.
\end{corollary}

\begin{exercise}\label{ex:extval1}
Let $K$ be the number field $\Q(\sqrt[5]{2})$.
\begin{enumerate}
\item In how many ways does the $2$-adic valuation $\absspc_2$ on $\Q$
extend to a valuation on $K$?
\item Let $v=\absspc$ be a valuation on $K$ that extends $\absspc_2$.
Let $K_v$ be the completion of $K$ with respect to $v$.
What is the residue class field $\F$ of $K_v$?
\end{enumerate}
\end{exercise}

\section{Extensions of Normalized Valuations}
Let $K$ be a complete field with valuation $\absspc$.
We consider the following three cases:
\begin{itemize}
\item[(1)] $\absspc$ is discrete non-archimedean and the
residue class field is finite.
\item[(2i)] The completion of $K$ with respect to $\absspc$ is $\R$.
\item[(2ii)] The completion of $K$ with respect to $\absspc$ is $\C$.
\end{itemize}
(Alternatively, these cases can be subsumed by the hypothesis that
the completion of $K$ is locally compact.)

In case (1) we defined the normalized valuation to
be the one such that if Haar measure of the ring of integers $\O$ is $1$,
then $\mu(a\O) = \abs{a}$ (see Definition~\ref{defn:normalized}).
In case (2i) we say that $\absspc$ is normalized if it is the ordinary
absolute value, and in (2ii) if it is the {\em square} of the ordinary
absolute value:
$$\abs{x+iy} = x^2 + y^2 \qquad \text{(normalized)}.$$
In every case, for every $a\in K$,  the map
$$
   a: x \mapsto a x
$$
on $K^+$ multiplies any choice of Haar measure by $\abs{a}$, and this characterizes
the normalized valuations among equivalent ones.

We have already verified the above characterization for
non-archimedean valuations, and it is clear for the ordinary absolute
value on $\R$, so it remains to verify it for $\C$.  The additive
group $\C^+$ is topologically isomorphic to $\R^+ \oplus \R^+$, so a
choice of Haar measure of $\C^+$ is the usual area measure on the
Euclidean plane.  Multiplication by $x+iy\in \C$ is the same as
rotation followed by scaling by a factor of $\sqrt{x^2+y^2}$, so if we
rescale a region by a factor of $x+iy$, the area of the region changes
by a factor of the square of $\sqrt{x^2+y^2}$. This explains why the
normalized valuation on $\C$ is the square of the usual absolute
value.  Note that the normalized valuation on $\C$ does not satisfy
the triangle inequality:
$$
\abs{1 + (1+i)} = \abs{2+i} = 2^2 + 1^2 = 5 \not\leq
 3= 1^2 + (1^2 + 1^2) =  \abs{1} + \abs{1+i}.
$$
The constant $C$ in axiom (3) of a valuation for the ordinary
absolute value on $\C$ is $2$, so the constant for the normalized
valuation $\absspc$ is $C\leq 4$:
$$
 \abs{x+iy} \leq 1 \implies \abs{x+iy+1} \leq 4.
$$
Note that $x^2 +y^2 \leq 1$ implies $$(x+1)^2 + y^2
 = x^2 + 2x + 1 + y^2 \leq 1 + 2x + 1 \leq 4$$ since
$x\leq 1$.

\begin{lemma}\ilem{extension of normalized valuation}
  Suppose~$K$ is a field that is complete with respect to a normalized
  valuation~$\absspc$ and let~$L$ be a finite extension of~$K$ of
  degree $N=[L:K]$.  Then the normalized valuation $\normspc$ on~$L$
  which is equivalent to the unique extension of $\absspc$ to~$L$ is
  given by the formula
\begin{equation}\label{eqn:normdef2}
 \norm{a} = \abs{\Norm_{L/K}(a)}\qquad \text{all } a\in L.
\end{equation}
\end{lemma}
\begin{proof}
Let $\normspc$ be the normalized valuation on $L$ that extends $\absspc$.
Our goal is to identify $\normspc$, and in particular to show
that it is given by (\ref{eqn:normdef2}).

By the preceding section there is a positive real number~$c$
such that for all $a\in L$ we have
$$\norm{a} = \abs{\Norm_{L/K}(a)}^c.$$
Thus all we have to do is prove that $c=1$.
In case 2 the only nontrivial situation is $L=\C$ and $K=\R$,
in which case $\abs{\Norm_{\C/\R}(x+iy)} = \abs{x^2+y^2}$,
which is the normalized valuation on $\C$ defined above.

One can argue in a unified way in all cases as follows.
Let $w_1,\ldots, w_N$ be a basis for $L/K$. Then the map
$$
  \vphi:L^+ \to \bigoplus_{n=1}^N K^+, \qquad
\sum a_n w_n \mapsto (a_1,\ldots, a_N)
$$
is an isomorphism between the additive group $L^+$
and the direct sum $\oplus_{n=1}^N K^+$, and
this is a homeomorphism if the right hand side
is given the product topology.  In particular, the
Haar measures on $L^+$ and on $\oplus_{n=1}^N K^+$
are the same up to a multiplicative constant in $\Q^*$.

Let $b\in K$.  Then the left-multiplication-by-$b$ map
$$
   b : \sum a_n w_n \mapsto \sum b a_n w_n
$$
on $L^+$ is the same as the map
$$
  (a_1,\ldots, a_N) \mapsto (ba_1,\ldots, ba_N)
$$
on $\oplus_{n=1}^N K^+$, so it multiplies the Haar
measure by $\abs{b}^N$, since $\absspc$ on~$K$
is assumed normalized (the measure of each factor
is multiplied by $\abs{b}$, so the measure on
the product is multiplied by $\abs{b}^N$).
Since $\normspc$ is assumed normalized, so
multiplication by~$b$ rescales by $\norm{b}$, we
have
$$
  \norm{b} = \abs{b}^N.
$$
But $b\in K$, so $\Norm_{L/K}(b) = b^N$.
Since $\absspc$ is nontrivial and for $a\in K$ we
have $$\norm{a} = \abs{a}^N = \abs{a^N} = \abs{\Norm_{L/K}(a)},$$
so we must have $c=1$ in (\ref{eqn:normdef2}), as claimed.
\end{proof}

In the case when~$K$ need not be complete with respect
to the valuation~$\absspc$ on~$K$, we have the following
theorem.
\begin{theorem}\label{thm:normprod}\ithm{product of extensions}
Suppose $\absspc$ is a (nontrivial as always) normalized valuation
of a field~$K$ and let~$L$ be a finite extension of~$K$.
Then for any $a\in L$,
$$
   \prod_{1\leq j \leq J} \norm{a}_j = \abs{\Norm_{L/K}(a)}
$$
where the $\normspc_j$ are the normalized valuations equivalent
to the extensions of~$\absspc$ to~$K$.
\end{theorem}
\begin{proof}
Let $K_v$ denote the completion of $K$ with respect to
$\absspc$.  Write
$$
  K_v\tensor_K L = \bigoplus_{1\leq j \leq J} L_j.
$$
Then Theorem~\ref{thm:normprod} asserts that
\begin{equation}\label{eqn:normprod}
 \Norm_{L/K}(a) = \prod_{1\leq j\leq J} \Norm_{L_j/K_v}(a).
\end{equation}
By Theorem~\ref{thm:extensions}, the $\normspc_j$ are exactly the
normalizations of the extensions of $\absspc$ to the $L_j$ (i.e., the
$L_j$ are in bijection with the extensions of valuations, so there are
no other valuations missed).  By Lemma~\ref{eqn:normdef}, the
normalized valuation $\normspc_j$ on $L_j$ is $\abs{a} =
\abs{\Norm_{L_J/K_v}(a)}$.  The theorem now follows by taking absolute
values of both sides of (\ref{eqn:normprod}).
\end{proof}

What's next?!  We're building up to give a new proof of finiteness of
the class group, which uses that the class group naturally has the
discrete topology and is the continuous image of a compact group.
\end{ch}
%%%%%%%%%%%%%%%%%%%%%%%%%%%%%%%%%%%%%%%%%%%%%%%%%%%%%%%%%%%%%%%%%%%%%%%%%%%%%%%
%%%%%%%%%%%%%%%%%%%%%%%%%%%% END EXTVAL %%%%%%%%%%%%%%%%%%%%%%%%%%%%%%%%%%%%%%%
%%%%%%%%%%%%%%%%%%%%%%%%%%%%%%%%%%%%%%%%%%%%%%%%%%%%%%%%%%%%%%%%%%%%%%%%%%%%%%%















































%%%%%%%%%%%%%%%%%%%%%%%%%%%%%%%%%%%%%%%%%%%%%%%%%%%%%%%%%%%%%%%%%%%%%%%%%%%%%%%
%%%%%%%%%%%%%%%%%%%%%%%%%%%% START ADELES %%%%%%%%%%%%%%%%%%%%%%%%%%%%%%%%%%%%%
%%%%%%%%%%%%%%%%%%%%%%%%%%%%%%%%%%%%%%%%%%%%%%%%%%%%%%%%%%%%%%%%%%%%%%%%%%%%%%%
\begin{ch}
\chapter{Global Fields and Adeles}
\section{Global Fields}\label{sec:global_fields}
\begin{definition}[Global Field]
  A \defn{global field} is a number field or a finite separable
  extension of $\F(t)$, where $\F$ is a finite field, and $t$ is
  transcendental over $\F$.
\end{definition}

In this chapter, we will focus attention on number fields, and leave
the function field case to the reader.

The following lemma essentially says that the denominator of an
element of a global field is only ``nontrivial'' at a finite number of
valuations.
\begin{lemma}\label{lem:absbig}\ilem{valuations such that $\abs{a}>1$}
Let $a\in K$ be a nonzero element of a global field $K$.  Then
there are only finitely many inequivalent valuations $\absspc$
of $K$ for which
$$
  \abs{a} > 1.
$$
\end{lemma}
\begin{proof}
  If $K=\Q$ or $\F(t)$ then the lemma follows by Ostrowski's
  classification of all the valuations on~$K$ (see
Theorem~\ref{thm:ostrowski}). For example,
  when $a=\frac{n}{d}\in\Q$, with $n,d\in \Z$, then the valuations
  where we could have $\abs{a}>1$ are the archimedean one, or the
  $p$-adic valuations $\absspc_p$ for which $p\mid d$.

Suppose now that $K$ is a finite extension of $\Q$, so
$a$ satisfies a monic polynomial
$$
  a^n + c_{n-1} a^{n-1} + \cdots + c_0 = 0,
$$
for some $n$ and $c_0,\ldots, c_{n-1}\in\Q$.
If $\absspc$ is a non-archimedean valuation on $K$, we have
\begin{align*}
  \abs{a}^n &= \abs{-(c_{n-1} a^{n-1} + \cdots + c_0)} \\
      &\leq \max(1,\abs{a}^{n-1})\cdot \max(\abs{c_0},\ldots,\abs{c_{n-1}}).
\end{align*}
Dividing each side by $\abs{a}^{n-1}$, we have
that
$$
   \abs{a} \leq \max(\abs{c_0},\ldots,\abs{c_{n-1}}),
$$
so in all cases we have
\begin{equation}\label{eqn:maxabs}
   \abs{a} \leq \max(1, \abs{c_0},
   \ldots,\abs{c_{n-1}})^{1/(n-1)}.
\end{equation}
We know the lemma for~$\Q$, so there are only finitely many
valuations~$\absspc$ on~$\Q$ such that the right hand side of
(\ref{eqn:maxabs}) is bigger than~$1$.  Since each valuation of~$\Q$
has finitely many extensions to~$K$, and there are only finitely many
archimedean valuations, it follows that there are only finitely many
valuations on~$K$ such that $\abs{a}>1$.
\end{proof}

Any valuation on a global field is either archimedean, or discrete
non-archimedean with finite residue class field, since this is true of
$\Q$ and $\F(t)$ and is a property preserved by extending a valuation
to a finite extension of the base field.  Hence it makes sense to talk
of normalized valuations.  Recall that the normalized $p$-adic
valuation on $\Q$ is $\abs{x}_p = p^{-\ord_p(x)}$, and if~$v$ is a
valuation on a number field~$K$ equivalent to an extension of
$\absspc_p$, then the normalization of $v$ is the composite of the
sequence of maps
$$
  K\hra K_v \xra{\Norm} \Q_p \xra{\absspc_p} \R,
$$
where $K_v$ is the completion of $K$ at $v$.

\begin{example}
Let $K=\Q(\sqrt{2})$, and let $p=2$.  Because $\sqrt{2}\not\in\Q_2$, there is
exactly one extension of $\absspc_2$ to~$K$, and
it sends $a=1/\sqrt{2}$ to
$$
  \abs{\Norm_{\Q_2(\sqrt{2})/\Q_2}(1/\sqrt{2})}^{1/2}_2 = \sqrt{2}.
$$
Thus the normalized valuation of $a$ is $2$.

There are two extensions of $\absspc_7$ to $\Q(\sqrt{2})$,
since $\Q(\sqrt{2})\tensor_\Q \Q_7 \isom \Q_7 \oplus \Q_7$,
as $x^2-2 = (x-3)(x-4)\pmod{7}$.  The image of $\sqrt{2}$
under each embedding into $\Q_7$ is a unit in $\Z_7$, so
the normalized valuation of $a=1/\sqrt{2}$ is, in both
cases, equal to $1$.  More generally, for any valuation
of $K$ of characteristic an odd prime $p$, the
normalized valuation of $a$ is $1$.

Since $K=\Q(\sqrt{2})\hra \R$ in two ways, there are exactly
two normalized archimedean valuations on $K$, and
both of their values on $a$ equal $1/\sqrt{2}$.
Notice that the product of the absolute values of $a$
with respect to all normalized valuations is
$$
   2 \cdot \frac{1}{\sqrt{2}} \cdot \frac{1}{\sqrt{2}} \cdot 1
   \cdot 1 \cdot 1 \cdots  = 1.
$$
This ``product formula'' holds in much more generality, as
we will now see.
\end{example}

\begin{theorem}[Product Formula]\label{thm:product_formula}\ithm{product formula}
Let $a\in K$ be a nonzero element of a global field~$K$.
Let $\absspc_v$ run through the normalized valuations
of $K$.  Then $\abs{a}_v=1$ for almost all $v$, and
$$
\prod_{\text{\rm all }v} \abs{a}_v = 1\qquad{\text{\rm (the product
    formula).}}
$$
\end{theorem}
We will later give a more conceptual proof of this
using Haar measure (see Remark~\ref{rem:conceptual_prod}).
\begin{proof}
By Lemma~\ref{lem:absbig}, we have $\abs{a}_v\leq 1$
for almost all~$v$.  Likewise, $1/\abs{a}_v = \abs{1/a}_v\leq 1$
for almost all~$v$, so $\abs{a}_v = 1$ for almost all~$v$.

Let $w$ run through all normalized valuations of $\Q$ (or of $\F(t)$),
and write $v\mid w$ if the restriction of $v$ to $\Q$ is equivalent to $w$.
Then by Theorem~\ref{thm:normprod},
$$
 \prod_{v} \abs{a}_v = \prod_w \left(\prod_{v\mid w} \abs{a}_v\right)
     = \prod_w \abs{\Norm_{K/\Q}(a)}_w,
$$
so it suffices to prove the theorem for $K=\Q$.

By multiplicativity of valuations, if the theorem is true for $b$ and
$c$ then it is true for the product $b c$ and quotient $b/c$ (when
$c\neq 0$). The theorem is clearly true for $-1$, which has valuation
$1$ at all valuations.  Thus to prove the theorem for $\Q$ it suffices
to prove it when $a=p$ is a prime number.  Then we have
$\abs{p}_\infty = p$, $\abs{p}_p = 1/p$, and for primes $q\neq p$ that
$\abs{p}_q = 1$.  Thus
$$\prod_v \abs{p}_v = p \cdot \frac{1}{p} \cdot 1 \cdot 1 \cdot 1 \cdots = 1,$$
as claimed.
\end{proof}
\begin{exercise}\label{ex:adeles1}
  Prove that the product formula holds for $\F(t)$ similar to the
  proof we gave in class using Ostrowski's theorem for $\Q$.  You may
  use the analogue of Ostrowski's theorem for $\F(t)$, which you had
  on the previous homework assignment~\ref{ex:valuations2}.
  (Don't give a measure-theoretic proof.)
\end{exercise}

If $v$ is a valuation on a field $K$, recall that
we let $K_v$ denote the completion of $K$ with respect to $v$. Also when
$v$ is non-archimedean, let
$$
  \O_v = \O_{K,v} = \{x \in K_v : \abs{x} \leq 1\}
$$
be the ring of integers of the completion.

\begin{definition}[Almost All]
We say a condition holds for \defn{almost all} elements
of a set if it holds for all but finitely many elements.
\end{definition}

We will use the following lemma later (see Lemma~\ref{lem:adelext}) to
prove that formation of the adeles of a global field
is compatible with base change.
\begin{lemma} \label{lem:ints_adeles}\ilem{matching integers}
Let $\omega_1,\ldots, \omega_n$ be a basis for $L/K$,
where $L$ is a finite separable extension of the global field
$K$ of degree~$n$.
Then for almost all normalized non-archimedean valuations $v$ on $K$ we
have
\begin{equation}\label{eqn:sum_int}
   \omega_1 \O_{v} \oplus \cdots \oplus \omega_n \O_{v}
      = \O_{w_1} \oplus \cdots \oplus \O_{w_g}
      \subset K_v\tensor_K L,
\end{equation}
where $w_1,\ldots, w_g$ are the extensions of $v$
to $L$.   Here we have identified $a\in L$ with
its canonical image in $K_v\tensor_K L$, and the direct
sum on the left is the sum taken inside the tensor
product (so directness means that the intersections are
trivial).
\end{lemma}
\begin{proof}
  The proof proceeds in two steps.  First we deduce easily from
  Lemma~\ref{lem:absbig} that for almost all $v$ the left hand side
of (\ref{eqn:sum_int}) is
  contained in the right hand side.  Then we use a trick involving
  discriminants to show the opposite inclusion for all but finitely
  many primes.

  Since $\O_v\subset \O_{w_i}$ for all $i$, the left hand side of
  (\ref{eqn:sum_int}) is contained in the right hand side if
  $\abs{\omega_i}_{w_j}\leq 1$ for $1\leq i\leq n$ and $1\leq j\leq
  g$.  Thus by Lemma~\ref{lem:absbig}, for all but finitely many~$v$
  the left hand side of (\ref{eqn:sum_int}) is contained in the right
  hand side.  We have just eliminated the finitely many primes
  corresponding to ``denominators'' of some $\omega_i$, and now only
  consider~$v$ such that $\omega_1,\ldots, \omega_n \in \O_{w}$ for
  all $w\mid v$.

  For any elements $a_1,\ldots, a_n \in K_v\tensor_K L$, consider the
  discriminant
  $$
  D(a_1,\ldots, a_n) = \det(\Tr(a_i a_j)) \in K_v,
  $$
  where the trace is induced from the $L/K$ trace.
  Since each $\omega_i$ is in each $\O_w$, for $w\mid v$, the
  traces $\Tr(\omega_i \omega_j)$ lie in $\O_v$, so
  $$d = D(\omega_1,\ldots, \omega_n)\in \O_v.$$
  Also note that $d\in
  K$ since each $\omega_i$ is in $L$.  Now suppose that
  $$
  \alpha = \sum_{i=1}^n a_i \omega_i \in \O_{w_1} \oplus \cdots
  \oplus \O_{w_g},
  $$
  with $a_i \in K_v$.  Then by properties of determinants for any
  $m$ with $1\leq m\leq n$, we have
  \begin{equation}\label{eqn:discsquare}
  D(\omega_1,\ldots, \omega_{m-1}, \alpha, \omega_{m+1}, \ldots, \omega_n)
    = a_m^2 D(\omega_1,\ldots, \omega_n).
  \end{equation}
  The left hand side of (\ref{eqn:discsquare}) is in $\O_v$, so the
  right hand side is as well, i.e.,
  $$
  a_m^2 \cdot d \in \O_v, \qquad\text{(for }m=1,\ldots, n\text{)},
  $$
  where $d\in K$. Since $\omega_1,\ldots, \omega_n$ are a basis for
  $L$ over $K$ and the trace pairing is nondegenerate, we have $d\neq
  0$, so by Theorem~\ref{thm:product_formula} we have $\abs{d}_v=1$
  for all but finitely many~$v$.  Then for all but finitely many~$v$
  we have that $a_m^2\in \O_v$.  For these $v$, that $a_m^2\in\O_v$
  implies $a_m\in \O_v$ since $a_m\in K_v$, i.e., $\alpha$ is in the
  left hand side of (\ref{eqn:sum_int}).
\end{proof}
\begin{example}
Let $K=\Q$ and $L=\Q(\sqrt{2})$.  Let $\omega_1 = 1/3$ and $\omega_2 = 2\sqrt{2}$.  In the first stage of the above proof we would eliminate
$\absspc_3$ because $\omega_2$ is not integral at $3$.  The discriminant
is
$$
 d = D\left(\frac{1}{3}, 2\sqrt{2}\right)
   =\det \mtwo{\frac{2}{9}}{0}{0}{16} = \frac{32}{9}.
$$
As explained in the second part of the proof, as long as $v\neq 2, 3$,
we have equality of the left and right hand sides in (\ref{eqn:sum_int}).
\end{example}


\section{Restricted Topological Products}

In this section we describe a topological tool, which we need in order
to define adeles (see Definition~\ref{def:adele}).

\begin{definition}[Restricted Topological Products]
  Let $X_\lambda$, for $\lambda\in\Lambda$, be a family of topological
  spaces, and for almost all~$\lambda$ let $Y_{\lambda}\subset
  X_{\lambda}$ be an open subset of $X_{\lambda}$.  Consider the space
  $X$ whose elements are sequences $\bx = \{x_\lambda\}_{\lambda\in
    \Lambda}$, where $x_\lambda\in X_\lambda$ for every $\lambda$, and
  $x_\lambda\in Y_{\lambda}$ for almost all~$\lambda$.  We give $X$ a
  topology by taking as a basis of open sets the sets $\prod
  U_{\lambda}$, where $U_{\lambda}\subset X_{\lambda}$ is open for all
  $\lambda$, and $U_{\lambda} = Y_{\lambda}$ for almost all $\lambda$.
  We call~$X$ with this topology the \defn{restricted topological
    product} of the $X_{\lambda}$ with respect to the $Y_{\lambda}$.
\end{definition}


\begin{corollary}\label{lem:xs}\icor{topology on adeles}
  Let $S$ be a finite subset of $\Lambda$, including at least all $\lambda$ for which $Y_\lambda$ is not defined, and let $X_S$ be the set of
  $\bx\in X$ with $x_\lambda\in Y_\lambda$ for all $\lambda\not\in S$,
  i.e.,
  $$
  X_S = \prod_{\lambda \in S} X_{\lambda} \times
  \prod_{\lambda\not\in S} Y_{\lambda} \subset X.
  $$
  Then $X_S$ is an open subset of $X$, and the topology induced on
  $X_S$ as a subset of $X$ is the same as the product topology.
\end{corollary}

The restricted topological product depends on the totality of the
$Y_{\lambda}$, but not on the individual $Y_{\lambda}$:
\begin{lemma}\ilem{restricted product}
  Let $Y_{\lambda}'\subset X_{\lambda}$ be open subsets, and suppose
  that $Y_{\lambda} = Y_{\lambda}'$ for almost all~$\lambda$.  Then
  the restricted topological product of the $X_\lambda$ with respect
  to the $Y_{\lambda}'$ is canonically isomorphic to the restricted
  topological product with respect to the $Y_{\lambda}$.
\end{lemma}

\begin{lemma}\label{lem:res_compact}\ilem{local compactness of restricted product}
  Suppose that the $X_\lambda$ are locally compact and that the
  $Y_\lambda$ are compact.  Then the restricted topological
product $X$ of the $X_\lambda$ is locally compact.
\end{lemma}
\begin{proof}
  For any finite subset $S$ of $\Lambda$, the open subset $X_S\subset
  X$ is locally compact, because by Lemma~\ref{lem:xs} it is a product
  of finitely many locally compact sets with an infinite product of
  compact sets.  (Here we are using Tychonoff's theorem from topology,
  which asserts that an arbitrary product of compact topological
  spaces is compact (see Munkres's {\em Topology, a first course},
  chapter 5).) Since $X=\cup_{S} X_S$, and the $X_S$ are open in $X$,
  the result follows.
\end{proof}

The following measure will be extremely important in deducing
topological properties of the ideles, which will be used in
proving finiteness of class groups.  See, e.g., the
proof of Lemma~\ref{lem:bignorm}, which is a key input
to the proof of strong approximation (Theorem~\ref{thm:strong}).
\begin{definition}[Product Measure]\label{defn:prodmeasure}
  For all $\lambda\in\Lambda$, suppose $\mu_\lambda$ is a measure on
  $X_\lambda$ with $\mu_\lambda(Y_\lambda) = 1$ when $Y_\lambda$ is
  defined.  We define the \defn{product measure} $\mu$ on $X$ to be
  that for which a basis of measurable sets is $$\prod_\lambda
  M_\lambda$$
  where each $M_\lambda\subset X_\lambda$ has finite
  $\mu_\lambda$-measure and
  $M_\lambda=Y_\lambda$ for almost all $\lambda$, and where
  $$
  \mu\left(\prod_\lambda M_\lambda\right) = \prod_\lambda
  \mu_\lambda(M_\lambda).
  $$
\end{definition}

\section{The Adele Ring}
Let $K$ be a global field.  For each normalized valuation $\absspc_v$ of $K$,
let $K_v$ denote the completion of $K$.  If $\absspc_v$ is
non-archimedean, let $\O_v$ denote the ring of integers of $K_v$.
\renewcommand{\AA}{\mathbb{A}}

\begin{definition}[Adele Ring]\label{def:adele}
  The \defn{adele ring} $\AA_K$ of $K$ is the topological ring whose
  underlying topological space is the restricted topological product
  of the $K_v$ with respect to the $\O_v$, and where addition and
  multiplication are defined componentwise:
\begin{equation}\label{eqn:adelearith}
(\bx \by)_v = \bx_v \by_v \qquad
(\bx + \by)_v = \bx_v + \by_v\qquad
\text{for }\bx, \by\in\AA_K.
\end{equation}
\end{definition}
It is readily verified that (i) this definition makes sense, i.e., if
$\bx, \by\in \AA_K$, then $\bx\by$ and $\bx+\by$, whose components are
given by (\ref{eqn:adelearith}), are also in $\AA_K$, and (ii) that
addition and multiplication are continuous in the $\AA_K$-topology, so
$\AA_K$ is a topological ring, as asserted.
Also,
Lemma~\ref{lem:res_compact} implies that $\AA_K$ is locally compact
because the $K_v$ are locally compact
(Corollary~\ref{cor:locally_compact}), and the $\O_v$ are
compact (Theorem~\ref{thm:compact}).

There is a natural continuous ring inclusion
\begin{equation}\label{eqn:princ_inc}
K\hra \AA_K
\end{equation}
that sends $x\in K$ to the adele every one of whose components is $x$.
This is an adele because $x\in \O_v$ for almost all $v$, by
Lemma~\ref{lem:absbig}.  The map is injective because each map $K\to
K_v$ is an inclusion.

\begin{definition}[Principal Adeles]
  The image of (\ref{eqn:princ_inc}) is the ring of \defn{principal
    adeles}.
\end{definition}
It will cause no trouble to identify $K$ with the principal adeles, so
we shall speak of~$K$ as a subring of $\AA_K$.

Formation of the adeles is compatible with base change, in the
following sense.
\begin{lemma}\label{lem:adelext}\ilem{base extension of adeles}
  Suppose $L$ is a finite (separable) extension of the global field
  $K$.  Then
\begin{equation}\label{eqn:baseext}
  \AA_K \tensor_K L \isom \AA_L
\end{equation}
both algebraically and topologically.  Under this isomorphism,
  $$L\isom K\tensor_K L \subset \AA_K \tensor_K L$$ maps isomorphically onto
  $L\subset \AA_L$.
\end{lemma}
\begin{proof}
Let $\omega_1,\ldots, \omega_n$
be a basis for $L/K$ and let $v$ run through the normalized valuations
on~$K$.  The left hand side of (\ref{eqn:baseext}), with
the tensor product topology, is the restricted product of the
tensor products
$$
  K_v \tensor_K L \isom K_v \cdot\omega_1 \oplus \cdots \oplus K_v\cdot \omega_n
$$
with respect to the integers
\begin{equation}\label{eqn:intsum}
   \O_v\cdot \omega_1 \oplus \cdots \oplus \O_v\cdot \omega_n.
 \end{equation}
 (An element of the left hand side is a finite linear combination $\sum
\bx_i \tensor a_i$ of adeles $\bx_i \in \AA_K$ and coefficients $a_i
\in L$, and there is a natural isomorphism from the ring of such formal
sums to the restricted product of the $K_v\tensor_K L$.)

We proved before (Theorem~\ref{thm:extensions}) that
 $$
  K_v \tensor_K L \isom L_{w_1} \oplus \cdots \oplus L_{w_g},
  $$
  where $w_1,\ldots, w_g$ are the normalizations of the extensions
  of $v$ to $L$.  Furthermore, as we proved using discriminants (see
  Lemma~\ref{lem:ints_adeles}), the above identification identifies
  (\ref{eqn:intsum}) with
$$
 \O_{w_1} \oplus \cdots \oplus \O_{w_g},
$$
for almost all~$v$.
Thus the left hand side of (\ref{eqn:baseext}) is the restricted
product of the $L_{w_1} \oplus \cdots \oplus L_{w_g}$
with respect to the $\O_{w_1} \oplus \cdots \oplus \O_{w_g}$.
But this is canonically isomorphic to the restricted product
of all completions $L_w$ with respect to $\O_w$, which
is the right hand side of (\ref{eqn:baseext}).  This
establishes an isomorphism between the two sides of (\ref{eqn:baseext})
as topological spaces.  The map is also a ring homomorphism, so
the two sides are algebraically isomorphic, as claimed.
\end{proof}

\begin{corollary}\label{cor:addstruct}\icor{$\AA_K^+$ and base extension}
Let $\AA_K^+$ denote the topological group obtained from the
additive structure on $\AA_K$.  Suppose $L$ is a finite seperable
extension of $K$.
 Then
$$
  \AA_L^+ = \AA_K^+ \oplus \cdots \oplus \AA_K^+,
\qquad ([L:K] \text{ summands}).
$$
In this isomorphism the additive group $L^+\subset \AA_L^+$ of the
principal adeles is mapped isomorphically onto $K^+\oplus \cdots
\oplus K^+$.
\end{corollary}
\begin{proof}
For any nonzero $\omega \in L$, the subgroup $\omega\cdot \AA_K^+$
of $\AA_L^+$ is isomorphic as a topological group to $\AA_K^+$
(the isomorphism is multiplication by $1/\omega$).  By
Lemma~\ref{lem:adelext}, we have isomorphisms
$$
\AA_L^+ = \AA_K^+ \tensor_K L
   \isom \omega_1\cdot \AA_K^+ \oplus \cdots \oplus \omega_n \cdot \AA_K^+
   \isom \AA_K^+ \oplus \cdots \oplus \AA_K^+.
$$
If $a \in L$, write $a=\sum b_i \omega_i$, with $b_i \in K$.
Then $a$ maps via the above map to
$$x = (\omega_1\cdot \{b_1\},\ldots, \omega_n \cdot \{b_n\}),$$
where $\{b_i\}$ denotes the principal adele defined by $b_i$.
Under the final map, $x$ maps to the tuple
$$(b_1,\ldots, b_n) \in K\oplus \cdots \oplus K \subset
\AA_K^+ \oplus \cdots \oplus \AA_K^+.$$
The dimensions of $L$ and of $K\oplus \cdots \oplus K$ over
$K$ are the same, so
this proves the final claim of the corollary.
\end{proof}

\begin{theorem}\label{thm:adelequo}\ithm{compact quotient of adeles}
  The global field $K$ is discrete in $\AA_K$ and the quotient
  $\AA_K^+/K^+$ of additive groups is compact in the quotient
  topology.
\end{theorem}
At this point Cassels remarks
\begin{quote}``It is impossible to conceive of any other uniquely
defined topology on $K$.  This metamathematical reason is more
persuasive than the argument that follows!''
\end{quote}
\begin{proof}
Corollary~\ref{cor:addstruct}, with $K$ for $L$ and $\Q$ or
$\F(t)$ for $K$, shows that it is enough to verify
the theorem for $\Q$ or $\F(t)$, and we shall do it
here for $\Q$.

To show that $\Q^+$ is discrete in $\AA_\Q^+$ it is enough, because of
the group structure, to find an open set $U$ that contains $0 \in
\AA_\Q^+$, but which contains no other elements of $\Q^+$.  (If
$\alpha\in\Q^+$, then $U+\alpha$ is an open subset of $\AA_\Q^+$
whose intersection with $\Q^+$ is $\{\alpha\}$.)
We take for $U$ the set of $\bx=\{x_v\}_v \in \AA_\Q^+$ with
$$
 \abs{x_\infty}_\infty < 1
\qquad\text{and}\qquad \abs{x_p}_p \leq 1 \quad \text{(all $p$)},
$$
where $\absspc_p$ and $\absspc_\infty$ are respectively the
$p$-adic and the usual archimedean absolute values on~$\Q$.
If $b\in \Q\cap U$, then in the first place $b\in \Z$
because $\abs{b}_p\leq 1$ for all $p$, and then $b=0$
because $\abs{b}_\infty<1$.  This proves that $K^+$
is discrete in $\AA_\Q^+$.  (If we leave out one valuation,
as we will see later (Theorem~\ref{thm:strong}), this theorem is
false---what goes wrong with the proof just given?)

Next we prove that the quotient $\AA_\Q^+/\Q^+$ is compact.
Let $W\subset \AA_\Q^+$ consist of the $\bx=\{x_v\}_v\in \AA_\Q^+$
with
$$
  \abs{x_\infty}_\infty \leq \frac{1}{2}\qquad\text{and}\qquad
   \abs{x_p}_p \leq 1\qquad\text{for all primes $p$}.
$$
We show that every adele $\by=\{y_v\}_v$ is of the form
$$
  \by = a + \bx, \qquad a\in \Q, \quad \bx\in W,
$$
which will imply that the compact set $W$ maps surjectively
onto $\AA_\Q^+ / \Q^+$.
Fix an adele $\by=\{y_v\}\in\AA_\Q^+$.  Since $\by$
is an adele, for each prime $p$ we can find
a rational number
$$
  r_p = \frac{z_p}{p^{n_p}}
\qquad \text{with} \quad z_p \in \Z \quad \text{and} \quad n_p \in \Z_{\geq 0}
$$
such that
$$
  \abs{y_p - r_p}_p \leq 1,
$$
and
$$
  r_p = 0 \qquad \text{almost all $p$}.
$$
More precisely, for the finitely
many $p$ such that $$y_p = \sum_{n\geq -\abs{s}} a_np^n \not\in\Z_p,$$ choose
$r_p$ to be a rational number that is the value of an appropriate truncation
of the $p$-adic expansion of $y_p$, and
when $y_p\in \Z_p$ just choose $r_p = 0$.
Hence $r=\sum_{p} r_p\in\Q$ is well defined.
The $r_q$ for $q\neq p$ do not mess up the inequality
$\abs{y_p - r}_p \leq 1$ since the
valuation $\absspc_p$ is non-archimedean and the $r_q$ do not have any $p$ in
their denominator:
$$\abs{y_p  - r}_p
   = \abs{y_p - r_p - \sum_{q\neq p} r_q}_p
   \leq \max\left(\abs{y_p - r_p}_p, \abs{\sum_{q\neq p} r_q}_p\right)
   \leq \max(1,1) = 1.
$$
Now choose $s\in\Z$ such that
$$
  \abs{y_\infty - r-s} \leq \frac{1}{2}.
$$
Then $a=r+s$ and $\bx = \by - a$ do what is required,
since $\by - a = \by - r - s$ has the desired property
(since $s\in\Z$ and the $p$-adic valuations are
non-archimedean).

Hence the continuous map $W\to \AA_\Q^+/\Q^+$ induced by the quotient
map $\AA_\Q^+ \to \AA_\Q^+/\Q^+$ is surjective.  But $W$ is compact
(being the topological product of the compact spaces
$\abs{x_\infty}_\infty\leq 1/2$ and the $\Z_p$ for all $p$), hence
$\AA_\Q^+/\Q^+$ is also compact.
\end{proof}

\begin{exercise}\label{ex:adeles2}
  Prove Theorem~\ref{thm:adelequo}, that ``The global field $K$
  is discrete in $\AA_K$ and the quotient $\AA_K^+/K^+$ of additive
  groups is compact in the quotient topology.'' in the case when $K$
  is a finite extension of $\F(t)$, where $\F$ is a finite field.
\end{exercise}


\begin{corollary}\label{cor:subsetW}\icor{compact subset of adeles}
There is a subset $W$ of $\AA_K$ defined by inequalities of the
type $\abs{x_v}_v \leq \delta_v$, where $\delta_v=1$
for almost all $v$, such that every $\by\in\AA_K$ can
be put in the form
$$
  \by = a + \bx, \qquad a\in K, \quad \bx \in W,
$$
i.e., $\AA_K = K + W$.
\end{corollary}
\begin{proof}
  We constructed such a set for $K=\Q$ when proving
  Theorem~\ref{thm:adelequo}.  For general~$K$ the $W$ coming from the
  proof determines compenent-wise a subset of $\AA_K^+\isom
  \AA_\Q^+\oplus \cdots \oplus \AA_\Q^+$ that is a subset of a set
  with the properties claimed by the corollary.
\end{proof}

As already remarked, $\AA_K^+$ is a locally compact group, so it has
an invariant Haar measure.  In fact one choice of this Haar measure is
the product of the Haar measures on the $K_v$, in the sense
of Definition~\ref{defn:prodmeasure}.

\begin{corollary}\label{cor:finitemeasure}\icor{$\AA_K^+/K^+$ has finite measure}
The quotient $\AA_K^+/K^+$ has finite measure in the quotient measure
induced by the Haar measure on $\AA_K^+$.
\end{corollary}
\begin{remark}
This statement is independent of the particular choice
of the multiplicative constant in the Haar measure
on $\AA_K^+$.  We do not here go into the question of
finding the measure $\AA_K^+/K^+$ in terms of our
explicitly given Haar measure.  (See Tate's thesis,
\cite[Chapter XV]{cassels-frohlich}.)
\end{remark}
\begin{proof}
This can be reduced similarly to the case of $\Q$
or $\F(t)$ which is immediate, e.g., the $W$ defined
above has measure $1$ for our Haar measure.

Alternatively, finite measure follows from compactness.  To see
this,  cover
$\AA_K/K^+$ with the translates of $U$, where $U$ is a nonempty open
set with finite measure.  The existence of a finite subcover implies
finite measure.
\end{proof}

\begin{remark}\label{rem:conceptual_prod}
  We give an alternative proof of the product formula
  $\prod\abs{a}_v=1$ for nonzero $a\in K$.  We have seen that if
  $x_v\in K_v$, then multiplication by $x_v$ magnifies the Haar
  measure in $K_v^+$ by a factor of $\abs{x_v}_v$.  Hence if
  $\bx=\{x_v\}\in\AA_K$, then multiplication by $\bx$ magnifies the
  Haar measure in $\AA_K^+$ by $\prod \abs{x_v}_v$.  But now
  multiplication by $a\in K$ takes $K^+\subset \AA_K^+$ into $K^+$, so
  gives a well-defined bijection of $\AA_K^+/K^+$ onto $\AA_K^+/K^+$
  which magnifies the measure by the factor $\prod\abs{a}_v$.  Hence
  $\prod\abs{a}_v=1$ by Corollary~\ref{cor:finitemeasure}.  (The point is
  that if $\mu$ is the measure of $\AA_K^+/K^+$, then $\mu =
  \prod\abs{a}_v \cdot \mu$, so because $\mu$ is finite we must have
  $\prod\abs{a}_v = 1$.)
\end{remark}


\section{Strong Approximation}
We first prove a technical lemma and corollary, then use them to
deduce the strong approximation theorem, which is an extreme
generalization of the Chinese Remainder Theorem; it asserts that $K^+$
is dense in the analogue of the adeles with one valuation removed.

The proof of Lemma~\ref{lem:bignorm} below will use in a crucial way
the normalized Haar measure on $\AA_K$ and the induced measure on the
compact quotient $\AA_K^+/K^+$.  Since I am not formally developing
Haar measure on locally compact groups, and since I didn't explain
induced measures on quotients well in the last chapter, hopefully the
following discussion will help clarify what is going on.

The real numbers $\R^+$ under addition is a locally compact
topological group.  Normalized Haar measure $\mu$ has the property
that $\mu([a,b]) = b-a$, where $a\leq b$ are real numbers and
$[a,b]$ is the closed interval from $a$ to $b$.  The subset
$\Z^+$ of $\R^+$ is discrete, and the quotient $S^1 = \R^+/\Z^+$
is a compact topological group, which thus
has a Haar measure.  Let $\overline{\mu}$ be the Haar measure
on $S^1$ normalized so that  the natural quotient $\pi:\R^+\to S^1$
preserves the measure, in the sense that if $X\subset \R^+$
is a measurable set that maps injectively into $S^1$, then
$\mu(X) = \overline{\mu}(\pi(X))$.  This determine
$\overline{\mu}$ and we have $\overline{\mu}(S^1)=1$ since
$X=[0,1)$ is a measurable set that maps bijectively onto
$S^1$ and has measure~$1$.  The situation for the map
$\AA_K \to \AA_K/K^+$ is pretty much the same.


\begin{lemma}\label{lem:bignorm}
  There is a constant $C>0$ that depends only on the global field $K$
  with the following property:

Whenever $\bx=\{x_v\}_v \in \AA_K$ is such that
\begin{equation}\label{eqn:bignorm}
\prod_v \abs{x_v}_v > C,
\end{equation}
then there is a nonzero principal adele $a \in K\subset \AA_K$ such
that
$$
  \abs{a}_v \leq \abs{x_v}_v \qquad\text{\rm for all $v$}.
$$
\end{lemma}
\begin{proof}
This proof is modelled on Blichfeldt's proof of Minkowski's
Theorem in the Geometry of Numbers, and works in quite general
circumstances.

First we show that (\ref{eqn:bignorm}) implies that $\abs{x_v}_v=1$
for almost all $v$.  Because $\bx$ is an adele, we have
$\abs{x_v}_v\leq 1$ for almost all $v$.  If $\abs{x_v}_v<1$ for
infinitely many $v$, then the product in (\ref{eqn:bignorm}) would
have to be $0$.  (We prove this only when $K$ is a finite extension of
$\Q$.)  Excluding archimedean valuations, this is because the
normalized valuation $\abs{x_v}_v = \abs{\Norm(x_v)}_p$, which if less
than $1$ is necessarily $\leq 1/p$. Any infinite product of numbers
$1/p_i$ must be $0$, whenever $p_i$ is a sequence of primes.

Let $c_0$ be the Haar measure of $\AA_K^+/K^+$ induced from normalized
Haar measure on $\AA_K^+$, and let $c_1$ be the Haar measure of the set
of $\by=\{y_v\}_v \in \AA_K^+$ that satisfy
\begin{align*}
  \abs{y_v}_v &\leq \frac{1}{2} \qquad \text{if $v$ is real archimedean},\\
  \abs{y_v}_v &\leq \frac{1}{2} \qquad \text{if $v$ is complex archimedean},\\
  \abs{y_v}_v &\leq 1 \,\qquad \text{if $v$ is non-archimedean}.
\end{align*}
(As we will see, any positive real number $\leq 1/2$ would suffice in
the definition of $c_1$ above.  For example, in Cassels's article he
uses the mysterious $1/10$.  He also doesn't discuss the subtleties
of the complex archimedean case separately.)

Then $0<c_0<\infty$ since $\AA_K/K^+$ is compact, and $0<c_1<\infty$
because the number of archimedean valuations $v$ is finite.  We show
that $$C=\frac{c_0}{c_1}$$ will do.  Thus suppose $\bx$ is as in
(\ref{eqn:bignorm}).

The set $T$ of $\bt=\{t_v\}_v\in \AA_K^+$ such that
\begin{align*}
  \abs{t_v}_v &\leq \frac{1}{2}\abs{x_v}_v \,\qquad\text{if $v$ is real archimedean},\\
  \abs{t_v}_v &\leq \frac{1}{2}\sqrt{\abs{x_v}_v} \,\qquad\text{if $v$ is complex archimedean},\\
  \abs{t_v}_v &\leq \abs{x_v}_v \quad\,\qquad \text{if $v$ is non-archimedean}
\end{align*}
has measure
\begin{equation}\label{eqn:tbigger}
 c_1 \cdot \prod_{v} \abs{x_v}_v > c_1 \cdot C = c_0.
\end{equation}
(Note:  If there are complex valuations, then the some of
the $\abs{x_v}_v$'s in the product must be squared.)

%% (Note: The reason for the square root for the complex archimedean valuations
%% is that the normalized Haar measure on~$\C$ is the usual
%% Lebesgue measure on~$\C$, and the disc in $\C$ of radius
%% $(m_v/2)\cdot \sqrt{\abs{x_v}_v}$
%% has measure $(m_v\sqrt{\abs{x_v}_v})^2$ times the
%% measure ($=\pi/4$) of the disk of radius $1/2$ in $\C$.)

Because of (\ref{eqn:tbigger}), in
the quotient map $\AA_K^+ \to \AA_K^+/K^+$
there
must be a pair of distinct points of $T$ that have
the same image in $\AA_K^+/K^+$, say
$$
   \bt' = \{t'_v\}_v \in T\quad\text{and}\quad \bt'' = \{t''_v\}_v\in T
$$
and
$$
  a = \bt' - \bt'' \in K^+
$$
is nonzero.
Then
$$  \abs{a}_v = \abs{t'_v - t''_v}_v
         \leq
\begin{cases}
\abs{t'_v} + \abs{t''_v} \leq 2\cdot \frac{1}{2}\abs{x_v}_v \leq \abs{x_v}_v &
    \text{if $v$ is real archimedean, or}\\
\max(\abs{t'_v},\abs{t''_v}) \leq \abs{x_v}_v  &
    \text{if $v$ is non-archimedean,}
\end{cases}
$$
for all $v$.  In the case of complex archimedean $v$, we must be
careful because the normalized valuation $\absspc_v$ is the {\em
  square} of the usual archimedean complex valuation $\absspc_\infty$
on $\C$, so e.g., it does not satisfy the triangle inequality.
In particular, the quantity $\abs{t_v'-t''_v}_v$ is at most
the square of the maximum distance between two points in the disc in
$\C$ of radius $\frac{1}{2}\sqrt{\abs{x_v}_v}$, where by distance we
mean the usual distance.  This maximum distance in such a disc
is at most $\sqrt{\abs{x_v}_v}$, so $\abs{t_v'-t''_v}_v$ is at most
$\abs{x_v}_v$, as required.  Thus $a$ satisfies the requirements of
the lemma.
\end{proof}

\begin{corollary}\label{cor:small_a}
Let $v_0$ be a normalized valuation and let $\delta_v>0$ be given
for all $v\neq v_0$ with $\delta_v = 1$ for almost all $v$.  Then
there is a nonzero $a\in K$ with
$$
  \abs{a}_v \leq \delta_v\qquad\text{(all $v\neq v_0$)}.
$$
\end{corollary}
\begin{proof}
This is just a degenerate case of Lemma~\ref{lem:bignorm}.
Choose $x_v\in K_v$ with $0< \abs{x_v}_v \leq \delta_v$
and $\abs{x_v}_v=1$ if $\delta_v=1$.  We can then choose
$x_{v_0}\in K_{v_0}$ so that
$$
\prod_{\text{all $v$ including $v_0$}} \abs{x_v}_v > C.
$$
Then Lemma~\ref{lem:bignorm} does what is required.
\end{proof}

\begin{remark}
  The character group of the locally compact group $\AA_K^+$ is
  isomorphic to $\AA_K^+$ and $K^+$ plays a special role.  See Chapter
  XV of \cite{cassels-frohlich}, Lang's \cite{lang:algebraic_numbers},
  Weil's \cite{weil:adeles}, and Godement's Bourbaki seminars 171 and
  176.  This duality lies behind the functional equation of $\zeta$
  and $L$-functions.  Iwasawa has shown \cite{iwasawa:adele} that the
  rings of adeles are characterized by certain general
  topologico-algebraic properties.
\end{remark}

We proved before that $K$ is discrete in $\AA_K$.  If one valuation is
removed, the situation is much different.
\begin{theorem}[Strong Approximation]\label{thm:strong}\ithm{strong approximation}
  Let $v_0$ be any normalized nontrivial valuation of the global field~$K$.
  Let $\AA_{K,v_0}$ be the restricted topological product of the
  $K_v$ with respect to the $\O_v$, where $v$ runs through all
  normalized valuations $v\neq v_0$.  Then~$K$ is dense in
  $\AA_{K,v_0}$.
\end{theorem}
\begin{proof}
This proof was suggested by Prof. Kneser at the Cassels-Frohlich
conference.

Recall that if $\bx =\{x_v\}_v\in \AA_{K,v_0}$ then a basis of open
sets about $\bx$ is the collection of products
$$\prod_{v\in S} B(x_v,\eps_v) \times \prod_{v\not\in S,\,\, v\neq v_0} \O_v,$$
where $B(x_v,\eps_v)$ is an open ball in $K_v$ about $x_v$, and
$S$ runs through finite sets of normalized valuations (not including
$v_0$).  Thus
denseness of $K$ in $\AA_{K,v_0}$ is equivalent to the following
statement about elements.  Suppose we are given (i) a finite set $S$
of valuations $v\neq v_0$, (ii) elements $x_v\in K_v$ for all $v\in
S$, and (iii) an $\eps>0$.  Then there is an element $b\in K$ such that
$\abs{b-x_v}_v<\eps$ for all $v\in S$ and $\abs{b}_v\leq 1$ for all
$v\not\in S$ with $v\neq v_0$.

By the corollary to our proof that $\AA_K^+/K^+$ is compact
(Corollary~\ref{cor:subsetW}), there is a $W\subset \AA_K$ that is
defined by inequalities of the form $\abs{y_v}_v\leq \delta_v$ (where
$\delta_v=1$ for almost all $v$) such that ever $\mathbf{z}\in \AA_K$
is of the form
\begin{equation}\label{eqn:wsum}
  \mathbf{z} = \by + c, \qquad \by\in W, \quad c\in K.
\end{equation}
By Corollary~\ref{cor:small_a}, there is a nonzero $a\in K$ such
that
\begin{align*}
  \abs{a}_v &< \frac{1}{\delta_v}\cdot \eps\qquad \text{ for }v\in S,\\
  \abs{a}_v &\leq \frac{1}{\delta_v} \qquad\,\quad \text{ for } v\not\in S,\, v\neq v_0.
\end{align*}
Hence on putting $\mathbf{z} = \frac{1}{a}\cdot \bx$
in (\ref{eqn:wsum}) and multiplying by $a$, we see that
every $\bx\in \AA_K$ is of the shape
$$
  \bx = \mathbf{w} + b,\qquad \mathbf{w}\in a\cdot W, \quad b\in K,
$$
where $a\cdot W$ is the set of $a\by$ for $\by\in W$.
If now we let $\bx$ have components the given $x_v$ at $v\in S$,
and (say) $0$ elsewhere, then $b=\bx-\bw$ has the properties required.
\end{proof}

\begin{remark}
The proof gives a quantitative form of the theorem (i.e.,
with a bound for $\abs{b}_{v_0}$).  For an alternative approach,
see \cite{mahler:inequalities}.
\end{remark}

In the next chapter we'll introduce the ideles $\AA_K^*$.  Finally,
we'll relate ideles to ideals, and use everything so far to give a new
interpretation of class groups and their finiteness.
\end{ch}
%%%%%%%%%%%%%%%%%%%%%%%%%%%%%%%%%%%%%%%%%%%%%%%%%%%%%%%%%%%%%%%%%%%%%%%%%%%%%%%
%%%%%%%%%%%%%%%%%%%%%%%%%%%% END ADELES %%%%%%%%%%%%%%%%%%%%%%%%%%%%%%%%%%%%%%%
%%%%%%%%%%%%%%%%%%%%%%%%%%%%%%%%%%%%%%%%%%%%%%%%%%%%%%%%%%%%%%%%%%%%%%%%%%%%%%%











































%%%%%%%%%%%%%%%%%%%%%%%%%%%%%%%%%%%%%%%%%%%%%%%%%%%%%%%%%%%%%%%%%%%%%%%%%%%%%%%
%%%%%%%%%%%%%%%%%%%%%%%%%%%% START IDELES %%%%%%%%%%%%%%%%%%%%%%%%%%%%%%%%%%%%%
%%%%%%%%%%%%%%%%%%%%%%%%%%%%%%%%%%%%%%%%%%%%%%%%%%%%%%%%%%%%%%%%%%%%%%%%%%%%%%%
\begin{ch}
\chapter{Ideles and Ideals}
In this chapter, we introduce the ideles $\II_K$, and relate ideles to
ideals, and use what we've done so far to give an alternative
interpretation of class groups and their finiteness, thus linking the
adelic point of view with the classical point of view of the first
part of this course.



\section{The Idele Group}
The invertible elements of any commutative
topological ring~$R$ are a group $R^*$ under multiplication.
In general $R^*$ is not a topological group if it is
endowed with the subset topology because inversion need
not be continuous (only multiplication and addition on
$R$ are required to be continuous).  It is usual therefore
to give $R^*$ the following topology.
There is an injection
\begin{equation}\label{eqn:prod_embed}
  x\mapsto \left( x, \,\,\,\frac{1}{x} \right)
\end{equation}
of $R^*$ into the topological product $R\times R$.  We give $R^*$ the
corresponding subset topology.  Then $R^*$ with this topology is a
topological group and the inclusion map $R^*\hra R$ is continous.  To
see continuity of inclusion, note that this topology is finer (has at
least as many open sets) than the subset topology induced by
$R^*\subset R$, since the projection maps $R\times R\to R$ are
continuous.

\begin{example}\label{ex:cont}
This is a ``non-example''. The inverse map on $\Z_p^*$ is continuous with
respect to the $p$-adic topology.  If $a,b\in \Z_p^*$,
then $\abs{a}=\abs{b}=1$, so if $\abs{a-b}<\eps$, then
$$
  \abs{\frac{1}{a} - \frac{1}{b}}
   = \abs{\frac{b-a}{ab}} = \frac{\abs{b-a}}{\abs{ab}} < \frac{\eps}{1}=\eps.
$$
\end{example}

\begin{definition}[Idele Group]
  The \defn{idele group} $\II_K$ of $K$ is the group $\AA_K^*$ of invertible
  elements of the adele ring $\AA_K$.
\end{definition}
We shall usually speak of $\II_K$ as a subset of $\AA_K$, and will
have to distinguish between the $\II_K$ and $\AA_K$-topologies.
\begin{example}
For a rational prime $p$, let $\bx_p\in \AA_\Q$ be the adele whose $p$th
component is $p$ and whose $v$th component, for $v\neq p$, is $1$.
Then $\bx_p \to 1$ as $p\to\infty$ in $\AA_\Q$, for the following reason.
We must show that if $U$ is a basic open set that contains the
adele $1=\{1\}_v$, the $\bx_p$ for all sufficiently large $p$
are contained in $U$.  Since $U$ contains $1$ and is a basic
open set, it is of the form
$$\prod_{v\in S} U_v \times \prod_{v\not\in S} \Z_v,$$
where $S$ is a finite set, and the $U_v$, for $v\in S$, are
arbitrary open subsets of $\Q_v$ that contain $1$.
If $q$ is a prime larger than any prime in $S$, then
$\bx_p$ for $p\geq q$, is in $U$.   This proves
convergence.
If the inverse map were continuous on $\II_K$, then
the sequence of $\bx_p^{-1}$ would converge to $1^{-1}=1$.
However, if $U$ is an open set as above about $1$, then
for sufficiently large $p$, {\em none} of the adeles $\bx_p$ are
contained in $U$.
\end{example}


\begin{lemma}\label{lem:idelesprod}\ilem{ideles are a restricted product}
The group of ideles $\II_K$ is the restricted topological project
of the $K_v^*$ with respect to the units $U_v=\O_v^*\subset K_v$,
with the restricted product topology.
\end{lemma}
We omit the proof of Lemma~\ref{lem:idelesprod}, which is a
matter of thinking carefully about the definitions.  The main
point is that inversion is continuous on $\O_v^*$ for each~$v$.
(See Example~\ref{ex:cont}.)


We have seen that $K$ is naturally embedded in $\AA_K$, so
$K^*$ is naturally embedded in~$\II_K$.
\begin{definition}[Principal Ideles]
  We call $K^*$, considered as a subgroup of $\II_K$, the
\defn{principal ideles}.
\end{definition}

\begin{lemma}\ilem{principal ideles are discrete}
The principal ideles $K^*$ are discrete as a subgroup of $\II_K$.
\end{lemma}
\begin{proof}
  For $K$ is discrete in $\AA_K$, so $K^*$ is embedded in $\AA_K\times
  \AA_K$ by (\ref{eqn:prod_embed}) as a discrete subset.
  (Alternatively, the subgroup topology on $\II_K$ is finer than the
  topology coming from $\II_K$ being a subset of $\AA_K$, and $K$ is
  already discrete in $\AA_K$.)
\end{proof}

\begin{definition}[Content of an Idele]
The \defn{content} of $\bx=\{x_v\}_v \in \II_K$ is
$$
  c(\bx) = \prod_{\text{all }v} \abs{x_v}_v \in \R_{>0}.
$$
\end{definition}

\begin{lemma}\ilem{content map is continuous}
The map $\bx\to c(\bx)$ is a continuous  homomorphism of
the topological group $\II_K$ into $\R_{>0}$, where
we view $\R_{>0}$ as a topological group under multiplication.
If~$K$ is a number field, then $c$ is surjective.
\end{lemma}
\begin{proof}
That the content map~$c$ satisfies the axioms of a homomorphisms
follows from the multiplicative nature of the defining formula
for~$c$.  For continuity, suppose $(a,b)$ is an open interval
in $\R_{>0}$.  Suppose $\bx\in\II_K$ is such that $c(\bx)\in (a,b)$.
By considering small intervals about each non-unit component of
$\bx$, we find an open neighborhood $U\subset \II_K$ of $\bx$
such that $c(U)\subset (a,b)$.  It follows the $c^{-1}((a,b))$
is open.

For surjectivity, use that each archimedean valuation is surjective,
and choose an idele that is~$1$ at all but one archimedean valuation.
\end{proof}
\begin{remark}
Note also that the $\II_K$-topology is that appropriate to a
group of operators on $\AA_K^+$: a basis of open sets
is the $S(C,U)$, where $C,U\subset \AA_K^+$
are, respectively, $\AA_K$-compact and $\AA_K$-open, and
$S$ consists of the $\bx\in \II_J$ such that
$(1-\bx) C\subset U$ and
$(1-\bx^{-1}) C\subset U$.
\end{remark}

\begin{definition}[$1$-Ideles]
  The subgroup $\II_K^1$ of \defn{$1$-ideles} is the subgroup of ideles
  $\bx=\{x_v\}$ such that $c(\bx)=1$.  Thus $\II_K^1$
is the kernel of $c$, so we have an exact sequence
$$
1 \to \II_K^1 \to \II_K \xra{c} \R_{>0}\to 1,
$$
where the surjectivity on the right is only if $K$
is a number field.
\end{definition}

\begin{lemma}\ilem{subset topology on $1$-ideles $\II_K^1$}
The subset $\II_K^1$ of $\AA_K$ is closed as a subset,
and the $\AA_K$-subset topology on $\II_K^1$ coincides
with the $\II_K$-subset topology on $\II_K^1$.
\end{lemma}
\begin{proof}
  Let $\bx\in\AA_K$ with $\bx\not\in\II_K^1$. To prove that $\II_K^1$
  is closed in $\AA_K$, we find an $\AA_K$-neighborhood $W$ of $\bx$
  that does not meet $\II_K^1$.

{\em 1st Case.}  Suppose that $\prod_v \abs{x_v}_v<1$ (possibly $=0$).
Then there is a finite set~$S$ of~$v$ such that
\begin{enumerate}
\item $S$ contains all the $v$ with $\abs{x_v}_v>1$, and
\item $\prod_{v\in S}\abs{x_v}_v < 1$.
\end{enumerate}
Then the set $W$ can be
defined by
\begin{align*}
  \abs{w_v - x_v}_v &< \eps \qquad v\in S\\
  \abs{w_v}_v &\leq 1 \qquad v\not\in S
\end{align*}
for sufficiently small $\eps$.

{\em 2nd Case.} Suppose that $C:=\prod_v \abs{x_v}_v>1$.
Then there is a finite set $S$ of $v$ such that
\begin{enumerate}
\item $S$ contains all the $v$ with $\abs{x_v}_v>1$, and
\item if $v\not\in S$ an inequality
$\abs{w_v}_v<1$ implies $\abs{w_v}_v < \frac{1}{2C}.$
(This is because for a non-archimedean valuation, the
largest absolute value less than $1$ is $1/p$, where $p$ is
the residue characteristic.  Also, the upper bound in
Cassels's article is $\frac{1}{2}C$ instead of $\frac{1}{2C}$,
but I think he got it wrong.)
\end{enumerate}
We can choose $\eps$ so small that
$\abs{w_v-x_v}_v<\eps$ (for $v\in S$) implies
$1<\prod_{v\in S} \abs{w_v}_v < 2C.$  Then $W$ may be defined
by
\begin{align*}
  \abs{w_v - x_v}_v &< \eps \qquad v\in S\\
  \abs{w_v}_v &\leq 1 \qquad v\not\in S.
\end{align*}
This works because if $\bw\in W$, then either
$\abs{w_v}_v=1$ for all $v\not\in S$, in
which case $1 < c(\bw)  < 2c$, so $\bw\not\in \II_K^1$,
or $\abs{w_{v_0}}_{v_0}<1$ for some $v_0\not\in S$, in
which case $$c(\bw) = \left(\prod_{v\in S} \abs{w_v}_v \right)\cdot
\abs{w_{v_0}} \cdots < 2C \cdot \frac{1}{2C} \cdots < 1,$$
so again $\bw\not\in\II_K^1$.

We next show that the $\II_K$- and $\AA_K$-topologies on $\II_K^1$
are the same.  If $\bx\in \II_K^1$, we must show that every
$\AA_K$-neighborhood of $\bx$ contains an $\II_K$-neighborhood
and vice-versa.

Let $W\subset \II_K^1$ be an $\AA_K$-neighborhood of $\bx$.  Then it
contains an $\AA_K$-neighborhood of the type
\begin{align}\label{eqn:adelicnbhd}
  \abs{w_v - x_v}_v &< \eps \qquad v\in S\\
  \abs{w_v}_v &\leq 1 \qquad v\not\in S
\end{align}
where $S$ is a finite set of valuations $v$.  This contains
the $\II_K$-neighborhood in which $\leq$ in (\ref{eqn:adelicnbhd})
is replaced by $=$.

Next let $H\subset \II_K^1$ be an $\II_K$-neighborhood.  Then it contains
an $\II_K$-neighborhood of the form
\begin{align}\label{eqn:adelicnbhd2}
  \abs{w_v - x_v}_v &< \eps \qquad v\in S\\
  \abs{w_v}_v &= 1 \qquad v\not\in S,
\end{align}
where the finite set~$S$ contains at least all archimedean
valuations~$v$ and all valuations~$v$ with
$\abs{x_v}_v\neq 1$.  Since $\prod\abs{x_v}_v=1$, we may also
suppose that $\eps$ is so small that (\ref{eqn:adelicnbhd2})
implies
$$
  \prod_v\abs{w_v}_v < 2.
$$
Then the intersection of  (\ref{eqn:adelicnbhd2}) with
$\II_K^1$ is the same as that of  (\ref{eqn:adelicnbhd})
with $\II_K^1$, i.e.,  (\ref{eqn:adelicnbhd2})
defines an $\AA_K$-neighborhood.
\end{proof}

By the product formula we have that $K^*\subset \II_K^1$.
The following result is of vital importance in class
field theory.
\begin{theorem}\label{thm:compquo}\ithm{compact quotient of ideles}
The quotient $\II_K^1/K^*$ with the quotient topology
is compact.
\end{theorem}
\begin{proof}
After the preceeding lemma, it is enough to find
an $\AA_K$-compact set $W\subset \AA_K$ such that the map
$$
  W\meet \II_K^1 \to \II_K^1/K^*
$$
is surjective.  We take for $W$ the set of
$\bw=\{w_v\}_v$ with
$$
  \abs{w_v}_v\leq \abs{x_v}_v,
$$
where $\bx=\{x_v\}_v$ is any idele of content greater than
the $C$ of Lemma~\ref{lem:bignorm}.

Let $\by=\{y_v\}_v\in \II_K^1$.  Then the content of $\bx/\by$ equals
the content of $\bx$, so by Lemma~\ref{lem:bignorm}
there is an $a\in K^*$ such that
$$
  \abs{a}_v \leq \abs{\frac{x_v}{y_v}}_v \qquad\text{all $v$}.
$$
Then $a\by\in W$, as required.
\end{proof}

\begin{remark}
  The quotient $\II_K^1/K^*$ is totally disconnected in the function
  field case.  For the structure of its connected component in the
  number field case, see papers of Artin and Weil in the ``Proceedings
  of the Tokyo Symposium on Algebraic Number Theory, 1955'' (Science
  Council of Japan) or \cite{artin-tate:cft}. The determination of the
  character group of $\II_K/K^*$ is global class field theory.
\end{remark}

\section{Ideals and Divisors}
Suppose that $K$ is a finite extension of $\Q$.  Let $F_K$ be the the
free abelian group on a set of symbols in bijection with the
non-archimedean valuation~$v$ of $K$.  Thus an element of $F_K$
is a formal linear combination
$$
  \sum_{v\text{ non arch.}} n_v \cdot v
$$
where $n_v\in\Z$ and all but finitely many $n_v$ are $0$.

\begin{lemma}\ilem{fractional ideals and formal sums of valuations}
  There is a natural bijection between $F_K$ and the group of nonzero
  fractional ideals of $\O_K$.  The correspondence is induced by
  $$ v\mapsto \wp_v = \{x \in \O_K : v(x)<1\},$$
where $v$ is a non-archimedean valuation.
\end{lemma}


Endow $F_K$ with the discrete topology.  Then there is a natural
continuous map $\pi:\II_K \to F_K$ given by
$$
\bx = \{x_v\}_v \mapsto \sum_v \ord_v(x_v)\cdot v.
$$
This map is continuous since the inverse image of
a valuation $v$ (a point) is the product
$$
\pi^{-1}(v) = \pi \O_v^* \quad \times
\prod_{w \text{ archimedean}} K_w^*\quad
\times \prod_{w\neq v \text{ non-arch.}} \O_w^*,
$$
which is an open set in the restricted product
topology on $\II_K$.
Moreover, the image of $K^*$ in $F_K$ is the group of nonzero
principal fractional ideals.

Recall that the \defn{class group} $C_K$ of the number field $K$
is by definition the quotient of $F_K$ by the image of $K^*$.

\begin{theorem}\label{thm:classgrpfin}\ithm{finiteness of class group}
The class group $C_K$ of a number field $K$ is finite.
\end{theorem}
\begin{proof}
  We first prove that the map $\II_K^1\to F_K$ is surjective.  Let
  $\infty$ be an archimedean valuation on $K$.  If $v$ is a
  non-archimedean valuation, let $\bx\in \II_K^1$ be a $1$-idele such
  that $x_w=1$ at ever valuation~$w$ except~$v$ and~$\infty$.  At~$v$,
  choose $x_v = \pi$ to be a generator for the maximal ideal of
  $\O_v$, and choose $x_\infty$ to be such that $\abs{x_\infty}_\infty
  = 1/\abs{x_v}_v$.  Then $\bx\in \II_K$ and $\prod_{w}\abs{x_w}_w =
  1$, so $\bx\in\II_K^1$.  Also $\bx$ maps to $v \in F_K$.

  Thus the group of ideal classes is the continuous image of the
  compact group $\II_K^1/K^*$ (see Theorem~\ref{thm:compquo}), hence
  compact.  But a compact discrete group is finite.
\end{proof}

\subsection{The Function Field Case}

When $K$ is a finite separable extension of $\F(t)$, we define the
divisor group $D_K$ of $K$ to be the free abelian group on all the
valuations~$v$.  For each $v$ the number of elements of the residue class
field $\F_v = \O_v/\wp_v$ of $v$ is a power, say $q^{n_v}$, of the number~$q$
of elements in $\F$.  We call $n_v$ the degree of $v$, and similarly
define $\sum n_v d_v$ to be the degree of the divisor $\sum n_v\cdot v$.
The divisors of degree $0$ form a group $D_K^0$.
As before, the principal divisor attached to $a\in K^*$  is
$\sum \ord_v(a) \cdot v \in D_K$.
The following theorem is proved in the same way as Theorem~\ref{thm:classgrpfin}.
\begin{theorem}\label{thm:finclassgrpff}\ithm{finiteness of function field class group}
The quotient of $D_K^0$ modulo the principal divisors is
a finite group.
\end{theorem}

\subsection{Jacobians of Curves}
For those familiar with algebraic geometry and algebraic curves, one
can prove Theorem~\ref{thm:finclassgrpff} from an alternative point of
view.  There is a bijection between nonsingular geometrically
irreducible projective curves over $\F$ and function fields $K$ over
$\F$ (which we assume are finite separable extensions of $\F(t)$ such
that $\Fbar\meet K = \F$).  Let $X$ be the curve corresponding to $K$.
The group $D_K^0$ is in bijection with the divisors of degree $0$ on
$X$, a group typically denoted $\Div^0(X)$.  The quotient of
$\Div^0(X)$ by principal divisors is denoted $\Pic^0(X)$.  The {\em
  Jacobian} of $X$ is an abelian variety $J=\Jac(X)$ over the finite
field $\F$ whose dimension is equal to the genus of $X$.  Moreover,
assuming $X$ has an $\F$-rational point, the elements of $\Pic^0(X)$
are in natural bijection with the $\F$-rational points on~$J$.  In
particular, with these hypothesis, the class group of $K$, which is
isomorphic to $\Pic^0(X)$, is in bijection with the group of
$\F$-rational points on an algebraic variety over a finite field.
This gives an alternative more complicated proof of finiteness of the
degree $0$ class group of a function field.

Without the degree $0$ condition, the divisor class group won't be finite.  It
is an extension of~$\Z$ by a finite group.
$$
  0 \to \Pic^0(X) \to \Pic(X) \xra{\deg} n\Z \to 0,
$$
where~$n$ is the greatest common divisor of the degrees of
elements of $\Pic(X)$, which is $1$ when $X$ has a rational
point.
\end{ch}
%%%%%%%%%%%%%%%%%%%%%%%%%%%%%%%%%%%%%%%%%%%%%%%%%%%%%%%%%%%%%%%%%%%%%%%%%%%%%%%
%%%%%%%%%%%%%%%%%%%%%%%%%%%% END IDELES %%%%%%%%%%%%%%%%%%%%%%%%%%%%%%%%%%%%%%%
%%%%%%%%%%%%%%%%%%%%%%%%%%%%%%%%%%%%%%%%%%%%%%%%%%%%%%%%%%%%%%%%%%%%%%%%%%%%%%%



































\bibliography{biblio}

\end{document}
